\documentclass[12pt]{beamer}

\usepackage[utf8]{inputenc}
\usepackage[T1]{fontenc}
\usepackage{amsmath}
\usepackage{caption}
\usepackage{sidecap}
\usepackage{animate}
\usepackage{textcomp, xspace}
\usepackage{listings}
\usepackage[pdflatex]{graphicx}

\usetheme{Goettingen}
\usecolortheme{whale}
\setbeamerfont{caption}{size={\fontsize{8pt}{10pt}}}
\setbeamerfont{note page}{size=\fontsize{6pt}{8pt}}
\setbeamerfont{footnote}{size=\fontsize{6pt}{6pt}}

\newcommand{\intro}{\underline{\textbf{INTRO:}}}
\definecolor{cBg}{RGB}{240,240,250}
\definecolor{cCm}{RGB}{0, 105, 0}
\definecolor{cSt}{RGB}{155, 0, 0}
\definecolor{cKw}{RGB}{0, 0, 105}
\definecolor{pBg}{RGB}{255,220,205}
\renewcommand{\lstlistingname}{Algorithm}
\lstset{
	basicstyle=\rmfamily,
    mathescape,
    frame=lines}

\title{How to Program a Calculator}
\date{}
\subtitle{A look at the geometric cosine method}
\author{Jake Darby}

\begin{document}

\section{Introduction}

\begin{frame}
\titlepage
\end{frame}

\note[itemize]{
\item Welcome, hopefully quick so we can all get our lunch
\item Jake Darby and as can be seen looking at how to program a calculator
\item Not got time for everything, so looking in a bit of detail at one method of calculating cos}


\begin{frame}
\frametitle{Project Overview}
\begin{itemize}[<+(1)->]
	\item Numerical Analysis of Common Functions
	\begin{itemize}[<+(1)->]
		\item Appoximations
		\item Errors
		\item Implementation
	\end{itemize}
	\item Functions examined have been studied since antiquity
	\begin{itemize}[<+(1)->]
		\item Square Roots
		\item Trigonometric Functions
		\item Logarithms and Exponentials
	\end{itemize}
	\item These functions can be complex to approximate well
	\item Implemented by modern computers and calculators
	\begin{itemize}[<+(1)->]
		\item Implemented in C
		\item GNU \& MPFR libraries
	\end{itemize}
\end{itemize}
\end{frame}

\note[itemize]{
	\item \intro like to give v. brief overview of whole project
	\item >Project deals with Numerical Analysis of Common Functions
	\item This analysis includes
	\item >How the functions are approximated
	\item Typically create an algorithm i.e. process which will approximate the correct function value
	\item >The error of the functions e.g. absolute error
	\item Examine how to calculate or approximate these errors
	\item >The implementation of the algorithm
	\item Includes computational complexity of algorithm, often can be improved with analysis
	\item >The functions I examined in my project have typically been studied since antiquity
	\item In particular I studied >Square Root functions, >Trig functions and >Logarithmic and exponential functions
	\item >The common feature of these functions is that they can all be complex to approximate well
	\item This has been a problem as these functions have many useful practical applications.
	\item >These functions are implemented in modern computers and calculators
	\item The aim of this project is to understand how they are implemented
	\item >I implemented in C for speed
	\item >used Gnu Multiple Precision (GNU) and Gnu MPFR (MPFR)
	\item Allows arbitrary precision in calculations
	\item Allowed me to calculate \(\sqrt{2}\) to 1 million places in about 1 sec}

\begin{frame}
\frametitle{Trigonometric Functions}
\begin{itemize}[<+(1)->]
	\item Looked at \(\sin\), \(\cos\) and \(\tan\)
	\begin{itemize}[<+(1)->]
		\item Studied as far back as the Egyptians
		\item Use Trigonometric Identities to reduce the problem
	\end{itemize}
	\item Analysed several different methods
	\begin{itemize}[<+(1)->]
		\item Each of the methods have their benefits
		\item Taylor
		\item CORDIC
		\item Geometric
	\end{itemize}
\end{itemize}
\end{frame}

\note[itemize]{
	\item \intro The section we're looking at is the Trigonometric Fnctns
	\item >In particular this section deals with sin, cos and tan
	\item Many uses over the years, particularly in architecture
	\item >As such studied since Ancient Egypt
	\item Used in the building of the pyramids
	\item Possibly used further back by the Babylonians, but that is debated
	\item >Typically use trig idents to reduce the problem
	\item If we find sin or cos, we can find all the others if we know \(\pi\)
	\item >In the course of the project I analysed several differnt methods
	\item >Each method works in a slightly different way with different porperties, which gives different benefits
	\item >The Taylor method uses a series to approximate the values
	\item These are easy to calculate and analyse
	\item >The Coordinate Rotation Digital Computer (aka CORDIC) uses rotations of unit vectors via Matrix Transformations
	\item Has applications in very basic compute systems
	\item Can also be implemented in hardware
	\item >The final method is the one we'll be looking at in more detail here}

\section{Derivation}

\begin{frame}
\frametitle{Geometric Method Figure 1}
\begin{figure}
	\centering
	\includegraphics[width=0.8\textwidth]{"./Diagrams/Geometric Trig Diagram 1"}
	\caption{First diagram in developing the geometric method}
\end{figure}
\end{frame}

\note[itemize]{
	\item I will be briefly demonstrating how the geometric method is constructed in the following slides.
	\item This is a sectn of a circle with radius 1
	\item Detailed are the useful values that we'll be using on this diagram, which we'll go over in detail on the next slide.
	\item note that only consider \(\theta \in [0, \tfrac{\pi}{2}\) as other values can be reduced to this range via trig idents}

\begin{frame}
\frametitle{Geometric Method Derivation 1}
\begin{columns}
	\begin{column}{0.5\textwidth}
		\begin{itemize}[<+(1)->]
			\item \(s^2 = \sin^2\theta + (1 - \cos\theta)^2\)
			\item \(s^2 = 2 - 2\cos\theta\)
			\item \(\cos\theta = 1 - \tfrac{1}{2}s^2\)
			\item \(s \approx \theta\)
			\item We wish to improve our approximation of \(s\)
		\end{itemize}
	\end{column}
	\begin{column}{0.5\textwidth}
		\begin{figure}
			\centering
			\includegraphics[width=\textwidth]{"./Diagrams/Geometric Trig Diagram 1"}
		\end{figure}
	\end{column}
\end{columns}
\end{frame}

\note[itemize]{
	\item \intro In analysing this we start by looking at theright hand trinagle
	\item >This gives the this equation by using Pythagarus' Theorem
	\item By using the basic trig identity of s2 + c2 = 1 we get>
	\item Finally re-arranging we get >this which gives us a formula for cos if we know s
	\item As r = 1, then we know the arc of the angle is \(\theta\) in length
	\item The chord approximates this so > \(s \approx \theta\)
	\item >The approximation is fairly poor, so we need to improve our approximation of s}

\begin{frame}
\frametitle{Geometric Method Diagrams 2}
\begin{figure}
	\centering
	\includegraphics[width=0.7\textwidth]{"./Diagrams/Geometric Trig Diagram 2"}
	\caption{Diagram to show how to recursively calculate s}
\end{figure}
\end{frame}

\note[itemize]{
	\item This diagram will allow us to find a much better approximation for s, as detailed on the next slide
	\item In particular we note that h is the chord of half the original arc}

\begin{frame}
\frametitle{Geometric Method Derivation 2}
\begin{columns}
	\begin{column}{0.5\textwidth}
	\begin{itemize}[<+(1)->]
		\item \(ABC\) is right angled
		\item \(\begin{aligned}[t]AB &= \sqrt{AC^2 - BC^2}\\ &= \sqrt{4-h^2}\end{aligned}\)
		\item Area of ABC:
		\begin{itemize}[<+(1)->]
			\item \(\tfrac{1}{2}h\sqrt{4-h^2}\)\medskip
			\item \(\tfrac{1}{2}\cdot2\cdot\tfrac{s}{2}\)
		\end{itemize}
		\item \(s^2 = h^2(4 - h^2)\)
		\item \(h \approx \tfrac{\theta}{2}\)
	\end{itemize}
	\end{column}
	\begin{column}{0.5\textwidth}
		\begin{figure}
			\centering
			\includegraphics[width=\textwidth]{"./Diagrams/Geometric Trig Diagram 2"}
		\end{figure}
	\end{column}
\end{columns}
\end{frame}

\note[itemize]{
	\item Now in considering this diagram, first observation >ABC Right Angled by elementary geometry
	\item Then by Pythagarus we get >
	\item >To tie it all together we consider the area of ABC
	\item By using BC as the base of the triangle we get >
	\item Also by using AC as the base of the triangle we get >
	\item >We get this by equating the two sides and simplifying
	\item Thus if we know h accurately we can calculate \(s^2\)
	\item >\(h \approx \tfrac{\theta}{2}\)
	\item repeating the process would give a a new \(h \approx \tfrac{\theta}{4}\), then \(\tfrac{\theta}{8}\), etc...}

\begin{frame}[fragile]
\frametitle{Geometric Method Algorithm}
\begin{lstlisting}
  geometric_cos($\theta \in [0, \tfrac{\pi}{2}], k \in \mathbb{N}$):
      $h_0 := \theta2^{-k}$
      $n := 0$
      while $n < k$:
          $h_{n+1}^2 := h_n^2\cdot(4-h_n^2)$
          $n \mapsto n + 1$
      return $1 - \tfrac{1}{2}h_k^2$
\end{lstlisting}
\end{frame}

\note[itemize]{
	\item Thus we get this algorithm
	\item starts with the approximation that \(h_0 := \theta2^{-k}\)
	\item Uses the recursive formula, that I mentioned on the previous slide
	\item Finally \(h_k = s\)}

\begin{frame}[fragile]
\lstset{basicstyle=\ttfamily,
        language=C,
		backgroundcolor=\color{cBg},
		basicstyle=\footnotesize,
		commentstyle=\color{cCm},
		frame=L,
		keywordstyle=\color{cKw},
		showstringspaces=false,
		stringstyle=\color{cSt},
		tabsize=2}
\frametitle{Geometric Method Implementation}
\begin{lstlisting}
#include <assert.h>
#include <math.h>

double geometric_cos(double theta, 
                     unsigned int k){
   //k > 0
   assert(k);
   //0 <= theta < pi/2
   assert(0 <= theta < 1.57079632679);

   double h = theta * pow(2, -k);
   h *= h;

   for(int i = 0; i < k; ++i)
      h = h * (4 - h);
	
   return 1 - h/2;
}
\end{lstlisting}
\end{frame}

\note[itemize]{
	\item Simple implementation in C
	\item checks bounds with assert commands
	\item only uses one \(h\) variable which is updated}

\section{Errors}

\begin{frame}
\frametitle{Basics and Assumptions}
\begin{itemize}[<+(1)->]
	\item Errors occur due to the assumption that \(h_0 = \theta2^{-k}\)
	\item We are concerned here with the absolute error
	\begin{itemize}[<+(1)->]
		\item If \(\bar{x} \approx x\), then the absolute error is: 
	\end{itemize}
	\onslide<4->\begin{displaymath}\epsilon_x := |x - \bar{x}|\end{displaymath}
	\item Assumptions:
	\begin{itemize}[<+(1)->]
		\item All calculations are performed without error
		\item All calculations are performed to arbitrary precision
	\end{itemize}
\end{itemize}
\end{frame}

\note[itemize]{
	\item There is only one real source of error
	\item >Lies in our assumption \(h_0 = \theta2^{-k}\)
	\item Actually \(h_0\) is close, but not equal.
	\item Error measure we will be considering is >absolute error
	\item >That is that if \(\bar{x}\) approximates \(x\)
	\item Then the absolute error is \(\epsilon_x\)
	\item >All of what follows relies on the following assumptions
	\item >All calculations are performed correctly
	\item That is no errors are ever made in the calculations
	\item >All the calculation are performed to arbitrary precision
	\item That is we can be working accurately with any number of decimal places
	\item This is not the case in most computer systems, which only have a finite precision}

\begin{frame}
\frametitle{Error Analysis 1}
\begin{itemize}[<+(1)->]
	\item Two important propositions:
	\begin{itemize}[<+(1)->]
		\item 4.3.1: \(h_n = 2\sin(2^n\sin^{-1}(\theta2^{-k-1}))\)
		\item 4.3.2: \(h_n > 2\sin(\theta2^{n-k-1})\)
	\end{itemize}
	\item Proposition 4.3.3:
	\begin{itemize}[<+(1)->]
		\item \(\epsilon_n = |h_n - 2\sin(\theta2^{n-k-1})|\)
		\item \(\epsilon_k < 2^k\epsilon_0\)
		\item Proven by showing \(\epsilon_{n+1} < 2\epsilon_n\)
		\item Uses simple trigonometry and algebraic re-arrangement
	\end{itemize}
\end{itemize}
\end{frame}

\note[itemize]{
	\item >When analysing the error of the method, I proved two propositions
	\item >The first is proposition 4.3.1
	\item This states that \(h_n\) is the length of an arc with angle related to \(\theta\).
	\item Used as a lemma in other proofs
	\item Simple to prove via induction
	\item >The second is that \(h_n > 2\sin(\theta2^{n-k-1})\)
	\item This is proven using the taylor expansion of \(\sin\)
	\item This allows us to conclude that \(h_n\) is always an overestimate
	\item This result allows us to then prove >propsition 4.3.3
	\item >If we define \(\epsilon_n\) as shown on the screen
	\item >Then \(\epsilon_k < 2^k\epsilon_0\)
	\item >To prove this I showed that \(\epsilon_{n+1} < 2\epsilon_n\)
	\item >Which was done via simple trig and re-arangement
	\item Now we would like to know if this is a useful error measure
	\item i.e. the error goes to 0 as k goes to infinity}

\begin{frame}
\frametitle{Error Analysis 2}
\begin{itemize}[<+(1)->]
	\item \(\epsilon_k = h_k - s\)
	\item \(\epsilon_0 = \theta2^{-k} - 2\sin(\theta2^{-k-1})\)
	\item Let \(\mathcal{C} := 1 - \tfrac{1}{2}h_k^2\)
	\item Let \(\epsilon_{\mathcal{C}} := |\mathcal{C} - \cos{\theta}|\)
	\item We can show that \(\epsilon_{\mathcal{C}} < 2\epsilon_k\)
	\item Thus \(\begin{aligned}[t]\epsilon_{\mathcal{C}} &< 2^{k+1}\epsilon_0\\&= 2\theta - 2^{k+2}\sin(\theta2^{-k-1})\end{aligned}\)
	\item \small\(2^{k+2}\sin(\theta2^{-k-1}) = 2\theta - \tfrac{1}{6}\theta^32^{-2k-1} + \tfrac{1}{120}\theta^52^{-4k-1} + \cdots\)\normalsize
	\item \(\epsilon_{\mathcal{C}} < \frac{1}{6}\theta^32^{-2k-1} + \mathcal{O}(2^{-4k-1})\)
\end{itemize}
\end{frame}

\note[itemize]{
	\item We will surmize that > \(\epsilon_k = h_k-s\) and > \(\epsilon_0 = \theta2^{-k} - 2\sin(\theta2^{-k-1})\)
	\item Also we consider >script C to be the approximation of cos
	\item >and thusly \(\epsilon\) script C to be the absolute error of the algorithm
	\item It fairly easy to show that \(\epsilon_\mathcal{C} < 2 \epsilon_k\)
	\item >Therefore we get this final inequality for \(\epsilon_\mathcal{C}\)
	\item the equality is from substituting in \(\epsilon_0\)
	\item It is still not obvious that limit of ep C = 0 as k to infty
	\item By using the taylor expansion of sin again we get >this
	\item Substituing this into the previous we get >this equation
	\item as this is \(\mathcal{O}(2^{-2k-1})\) then it is obvious the RHS goes to 0 as k goes to infty
	\item Hence lim ep C = 0 as k to infty, and we conclude the method correctly converges}

\section{Conclusion}

\begin{frame}
\frametitle{Digits of Accuracy}
\begin{itemize}[<+(1)->]
	\item \(2\theta - 2^{k+2}\sin(\theta2^{-k-1}) < 10^{-N}\)
	\begin{itemize}[<+(1)->]
		\item Guarantee \(N\) digits of accuracy
		\item Must solve \(2^{k+2}\sin(\theta2^{-k-1}) > 2\theta - 10^{-N}\)
	\end{itemize}
	\item Use a test value of \(\theta = 0.5\)
\end{itemize}
\onslide<6->\begin{figure}
\centering
\begin{tabular}{|p{50pt}|p{50pt}|}
	\hline
	\(N\) & \(k\)\\
	\hline
	5 & 6 \\\hline
	10 & 14\\\hline
	50 & 80\\\hline
	100 & 163\\\hline
	1000 & 1658\\\hline
\end{tabular}
\caption{This table shows the minimum \(k\) required to guarantee \(N\) digits of accuracy for our approximation of \(\cos(0.5)\)}
\end{figure}
\end{frame}

\note[itemize]{
	\item >If we have k which validates this inequality then we know that ep C is also less than \(10^{-N}\)
	\item >This will guarantee us N decimal places of accuracy
	\item The actual accuracy may be more than this
	\item >Equivalent to finding k s.t. this eqn is satisfied
	\item >Using a test value of 0.5, I found solutions to this
	\item >This table details the minium k to get N digits of accuracy
	\item We can see that the growth of k required is roughly linear
	\item We could use this data to calculate a sufficient k, given a required N digits
	\item For example if we are told we need 10,000 digits we could use k = 18,000 to be sufficient.}

\begin{frame}
\frametitle{Final Remarks}
\begin{itemize}[<+(1)->]
	\item Interesting method, but better exist
	\begin{itemize}[<+(1)->]
		\item e.g. Taylor Series method
	\end{itemize}
	\item Fairly trivial to reverse the algorithm to find \(\cos^{-1}\)
	\item Just one method, in one section
	\begin{itemize}[<+(1)->]
		\item Digit by digit square root
		\item Hardware implementable trig calculations
		\item Continued fractions for exponentials
	\end{itemize}
\end{itemize}
\end{frame}

\note[itemize]{
	\item >While this method interests me and I wished to share it with you better methods do exist for approximating cos
	\item >For example the Taylor Series Method works better
	\item However it is relatively uninteresting for a presentation
	\item >One observation is that if you consider the algorithm presented, it is fairly easy to reverese
	\item Thus if you start with a value for \(\cos\theta\) you can approximate \(\cos^{-1}\theta\)
	\item >This method presented here is just one of many in the project
	\item Some other interesting methods I discuss include
	\item >A square root method that gives precisely the next digit of the root each time in sequence
	\item >A method of calculating trig values that can be implemented in circuitary hardware
	\item Uses the CORDIC algorithm, which isn't as accurate, but is very fast
	\item >A method of using continued fractions to approximate natural exponentials}

\begin{frame}
\frametitle{\null}
\centering
{\huge Thank you for listening}\\\vspace{20pt}\pause
{\large Project at: https://github.com/Ybrad/Year-4-Project}\\\vspace{20pt}\pause
{\large Any questions?}
\end{frame}

\note[itemize]{
	\item I hope this presentation was interesting for some of you
	\item >If you wish to view my project in it's entirety you can visit this github repository
	\item This contains all the latex files, pdfs, images and code used
	\item >Now any questions?}

\end{document}
