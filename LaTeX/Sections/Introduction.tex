%SEC%
\section{Introduction}
\label{SEC_"Introduction"}
For many thousands of years all calculations that a person might want performing had to be done by hand. For simple calculations such as addition, subtraction and multiplication this was not such an issue, but as society evolved we wanted to know the answer to increasingly hard questions. The Greeks' sought to find a value for $\pi$, and ended up with the bounds that $\frac{223}{71} < \pi < \frac{22}{7}$\cite{ART_ArchPi}\cite[][106]{BOK_CalcBook}, which while sufficient for their needs is not sufficient for ours in the present. \\

At the same time many functions were being studied to find solutions, often arising from practical concerns. For instance finding the square root of any arbitrary number has been important to architects since the time of the ancient Babylonian mathematics\cite{ONL_BabylAnly}. Similarly relevant have been the periodic trigonometric functions due to their relation to triangles, and exponential functions due to their use in finance for example to find interest on loans.\\

The difficulty of these methods is that there is typically no simple way of getting an exact answer, if in fact one is available. Over time methods were developed that would allow a person to calculate an approximate answer to their problem, given enough time and patience. Such methods tended to be long and tedious work, which even lead to the profession of a human computer from the early 17\textsuperscript{th} century until the 20\textsuperscript{th} century; who would be hired for that purpose.\\

By the time of the Renaissance period people had started to build early mechanical calculators to help in these endeavours. Such calculators were typically capable of only addition and subtraction, which could be used to implement multiplication and division if one so wished. Later these machines became more elaborate, capable of multiple simple functions, or designed to perform one more complicated function. A famous example is Charles Babbage's difference engine\cite{ONL_Babbage} which was a large mechanical calculator that would tabulate polynomial functions developed in the early 1800s.\\

Eventually in the 20\textsuperscirpt{th} century electronic computers were created and soon replaced both mechanical and human calculators. Such electronic machines had many benefits over both their human and mechanical counterparts, and soon it became common place to use electronic computers to perform mathematical computations. Today computers have become faster and smaller, and the average person's phone outstrips the entire computing power of NASA during the Apollo missions.\\

However despite the speed of the calculations these modern computers still need to be instructed in how to evaluate the functions asked of it. This document will take some common functions that any calculator will answer in the blink of an eye accurate to around 10 significant digits, and explore how they may be computed. In particular this document will be comparing the speed at which these computations can be performed versus the accuracy of their results.\\

%SUB%
\subsection{Code and Computers used}
\label{SUB_"Code and Computers used"}
During this project I will be discussing the implementation of various algorithms. I will be implementing these algorithms in the C programming language, using the C11 standard.\\

I chose the C programming language to implement my algorithms in, because once it compiles to binary machine code, the programs produced tend to be very efficient. This is partly due to the low-level of C programming, having relatively close control over direct CPU actions; however this does come at the cost of losing higher functionality that many other programming languages offer. A second reason for the efficiency is due to C's long history, originally being developed in 1969-1970, which has lead to several very efficient compilers being developed.\\

I will be implementing most programs using C's built in primitive types, typically \codeinline{int}, \codeinline{unsigned int} and \codeinline{double}. On a computer an \codeinline{int} is an integer that can represent both positive and negative bits using twos compliment, this gives an \codeinline{int} using \(n\) bits a minimum value of \(-2^n\) and a maximum value of \(2^n-1\). Typically a computer will store an \codeinline{int} as 32 bits, though some computers may use more or less bits. An \codeinline{unsigned int} is very similar to an \codeinline{int}, but does not represent negative values, and thus an \codeinline{unsigned int} of \(n\) bits has a minimum value of \(0\) and a maximum value of \(2^{n+1}-1\).\\

If an integer of a specific number of bits is needed then the header \codeinline{stdint.h} may be used which defines \codeinline{int\_N} and \codeinline{uint\_N} which respectively represent \codeinline{int} of N bits and \codeinline{unsigned int} of N bits; The typical values of N are 8, 16, 32 and 64.\\

In C a \codeinline{double} is a floating point representation of a real value, that typically follows the IEEE 754 standard\cite{MAN_ieee754} for double-precision binary floating points. This standard has:
\begin{itemize}
\item 1 bit for the sign of the number, \(s\)
\item 11 bits for the exponent, \(e\)
\item 52 bits for the significand, \(b = b_0b_1b_2\dots b_{51}\)
\item A value that is understood to be:
	\[(-1)^s\left(1 + \sum_{i=1}^{52}b_{52-i}2^{-i}\right) \times 2^{e-1023
}\]
\end{itemize}

This gives a \codeinline{double} value a precision of around 15-17 significant decimal digits. While this is good for most applications, there are some applications where we may desire even more precision than this. To solve this I will be implementing certain algorithms using the GNU Multiple Precision Arithmetic Library\cite{ONL_gmp} (referred to as GMP) as well as GNU MPFR Library\cite{ONL_mpfr} (referred to as MPFR), which was built upon GMP to correct and optimise the original. These libraries allow the use of arbitrary precision real values, given enough memory space, as well as integers longer than C's standard integer types can hold.\\

An important point to note that will be useful later on is that due to the storage structure of C's \codeinline{double}s and the MPFR \codeinline{mpfr\_t}s which also use a floating point representation. In the storage of the significand both data types work such that the value of \(b\) is in the range \([\frac{1}{2}, 1)\). This is useful as it means that if we have a stored value \(x\), then it is very easy to extract \(\alpha\in[\frac{1}{2}, 1), \beta \in \Z\) such that \(x = \alpha\cdot2^\beta\); an operation that would usually be equivalent to calculating the non-trivial \(\log_2(x)\). The value of this observation will be in restricting the range over which functions need to be evaluated later in the document.\\

I will be compiling and testing all of my code on a benchmark machine running a light version of Ubuntu 14.04, using the GNU C Compiler. The specifications of the machine, that may impact performance are:
\begin{itemize}
\item An Intel i5-4690K processor running at 4GHz. 
\begin{itemize}
	\item This processor uses a 64 bit architecture.
\end{itemize}
\item 8Gb of DDR3 RAM
\item A modern chipset on the motherboard
\end{itemize}
