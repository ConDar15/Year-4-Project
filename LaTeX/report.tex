\documentclass[12pt]{article}

\usepackage[utf8]{inputenc}
\usepackage{amsthm}
\usepackage{amsmath}
\usepackage{amssymb}
\usepackage[a4paper, margin=2.5cm]{geometry}
\usepackage{natbib}
\usepackage{chngcntr}
\usepackage{color}
\usepackage{listings}
\usepackage{caption}
\usepackage{graphicx}

\definecolor{cBg}{RGB}{240,240,250}
\definecolor{cCm}{RGB}{0, 105, 0}
\definecolor{cSt}{RGB}{155, 0, 0}
\definecolor{cKw}{RGB}{0, 0, 105}

\newcommand{\TODO}[1]{\textcolor{red}{TODO: #1}}
\newcommand{\N}{\mathbb{N}}
\newcommand{\R}{\mathbb{R}}
\newcommand{\Rp}{\mathbb{R}^+}
\newcommand{\Rpz}{\mathbb{R}^+_0}
\newcommand{\Z}{\mathbb{Z}}
\newcommand{\Zp}{\mathbb{Z}^+}
\newcommand{\Zn}{\mathbb{Z}^-}
\newcommand{\Zpz}{\mathbb{Z}^+_0}
\newcommand{\bigO}{\mathcal{O}}
\newcommand{\codeinline}[1]{\colorbox{cBg}{\texttt{#1}}}

\newenvironment{subproof}[1][\proofname]{%
	\renewcommand{\qedsymbol}{$\blacksquare$}%
	\begin{proof}[#1]%
	}{\end{proof}}

\newenvironment{codelisting}[1]{%
\lstset{basicstyle=\ttfamily,%
		language=C,%
		backgroundcolor=\color{cBg},%
		basicstyle=\footnotesize,%
		commentstyle=\color{cCm},%
		frame=L,%
		keywordstyle=\color{cKw},%
		showstringspaces=false,%
		stringstyle=\color{cSt},%
		tabsize=2%
		caption=#1}%
\renewcommand{\lstlistingname}{Implementation}%
\begin{lstlisting}}{}

\setlength{\parindent}{0pt}
\counterwithin{figure}{subsection}
%\counterwithin{lstlisting}{subsubsection}
\renewcommand{\familydefault}{\sfdefault}
\renewcommand{\lstlistingname}{Algorithm}
\lstset{basicstyle=\rmfamily,mathescape}
\bibpunct{[}{]}{;}{s}{,}{,}
\numberwithin{equation}{subsection}

\AtBeginDocument{%
	\counterwithin*{lstlisting}{section}
	\counterwithin*{lstlisting}{subsection}
	\counterwithin*{lstlisting}{subsubsection}
	\renewcommand{\thelstlisting}{%
		\ifnum\value{subsection}=0
			\thesection.\arabic{lstlisting}%
		\else
			\ifnum\value{subsubsection}=0
				\thesubsection.\arabic{lstlisting}%
			\else
				\thesubsubsection.\arabic{lstlisting}%
			\fi
		\fi
	}%
}

%\includeonly{./Sections/Introduction}
\includeonly{./Sections/Trigonometric_Functions}
%\includeonly{./Sections/Root_Functions}
%\includeonly{}

\begin{document}

\author{Jake Darby}
\title{Accuracy vs Efficiency of Numerical Methods \\ \large How to program a Calculator}
\date{}
\maketitle

\begin{abstract}
\begin{center}
This document will discuss and analyse various numerical methods for computing functions commonly found on calculators. The aim of this paper is to, for each set of functions, compare and contrast several algorithms in regards to their effiency and accuracy.
\end{center}
\end{abstract}

\iffalse
\newpage
\tableofcontents
\newpage
\fi

%SEC%
\section{Introduction}
\label{SEC_"Introduction"}
For many thousands of years all calculations that a person might want performing had to be done by hand. For simple calculations such as addition, subtraction and multiplication this was not such an issue, but as society evolved we wanted to know the answer to increasingly hard questions. The Greeks' sought to find a value for $\pi$, and ended up with the bounds that $\frac{223}{71} < \pi < \frac{22}{7}$\cite{ART_ArchPi}\cite[][106]{BOK_CalcBook}, which while sufficient for their needs is not sufficient for ours in the present. \\

At the same time many functions were being studied to find solutions, often arising from practical concerns. For instance finding the square root of any arbitrary number has been important to architects since the time of the ancient Babylonian mathematics\cite{ONL_BabylAnly}. Similarly relevant have been the periodic trigonometric functions due to their relation to triangles, and exponential functions due to their use in finance for example to find interest on loans.\\

The difficulty of these methods is that there is typically no simple way of getting an exact answer, if in fact one is available. Over time methods were developed that would allow a person to calculate an approximate answer to their problem, given enough time and patience. Such methods tended to be long and tedious work, which even lead to the profession of a human computer from the early 17\textsuperscript{th} century until the 20\textsuperscript{th} century; who would be hired for that purpose.\\

By the time of the Renaissance period people had started to build early mechanical calculators to help in these endeavours. Such calculators were typically capable of only addition and subtraction, which could be used to implement multiplication and division if one so wished. Later these machines became more elaborate, capable of multiple simple functions, or designed to perform one more complicated function. A famous example is Charles Babbage's difference engine\cite{ONL_Babbage} which was a large mechanical calculator that would tabulate polynomial functions developed in the early 1800s.\\

Eventually in the 20\textsuperscirpt{th} century electronic computers were created and soon replaced both mechanical and human calculators. Such electronic machines had many benefits over both their human and mechanical counterparts, and soon it became common place to use electronic computers to perform mathematical computations. Today computers have become faster and smaller, and the average person's phone outstrips the entire computing power of NASA during the Apollo missions.\\

However despite the speed of the calculations these modern computers still need to be instructed in how to evaluate the functions asked of it. This document will take some common functions that any calculator will answer in the blink of an eye accurate to around 10 significant digits, and explore how they may be computed. In particular this document will be comparing the speed at which these computations can be performed versus the accuracy of their results.\\

%SUB%
\subsection{Code and Computers used}
\label{SUB_"Code and Computers used"}
During this project I will be discussing the implementation of various algorithms. I will be implementing these algorithms in the C programming language, using the C11 standard.\\

I chose the C programming language to implement my algorithms in, because once it compiles to binary machine code, the programs produced tend to be very efficient. This is partly due to the low-level of C programming, having relatively close control over direct CPU actions; however this does come at the cost of losing higher functionality that many other programming languages offer. A second reason for the efficiency is due to C's long history, originally being developed in 1969-1970, which has lead to several very efficient compilers being developed.\\

I will be implementing most programs using C's built in primitive types, typically \codeinline{int}, \codeinline{unsigned int} and \codeinline{double}. On a computer an \codeinline{int} is an integer that can represent both positive and negative bits using twos compliment, this gives an \codeinline{int} using \(n\) bits a minimum value of \(-2^n\) and a maximum value of \(2^n-1\). Typically a computer will store an \codeinline{int} as 32 bits, though some computers may use more or less bits. An \codeinline{unsigned int} is very similar to an \codeinline{int}, but does not represent negative values, and thus an \codeinline{unsigned int} of \(n\) bits has a minimum value of \(0\) and a maximum value of \(2^{n+1}-1\).\\

If an integer of a specific number of bits is needed then the header \codeinline{stdint.h} may be used which defines \codeinline{int\_N} and \codeinline{uint\_N} which respectively represent \codeinline{int} of N bits and \codeinline{unsigned int} of N bits; The typical values of N are 8, 16, 32 and 64.\\

In C a \codeinline{double} is a floating point representation of a real value, that typically follows the IEEE 754 standard\cite{MAN_ieee754} for double-precision binary floating points. This standard has:
\begin{itemize}
\item 1 bit for the sign of the number, \(s\)
\item 11 bits for the exponent, \(e\)
\item 52 bits for the significand, \(b = b_0b_1b_2\dots b_{51}\)
\item A value that is understood to be:
	\[(-1)^s\left(1 + \sum_{i=1}^{52}b_{52-i}2^{-i}\right) \times 2^{e-1023
}\]
\end{itemize}

This gives a \codeinline{double} value a precision of around 15-17 significant decimal digits. While this is good for most applications, there are some applications where we may desire even more precision than this. To solve this I will be implementing certain algorithms using the GNU Multiple Precision Arithmetic Library\cite{ONL_gmp} (referred to as GMP) as well as GNU MPFR Library\cite{ONL_mpfr} (referred to as MPFR), which was built upon GMP to correct and optimise the original. These libraries allow the use of arbitrary precision real values, given enough memory space, as well as integers longer than C's standard integer types can hold.\\

An important point to note that will be useful later on is that due to the storage structure of C's \codeinline{double}s and the MPFR \codeinline{mpfr\_t}s which also use a floating point representation. In the storage of the significand both data types work such that the value of \(b\) is in the range \([\frac{1}{2}, 1)\). This is useful as it means that if we have a stored value \(x\), then it is very easy to extract \(\alpha\in[\frac{1}{2}, 1), \beta \in \Z\) such that \(x = \alpha\cdot2^\beta\); an operation that would usually be equivalent to calculating the non-trivial \(\log_2(x)\). The value of this observation will be in restricting the range over which functions need to be evaluated later in the document.\\

I will be compiling and testing all of my code on a benchmark machine running a light version of Ubuntu 14.04, using the GNU C Compiler. The specifications of the machine, that may impact performance are:
\begin{itemize}
\item An Intel i5-4690K processor running at 4GHz. 
\begin{itemize}
	\item This processor uses a 64 bit architecture.
\end{itemize}
\item 8Gb of DDR3 RAM
\item A modern chipset on the motherboard
\end{itemize}

%SEC%
\section{General Definitions and Theorems}
\label{SEC_"General Definitions and Theorems"}

This section will list some general definitions and theorems which will be used throughout the document. This will not be an exhaustive or in depth view of such concepts but merely an overview to allow easier reading of the material moving forwards.

%SUB%
\subsection{Methods}

In this document we will look at various functions, such as root functions and trigonometric functions, among others. Despite the variety of functions being analysed there are several methods that are useful for more than one function, or are worth analysing before their use.

%SUBSUB%
\subsubsection{Newton-Raphson Method}
\label{SUBSUB_"Newton-Raphson Method"}

\theoremstyle{definition}
\newtheorem{Newton Method}{Definition}[subsubsection]

The Newton-Raphson Method is named after Sir Isaac Newton and Joseph Raphson\cite[][84]{BOK_NumMeth}. It is a method that takes a continuously differentiable function \(f\) and it's derivative \(f'\), as well as an initial guess \(x_0\), to create successively more accurate solutions to \(x\) where \(f(x) = 0\).\\

The motivation of the method can be seen in figure \ref{FIG_"Newton-Raphson Demonstration"}, where we take an initial guess \(x_0\) of the root \(x^\ast\). The tangent to the curve above \(x_0\) is then found, and has the equation \(y = f'(x_0)(x-x_0) + f(x_0)\), by setting \(y = 0\) and solving for \(x\) we find \(x_1\). By repeating this process and starting from a good enough \(x_0\) we hope to find successively closer approximations to \(x^\ast\).\\

%FIG%
\begin{figure}[!ht]
	\caption{Demonstration of Newton-Raphson Method}
	\label{FIG_"Newton-Raphson Demonstration"}
	\centering
	\includegraphics[width=0.75\textwidth]{"./Diagrams/Newton-Raphson Diagram"}
\end{figure}

The specific definition of the Newton-Raphson method that I will be using in this document is below:

%DEF%
\begin{Newton Method}
\label{DEF_"Newton-Raphson Method"}
Given \(f \in \mathcal{C^1}(\R)\), \(f'\) being the derivative of \(f\), and \(x_0 \in \R\); then we define:
\begin{displaymath}
	x_{n+1} := x_n - \frac{f(x)}{f'(x)}\quad \forall\: n \in \N
\end{displaymath}
\end{Newton Method}

The Newton Raphson method is not suitable for all problems and there are in fact many cases in which it behaves poorly. One such case is when \(f'(x_n) \approx 0\) as the value of \(x_{n+1}\) will be very close to \(x_n\) and thus \(f'(x_{n+1}) \approx 0\). Further bad choices of \(x_0\) can lead to the method diverging or entering cycles between two points indefinitely, however we will see that we do not need to be concerned with these issues for our uses of the method.\\

%SUBSUB%
\subsubsection{Taylor Series Expansion}
\label{SUBSUB_"Taylor Series"}
\theoremstyle{definition}
\newtheorem{Taylor Series}{Definition}[subsubsection]
\newtheorem{Taylor Polynomial}[Taylor Series]{Definition}

The Taylor Series formulation was created by Brook Taylor in 1715\cite{BOK_Taylor}, based off of the work of Scottish mathematician James Gregory. The Taylor Series describes a method of representing any infinitely differentiable function as an infinite power series.

%DEF%
\begin{Taylor Series}
\label{DEF_"Taylor Series"}
Given \(f : \R \to \R\) which is infinitely differentiable on an open interval \(\mathcal{I}\) centred at \(a \in \R\), we define the Taylor Series of \(f\) on \(\mathcal{I}\) to be:
\[\sum_{n=0}^{\infty} \frac{f^{(n)}(a)}{n!}(x-a)^n\]
\end{Taylor Series}

It was shown that on the open interval \(\mathcal{I}\) from the above definition we have that \(f(x) = \sum_{n=0}^{\infty} \frac{f^{(n)}(a)}{n!}(x-a)^n\), i.e. a function is equal to it's Taylor polynomial on the interval for which it is defined. We can then use this fact to define a polynomial that will approximate our function \(f\) at \(x \in \mathcal{I} \subset \R\)\\

%DEF%
\begin{Taylor Polynomial}
\label{DEF_"Taylor Polynomial"}
Given \(f : \R \to \R\) which has a Taylor Series of
\( \sum_{n=0}^\infty c_n x^n \), we define the Taylor Polynomial of degree \(N \in \N\) to be
\[ p_N(x) := \sum_{n=0}^N c_n x^n = c_0 + c_1 x + c_2 x^2 + \dotsb + c_N x^N\]
\end{Taylor Polynomial}

A commonly used type of Taylor series is the Maclaurin series which is a Taylor series in an interval around \(a = 0\). Thus a Maclaurin series has the form:
\[\sum_{n=0}^N \frac{f^{(n)}(0)}{n!}x^n\]

Some examples of simple Maclaurin Series are:
\begin{align}
\frac{1}{1-x} &= \sum_{n=0}^\infty x^n &&\forall\: x \in (-1,1)\\
(1 + x)^k	  &= \sum_{n=0}^\infty \binom{k}{n} x^n 
					&&\forall\: x \in (-1,1),\  k \in \N
					\label{EQN_"Maclaurin Series 1"}
\end{align}

%SUB%
\subsection{Errors}
\label{SUB_"Error Definitions"}
\theoremstyle{definition}
\newtheorem{Absolute Error}{Definition}[subsection]
\newtheorem{Iteration Error}[Absolute Error]{Definition}

The error of an approximation \(\tilde{v}\) for some \(v\) is a measure of how much \(\tilde{v}\) differs from \(v\). We will use the error of approximations to discuss the convergence of methods as well as describing their accuracy.\\

There are several ways of evaluating the error of an approximation which each have their own uses. The error measures that we will use in this document are detailed below:\\

%DEF%
\begin{Absolute Error}
\label{DEF_"Absolute Error"}
If we have a value \(v\) and it's approximation \(\tilde{v}\), then the absolute error is
\[ \epsilon := \left| v - \tilde{v} \right| \]
\end{Absolute Error}

The absolute error is useful in guaranteeing a certain level of accuracy that a given implementation of a method will give; for instance if \(\epsilon < 10^{-3}\) then the approximation is accurate to at least 3 decimal places. Uses of absolute error in the document will use \(\epsilon\) as their absolute error variable.\\

As the absolute error of an approximation is hard or impossible to accurately calculate during program execution, we need a way to estimate it. Typically our computations will produce a sequence of approximations \(x_0, x_1, x_2, \ldots\), and thus we define the following:

%DEF%
\begin{Iteration Error}
\label{DEF_"Iteration Error"}
If we have the sequence \(\left(x_n : n \in \N\right)\), then the iteration error is defined as:
\[ \delta_n := \left|x_n - x_{n-1}\right| \]
\end{Iteration Error}

While it is often impossible to calculate \(\epsilon_n\) it is very easy to calculate \(\delta_n\) from the generated approximations. This estimate is best used when we know that the convergence is rapid, as in these cases \(\delta_n\) is a good approximation of \(\epsilon_n\).

%SUB%
\subsection{Convergence}
\label{SEC_"Convergence"}

\theoremstyle{definition}
\newtheorem{Uniform Convergence}{Definition}[subsection]
\newtheorem{Rate of Convergence}[Uniform Convergence]{Definition}

\theoremstyle{remark}
\newtheorem{Uniform Convergence R1}{Remark}[Uniform Convergence]

\theoremstyle{plain}
\newtheorem{Uniform Convergence Thm}{Theorem}[subsection]
\newtheorem{Quad Convergence of Newton}[Uniform Convergence Thm]{Theorem}

As our methods of approximating functions will typically generate a sequence of values \(x_0, x_1, x_2, \ldots\) then we want to ensure that the approximations are approaching the correct value. We consider here what it means for a sequence to converge to a limit value, and some useful results for later chapters.

%DEF%
\begin{Uniform Convergence}
\label{DEF_"Uniform Convergence"}
A sequence \((x_n \in \R: n \in \N)\) converges to \(x\) uniformly if 
\begin{displaymath}
	\forall\: \tau \in \Rpz\  \exists\: N \in \N \textrm{ s.t. }\  
		\epsilon_n := |x - x_n| < \tau\:\:\forall\: n \in [N, \infty)\cap\Z
\end{displaymath}
\end{Uniform Convergence}

%RMK%
\begin{Uniform Convergence R1}
\label{RMK_"Uniform Convergence R1"}
We will typically use the notation that \(\lim_{n \to \infty} |x_n - x| = 0\), to denote that \((x_n : n \in \N)\) converges to \(x\).
\end{Uniform Convergence R1}

%THM%
\begin{Uniform Convergence Thm}
\label{THM_"Uniform Convergence Thm"}
\((x_n \in \R : n \in \N)\) converges to \(x\) uniformly if and only if 
\begin{displaymath}
	\forall\: \tau \in \Rpz\  \exists\: N \in \N \textrm{ s.t. }\ 
		|x_n - x_m| < \tau\:\: \forall\: m, n \in [N, \infty) \cap \Z
\end{displaymath}
\end{Uniform Convergence Delta}
\begin{proof}
For \(\implies\):\\
Suppose that \((x_n : n \in \N)\) converges to \(x\) uniformly. Then 
\begin{displaymath} 
	\forall\: \tau \in \Rpz\  \exists\: N \in \N \textrm{ s.t. }\ 
	 	|x_n - x| < \tau\:\: \forall\: n \in [N, \infty) \cap \Z
\end{displaymath}
Thus suppose \(N \in \N\) is such that \(|x_n - x| < \frac{\tau}{2}\  \forall\: n \in [N, \infty) \cap \Z\).\\
Then if \(n, m \ge N\) we see that \[|x_n - x_m| \le |x_n - x| + |x_m - x| \le \tau\]

For \(\Leftarrow\):\\
Omitted for brevity.
\end{proof}

We have shown now what it means for a value to converge to a limit, but not all sequences that approach a limit do so at the same pace. For example if we consider the sequences \(x_n := 2^{-n}\) and \(y_n := 10^{-n}\), then it is obvious that the limit of both sequences is \(0\), but \(y_n\) approaches the limit faster. This leads to the following definition of the rate of convergence.

\begin{Rate of Convergence}
If \((x_n \in \R : n \in \N)\) is a sequence that converges to \(x\), then it is said to converge:
\begin{itemize}
\item Linearly if \(\lambda \in \Rp\) and \[\lim_{n\to\infty}\frac{|x_{n+1} - x|}{|x_n - x|} = \lambda\]
\item Quadratically if \(\lambda \in \Rp\) and \[\lim_{n\to\infty}\frac{|x_{n+1} - x|}{|x_n - x|^2} = \lambda\]
\item Order \(\alpha \in \Rpz\) if \(\lambda \in \Rp\) and \[\lim_{n\to\infty}\frac{|x_{n+1} - x|}{|x_n - x|^\alpha} = \lambda\]
\end{itemize}
\end{Rate of Convergence}

The higher the order of convergence of a sequence the faster it approaches it's limit, therefore we are looking for algorithms with high orders of convergence. Many regular series have linear convergence and quadratic convergence is typically very rapid, while orders above quadratic are hard to construct for useful functions.\\

A useful result is that, under the correct circumstances, the Newton-Raphson method can be shown to have quadratic convergence. The following proof assumes that \(\epsilon_n := |x^\ast - x_n|\):

%THM%
\begin{Quad Convergence of Newton}
\label{THM_"Quad Conv Newton"}
Let \(f\) be a twice differentiable function, \(x^\ast\) be a solution to \(f(x) = 0\) and \((x_n : n \in \N)\) be a sequence produced by the Newton-Raphson Method from some initial point \(x_0\). If the following are satisfied, then \((x_n : n \in \N_0)\) converges quadratically to \(x^\ast\):
\begin{description}

\item[\(\textrm{NR}_1\):]
\begin{math}
	f'(x) \neq 0\  \forall\: x \in I := [x^\ast - r, x^\ast + r], \ \mathrm{where}\ r \in \left[\left|x^\ast - x_0\right|, \infty\right)
\end{math}

\item[\(\textrm{NR}_2\):]
\begin{math}
	f''(x) \ \textrm{is continuous}\  \forall\: x \in I
\end{math}

\item[\(\textrm{NR}_3\):]
\begin{math}
	M\left|\epsilon_0\right| < 1 \ \mathrm{where}\ M := \sup\left\{\left|\frac{f''(x)}{f'(x)}\right| : x \in I\right\}\\
\end{math}
\end{description}
\end{Quad Convergence of Newton}

\begin{proof}
By Taylor's Theorem with Lagrange Remainders\cite[][80]{BOK_Taylor} we have that
\begin{displaymath}
	0 = f(x^\ast) = f(x_n) + (x^\ast - x_n)f'(x_n) + \tfrac{1}{2}
		(x^\ast - x_n)^2f''(y_n)
\end{displaymath}
where \(0 < |x^\ast - y_n| < |x^\ast - x_n|\).\\

Then we get the following derivation:
\begin{displaymath}
\begin{align*}
	&f(x_n) + (x^\ast - x_n)f'(x_n) = 
		-\tfrac{1}{2}(x^\ast - x_n)^2f''(y_n)\\
	\implies &\left(\frac{f(x_n)}{f'(x_n)} - x_n\right) + x^\ast =
		-\frac{1}{2}\frac{f''(y_n)}{f'(x_n)}(x^\ast - x_n)^2
		&\textrm{as} \ \textrm{NR}_3 \implies f'(x_n) \neq 0\\
	\implies &x^\ast - x_{n+1} = 
		-\frac{1}{2}\frac{f''(y_n)}{f'(x_n)}(x^\ast - x_n)^2\\
	\implies &\epsilon_{n+1} =
		\frac{1}{2}\left|\frac{f''(y_n)}{f'(x_n)}\right|\epsilon_n^2
		&\textrm{by taking absolute values}
\end{align*}
\end{displaymath}
As \(\textrm{NR}_2\) holds then \(M\) exists and is positive, and therefore we have:
\[\epsilon_n \le M\epsilon_{n-1}^2 \le M^{2^n - 1}\epsilon_0^{2^n}\]

We now aim to show that we have convergence, i.e. \(\lim_{n \to \infty} x_n = x^\ast\); to do this it suffices to show that \(\lim_{n\to\infty}\epsilon_n = 0\).\\

Consider the sequence \((z_n := M^{2^n - 1}\epsilon_0^{2^n} : n \in \N_0)\). We know that \(0 \le \epsilon_n \le z_n \forall n \in \N_0\), so it then follows that if \(\lim_{n \to \infty}z_n = 0\), then \(\lim_{n \to \infty}\epsilon_n = 0\) by the Squeeze Theorem\cite[][909]{BOK_Squeeze}.\\

Now as \(M\epsilon_0 < 1\) by \(\textrm{NR}_3\), then we see that:

\begin{displaymath}
\begin{align*}
\lim_{n\to\infty}z_n 
	&= \lim_{n\to\infty}(M\epsilon_0)^{2^n - 1}\epsilon_0\\ 
	&= \epsilon_0\lim_{n\to\infty}(M\epsilon_0)^{2^n - 1}\\
	&= \epsilon_0\cdot0
		&\textrm{because \(M|\epsilon_0| < 1\)}\\
	&= 0
\end{align*}
\end{displaymath}

Now to show that this sequence converges quadratically we see that \(\epsilon_{n+1} = \frac{1}{2}\left|\frac{f''(y_n)}{f'(x_n)}\right|\epsilon_n^2\), and therefore \(\frac{\epsilon_{n+1}}{\epsilon_n^2} = \frac{1}{2}\left|\frac{f''(y_n)}{f'(x_n)}\right|\).\\

Because \(|x^\ast - y_n| < |x^\ast - x_n|\) and \(\lim_{n\to\infty}x_n = x^\ast\), then it follows that \(\lim_{n\to\infty}y_n = x^\ast\). Therefore we see that
\[\lim_{n\to\infty}\frac{\epsilon_{n+1}}{\epsilon_n} = \frac{1}{2}\left|\frac{f''(x^\ast)}{f'(x^\ast)}\right| \in \Rp\]

Hence as the above limit exists and is positive then the sequence is quadratically convergent.
\end{proof}

%SUB%
\subsection{Efficiency Metrics}
\label{SUB_"Efficiency Metrics"}

Now that we have discussed how to measure the accuracy of our results by their errors, we wish to consider the efficiency method. There is typically a trade-off between accuracy and efficiency in that to gain a more accurate result, more calculations are required thus taking up more resources. In general however, we will be using efficiency metrics to compare how efficient two different algorithms are at getting the same result.\\

There are two main ways in which we will measure the efficiency of an algorithm. The first of these methods is the theoretical complexity of the algorithm, which represents the number of steps/operations an algorithm needs to achieve it's goal. The complexity of an algorithm is denoted by the big O notation, which represents the order of the complexity, i.e. the highest order term in the number of operations required.\\

Typically the execution of an algorithm depends on the size of the input and so if we consider that an input has size \(n\) we can discuss different complexities. The first consideration is that if one algorithm takes \(2n\) operations while another takes \(20n\) operations, then both algorithms have a complexity of \(\bigO(n)\). \\

A complexity of \(\bigO(n)\) is not a bad complexity for an algorithm as the number of operations needed rises linearly with the size of the input. Complexities of \(\bigO(n^2)\), \(\bigO(2^n)\) and \(\bigO(n!)\) are all poor complexities for an algorithm\cite{ART_Complexity} with the latter two becoming infeasible for larger \(n\). On the other hand complexities better than \(\bigO(n)\) include \(\bigO(\log(n))\) and \(\bigO(1)\), the latter of these is particularly significant as it means that the algorithm takes the same number of steps regardless of the size of the input.\\

The second method of assessing efficiency consists of timing of functions during execution. This method directly observes how long it takes a computer to perform the calculations for a given algorithm and can be used to empirically test the speed of two algorithms. One remark is that due to the speed of modern computers it is infeasible to time the execution of a single function, and one typically times the same algorithm with the same input being calculated multiple times to get accurate and measurable timings.

%SUB%
\section{Division and Multiplication}
\label{SUB_"Division and Multiplication"}
\subsection{Introduction}
Though the idea of Division and Multiplication can seem fairly simple, particularly from an abstract pure mathematical point of view, these calculations can be computationally difficult. This section will show a few algorithms designed to calculated these values, and discuss their implementation and efficiency.

\subsection{Multiplication}
\subsubsection{Basic Methods}
\subsubsection{Advanced Methods}
\subsubsection{Analysis}

\subsection{Division}
\subsubsection{Basic Methods}
\subsubsection{Advanced Methods}
\subsubsection{Analysis}
\TODO{Section to be filled out, no work currently done on this section}

%SEC%
\section{Trigonometric Functions}

%SUB%
\subsection{Geometric Method}
\label{SUB_"Trig Geometric Method"}

The first method I will be discussing is a method based on geometric properties that are derived on a circle, and we will start by considering values of \(\cos\) in the range \([0, \frac{\pi}{2})\). To do this we will consider the following figure of the unit circle:

%FIG%
\begin{figure}[!ht]
	\label{FIG_"Geometric Trig 1"}
	\caption{Diagram showing angles to be dealt with}
	\centering
	\includegraphics[width=0.5\textwidth]{"./Diagrams/Geometric Trig Diagram 1"}
\end{figure}

Here theta will be given in radians, and we can note that the labelled arc has length \(\theta\) due the formula for the circumference of a circle. By using the following derivation we can find a formula for \(\theta\) in terms of \(s\):

\begin{displaymath}
\begin{align*}
	s^2 &= \sin^2\theta + (1 - \cos\theta)^2\\
	    &= (\sin^2\theta + \cos^2\theta) + 1 - 2\cos\theta\\
		&= 2 - 2 \cos\theta 
			&\mathrm{By using } \sin^2\theta + \cos^2\theta = 1\\
	\cos\theta &= 1 - \frac{s^2}{2}
\end{align*}
\end{displaymath}

We will now consider a second diagram which will allow us to calculate an approximate value of \(s\).

%FIG%
\begin{figure}[!ht]
	\label{FIG_"Geometric Trig 2"}
	\caption{Diagram detailing how to calculate \(s\)}
	\centering
	\includegraphics[width=0.5\textwidth]{"./Diagrams/Geometric Trig Diagram 2"}
\end{figure}

We will first note that by an elementary geometry result we can know that the angle \(ABC\) is a right-angle; also we can consider that \(h\) is an approximation of \(\tfrac{\theta}{2}\), which will become relevant later. Now because \(AC\) is a diameter of our circle then it's length is 2 and thus, by utilising Pythagarus' Theorem, we get that the length of \(AB\) is \(\sqrt{AC^2 - BC^2} = \sqrt{4 - h^2}\).\\

From here we consider the area of triangle \(ABC\), which can be calculated as \(\frac{1}{2}\cdot h\cdot\sqrt{4-h^2}\) and as \(\frac{1}{2}\cdot2\cdot\frac{s}{2}\); by equating these two, squaring both sides and re-arranging we get that \(s^2 = h^2(4 - h^2)\). Now we have the basis for a method that will allow us to calculate \(\cos\theta\).\\

To complete our method we will consider introducing a new line that is to \(h\) what \(h\) is to \(s\) as shown in the diagram below:

%FIG%
\begin{figure}[!ht]
	\label{FIG_"Geometric Trig 3"}
	\caption{Detailing the recursive steps}
	\centering
	\includegraphics[width=0.5\textwidth]{"./Diagrams/Geometric Trig Diagram 3"}
\end{figure}

It is easy to see that if we repeat the steps above we get that \(h^2 = \hat{h}^2(4 - \hat{h}^2)\), and it also follows that \(\hat{h} \approx \frac{\theta}{4}\). Using this we can take an initial guess of \(h_0 := \frac{\theta}{2^k}\), for some \(k \in \N\), and then calculate \(h_{n+1}^2 = h_n^2(4 - h_n^2)\) where \(n \in [0, k] \cap \Z\); finally we calculate \(\cos\theta = 1 - \frac{h_k^2}{2}\), giving the following algorithm:
  
%PCD%
\begin{lstlisting}[numbers=left,frame=single,mathescape,caption={Geometric calculation of \(\cos\)},label={PCD_"Gometric Cos"}]
  geometric_cos($\theta \in [0, \frac{\pi}{2}), k \in \N$)
      $h_0 := \tfrac{\theta}{2^k}$
      $n := 0$
      while $n < K$:
          $h_{n+1}^2 := h_n^2\cdot(4 - h_n^2)$
          $n \mapsto n + 1$
      return $1 - \tfrac{h_k^2}{2}$
\end{lstlisting}

\subsection{Taylor Series}
\subsection{CORDIC}
CORDIC is an algorithm that stands for COrdinate Rotation DIgital Computer and can be used to calculate many functions, including Trigonometric Values. The CORDIC algorithm works by utilising Matrix Rotations of unit vectors. This algorithm is less accurate than some other methods but has the advantage of being able to be implemented for fixed point real numbers in efficient ways using only addition and bitshifting.\\

CORDIC works by taking an initial guess of
\begin{math}
	\mathbf{x}_0 = \left( 
		\begin{array}{c}
			1 \\
			0
		\end{array} \right)
\end{math}
which can be rotated through an anti-clockwise angle of $\gamma$ by the matrix
\begin{displaymath}
	\left( \begin{array}{cc}
		\cos{\gamma} & -\sin{\gamma} \\
		\sin{\gamma} &  \cos{\gamma}
	\end{array} \right)
	= \frac{1}{\sqrt{1 + \tan{\gamma}^2}} \left( \begin{array}{cc}
		1 & -\tan{\gamma} \\
		\tan{\gamma} & 1
	\end{array} \right)
\end{displaymath}

By taking taking smaller and smaller values of $\gamma$ we can create an iterative process to find $\mathbf{x}_n$ which converges, for a given $\beta \in (-\frac{\pi}{2}, \frac{\pi}{2})$, to
\begin{displaymath}
	\left( \begin{array}{c}
		\cos{\beta}\\
		\sin{\beta}
	\end{array} \right)
\end{displaymath}


%SEC%
\section{Root Functions}
\label{SEC_"Root Functions"}

In this section of the document we will consider several methods for approximating root functions. For our purposes we are only going to consider roots of \(N \in \Rpz\), this is because if \(N \in \R^-\) then it follows that \(\sqrt{N} = i\sqrt{|N|}\).

%SUB%
\subsection{Digit by Digit Method}
\label{SUB_"Digit by Digit Method"}

The first method we will examine is an old method, that has been observed in Babylonian Mathematics over 2000 years ago, which is used to accurately generate the square root of numbers one digit at a time. This method differs from others discussed as it generates each digit of the root with perfect accuarcy, one at a time, thus in a theoretical sense this algorithm is the most accurate of the methods we will view; we will see however that this method is slow.\\

Now suppose we are looking for \(\sqrt{N}\), then we know that \(\sqrt{N} = a_010^n + a_110^{n-1} + a_210^{n-2} + \dots\) for some \(n \in \Z\); it then follows that \(N = (a_010^n + a_110^{n-1} + a_210^{n-1} + \dots)^2\). By expanding the quadratic value we get that \[N = a_0^210^{2n} + (20a_0 + a_1)a_110^{2n-2} + (20(a_010 + a_1) + a_2)a_210^{2n-4} + \dots + (20\sum_{i=0}^{k-1}a_i10^{k-i-1} + a_k)a_k10^{2n - 2k}\]

An observation should be made regarding the value of \(n\) that we use for the theorem. We could of course try different values of \(n\), in some structured procedure, that will find the largest \(n\) such that \(10^n \le N\). However we can note that \(log_{10}(\sqrt{N}) = \tfrac{1}{2}log_{10}(N)\), thus \(10^{\frac{1}{2}log_{10}(N)} = \sqrt{N}\). Using this information, and the fact that \(n \in \Z\), we can have \(n := \left\lfloor \tfrac{1}{2}log_{10}(N) \right\rfloor\). \\

This allows us to get successive apporximations of \(N\) where \(N_0 = a_0^210^{2n}\), \(N_1 = N_0 + (20a_0 + a_1)a_110^{2n-2}\), \(N_2 = N_1 + (20(a_010 + a_1) + a_2)a_210^{2n-4}\). This will alllow us to create an algorithm that will give successive approximations of \(sqrt{N} = a_010^n + a_110^{n-1} + \dots\), more importantly each approximation will give us the exact next digit in the decimal representation of \(\sqrt{N}\).\\

Thus we can have an iterative method to solve the problem, where at each stage we are trying to find the largest digit which satisfies the inequality \((20\sum_{i=0}^{k-1}a_i10^{k-i-1} + a_k)a_k10^{2n-2k} \le N - N_{k-1}\). Thus we get the following pseudocode, which outputs two sequences, one indicating the digits before the decimal point and one afterwards. I will use set notation to indicate the sequences, but in this case order is important and repetition is allowed.

%PCD%
\begin{lstlisting}[numbers=left,frame=single,mathescape,caption={Exact Digit by Digits Square Root}]
  exactRootDigits($N \in \Rpz, d \in \N$):
      $Digits_a := \emptyset$
      $Digits_b := \emptyset$
      $k := 0$
      $n := \left\lfloor\tfrac{1}{2}log_{10}(N)\right\rfloor$
      while $k < d$:
          $a_k := \max\left\{t \in [0, 9] \cap \Z : \left(20\sum_{i=0}^{k-1}a_i10^{k-i-1} + t\right)t10^{2n-2k} \le N\right\}$
          $N \mapsto N - \left(20\sum_{i=0}^{k-1}a_i10^{k-i-1} + a_k\right)a_k10^{2n-2k}$
          if $n-k < 0$:
              $Digits_b \mapsto Digits_b \cup \{a_k\}$
          else:
              $Digits_a \mapsto Digits_a \cup \{a_k\}$
          $k \mapsto k+1$
      if $Digits_a = \emptyset$:
          $Digits_a := \{0\}$
      if $Digits_b = \emptyset$:
          $Digits_b := \{0\}$
      return $(Digits_a, Digits_b)$
\end{lstlisting}

This method has a computational complexity of \(\bigO(d^2)\), as each loop requires the operations of summing \(k\) elements, and the loop is repeated for \(k = 0 \to d\). We will see that by considering some changes to the algorithm we can change the complexity class to be \(\bigO(d)\).\\

First we will note that line 5 is not an issue, as if we only care about the first significant digit of \(\tfrac{1}{2}log_{10}(N)\), then this is \(\bigO(|log(N)|)\). This can be seen as if we start from \(n = 0\) we can either count up or down until a we find \(10^{2n}\) at most or at least N, respectively. This obviously takes at most \(|log_{10}(N)|\) steps, giving us our stated complexity. We will also assume that \(\bigO(|log(N)|) \le \bigO(d)\), as we have already seen that we can manipulate our input N to be within a reasonable range.

Second we note that on line 7 we calculate \(\sum_{i=0}^{k-1}a_i10^{k-i-1}\) for each value of \(t\); we can reduce the complexity of this line by pre-calculating this value. However we can do even better if we consider that at step \(k+1\) we are calculating \(\sum_{i=0}^{k}a_i10^{k-i} = a_k + 10\sum_{i=0}^{k-1}a_i10^{k-i-1}\). Thus if we introduce \(P_0 := 0\), and fore each k we calculate \(P_{k+1} := 10P_k + a_k\), then we can reduce the complexity from \(\bigO(k)\) to \(\bigO(1)\).\\

This calculation of \(P_k\), then carries over to reduce the complexity of line 8 to be \(\bigO(1)\) instead of \(\bigO(k)\). Combining this we can create the modified algorithm below:

%PCD%
\begin{lstlisting}[numbers=left,frame=single,mathescape,caption={Exact Digit by Digits Square Root version 2}]
  exactRootDigits_v2($N \in \Rpz, d \in \N$):
      $Digits_a := \emptyset$
      $Digits_b := \emptyset$
      $k := 0$
      $n := \left\lfloor\tfrac{1}{2}log_{10}(N)\right\rfloor$
      $P_0 := 0$
      while $k < d$:
          $a_k := \max\left\{t \in [0, 9] \cap \Z : \left(20P_k + t\right)t10^{2n-2k} \le N\right\}$
          $N \mapsto N - \left(20P_k + a_k\right)a_k10^{2n-2k}$
          $P_{k+1} := 10P_k + a_k$
          if $n-k < 0$:
              $Digits_b \mapsto Digits_b \cup \{a_k\}$
          else:
              $Digits_a \mapsto Digits_a \cup \{a_k\}$
          $k \mapsto k+1$
      if $Digits_a = \emptyset$:
          $Digits_a := \{0\}$
      if $Digits_b = \emptyset$:
          $Digits_b := \{0\}$
      return $(Digits_a, Digits_b)$
\end{lstlisting}

This method is usefull, but can be difficult to implement as it requires high precision for the representation of the real value of \(N\). In my implementation using C, I utilised the MPFR library to utilise high precision integers, but still encountered issues regarding loss of precision.\\

As an example the table below shows the number of digits of accuracy I was able to calculate for \(\sqrt{2}\) using the above algorithm, compared to the number of bits of precision used in the calculations.\\

%TBL%
\begin{center}
\begin{tabular}{|p{3cm}|p{3cm}|}
\hline
Bits of Precision & Maximum Accuracy\\ \hline
8 & 2 \\ \hline
16 & 5 \\ \hline
32 & 9 \\ \hline
64 & 18 \\ \hline
128 & 39 \\ \hline
256 & 77 \\ \hline
512 & 154 \\ \hline
1024 & 308 \\ \hline
2048 & 615 \\ \hline
4096 & 1234 \\ \hline
8192 & 2466 \\ \hline
\end{tabular}
\end{center}

This data is highly structured and so we can hope to create a simple function that would allow us to calculate how much precision would be needed for a given number of digits of accuracy, at least for single digit inputs for \(N\). We can see that the average ratio of Precision to Accuracy is 3.41259..., which ranges from 3.31928... to 4.0. From this we can draw a general trend that Digits of Accuracy \(\approx\) 3.4 \(\times\) Bits of Precision; thus if we take the more generous assumption that Digits of Accuracy \(\appprox\) 4 \(\times\) Bits of Precision, we can use this to pre-determine the accuracy needed.\\

It should be noted that to ensure accuracy we should over-estimate the required precision, however if we overestimate the precision, then our calculations will be performed using unnecsarily large data structures and thus computation time will increase.\\

One particular use of this technique is to find an approximation of a squar root to it's integer part, calculated in base 2. This algorithm is of note as we will see that it has a computation time of \(\bigo(1)\).\\

The algorithm uses the same basis as the base 10 version, for it's calculations, but due to the nature of being in binary several changes can be made for computational efficiency. To do this we will view the problem as follows: if we know some \(r \in \Zpz\) which is our current approximation of our root, we are looking for some \(e \in \Zpz\) such that \((r+e)^2 \le N\). Expanding this out we get \(r^2 + 2re + e^2 \le N\), and if we keep track of \(M = N - r^2\), we can test if \(2re + e^2 \le M\).\\

Now we can consider our choice of \(e\), the most practical method is to test successeive \(e_m := 2^m\), where \(m\) is descending starting with \(m = \max{m \in \Zpz : 4^m \le N}\). We can use an iterative formula to build up the integer square root, where we start with \(r = 0, M = N\) and have \(r \mapto r + e_m\) whenever \(2re_m + e_m^2 \le M\), stopping when \(m < 0\). This is then implemented as follows:\\

%PCD%
\begin{lstlisting}[numbers=left,frame=single,mathescape,caption={Integer Square Root Algorithm}]
  integerSquareRoot($N \in \Zpz$):
      $M := N$
      $m := \max{m \in \Zpz : 4^m \le M}$
      $r := 0$
      while $m \ge 0$:
          if $2r(2^m) + 4^m \le M$:
              $M \mapsto M - 2r(2^m) + 4^m$
              $r \mapsto r + 2^m$
          $m \mapsto m - 1$
      return $r$
\end{lstlisting}

If we now conisder an implementation of the above algorithm using an unsigned integer system with \(K\) bits, where \(2 | K\). We will use \codeinline{res} to represent \(2re_m\), which means at the start of the algorihtm we will have \codeinline{res = 0}; similarly we can use \codeinline{bit} to represent \(e_m^2\). As we know that \(K\) bits are used and \(2 | K\), it then follows that the largest power of 4 less than the maximum representable value (\(2^K - 1\) is \(2^{K-2}\), which can be calculated as \codeinline{bit = 1 << (K - 2)} using bitshift operations. Finally we will use \codeinline{num} to represent \(M\).\\

Now that we have discussed the setup we can consider how to implement some of the steps above. First to implement line 3 we can simply keep dividing \codeinline{bit} by 4 while \codeinline{bit > num}, which can be efficiently implemented as \codeinline{bit >> 2} by using bitshifts in place of division by powers of 2. The same technique can be used in place of line 9, which leads us to re-evaluating our usage of line 5. As we are using bitshifting and a bitshift that would take a number past 0 instead results in 0, we also know that \(2 | K\) and so eventually we will reach \codeinline{bit == 1}, which represents \(m = 0\); therefore we can use \codeinline{bit > 0} as our stopping criteria on line 5.\\

Line 6 is easy to convert, given our definitions of \codeinline{res}, \codeinline{bit} and \{num}, as is line 7. All that remains is to consider how to update \codeinline{res}, which has two different ways of being updated depending on whether \codeinline{res + bit <= num}. If it is false that \codeinline{res + bit <= num}, then we wish for \codeinline{res} to represent \(2re_{m-1}\); this is easily acheived if we consider that \(2re_{m-1} = \frac{1}{2}(2re_m)\), which prompts the update \codeinline{res = res >> 1}. For the second case, when \codeinline{res + bit <= num} is true, we want \codeinline{res} to represent \(2(r+e_m)e_{m-1}\); to implement this we consider the following derivation:

\begin{displaymath}
\begin{align*}
	2(r+e_m)e_{m-1} 
		&= \frac{1}{2}\cdot 2(r+e_m)e_m\\
		&= \frac{1}{2}\cdot 2(re_m + e_m^2)\\
		&= \frac{1}{2}(2re_m) + e_m^2
\end{align*}
\end{displaymath}

Using this above derivation we see that we can calculate this as \codeinline{res = (res >> 1) + bit}. Below is a simple implementaion of this in C using the unsigned 32 bit integer type \codeinline{uint32\_t}. A more commented and slightly modified version can be found in Appendix \ref{#APP#}, File \ref{#FILE#}.

%PCD%
\begin{codelisting}{Integer Square Root in C}
uint32_t int_sqrt(uint32_t num)
{
	uint32_t res = 0, bit = (1 << 30);
	
	while (bit > num)
		bit = bit >> 2;
	
	while (bit > 0)
	{
		if (res + bit <= num)
		{
			num = num - (res + bit);
			res = (res >> 1) + bit;
		}
		else
			res = res >> 1;
		
		bit = bit >> 2;
	}

	return res;
}
\end{lstlisting}
\end{codelisting}

We should consider the final step of the loop, when \codeinline{bit == 1}. In this case when \codeinline{res} is updated we have \codeinline{res} represent either \(2(r+e_0)e_{-1} = r + e_0\), or \(2re_{-1} = r\); thus the algorithm exits with the correct value.\\

Now that the algorithm is correctly constructed using simple unsigned integer addition, subtraction and bitshifting (which we can assume all have computational time of \(\bigO{1}\)), we can look at the worst case complexity of the algorithm:

\begin{itemize}
\item The complexity of the set up of variables is constant time.
\item The worst case complexity would be to to have \codeinline{bit <= num} at the start.
\item The loop would execute 16 times for our 32 bit integers, and contains a single operation which is \(\bigO(1)\) complexity.
\begin{itemize}
	\item The worst case within the loop is to have \codeinline{res + bit <= num} for each iteration.
	\item Within the first \codeinline{if} branch there are a constant 4 operations.
	\item Each loop has an additional operation operation to update \codeinline{bit}.
	\item This makes 5 operations per loop, giving \(\bigO(1)\) complexity within the loops.
\end{itemize}
\end{itemize}

Therefore we see that the algorithm has \(\bigO(1)\) time complexity, and even has the same in storage complexity. In particular our 32 bit example requires 163 opertaions, including assignments, comparrisons and calucluations. This means that the integer square root of any number up to 4294967295 can be calculated extremely quickly.

%SUB%
\subsection{Bisection Method}
\label{SUB_"Bisection Method for Roots"}
\theoremstyle{plain}
\newtheorem{Bisection Converges}{Proposition}[subsection]

The Bisection Method is a general method for approximating the zero, \(\alpha\), of a function, \(f\), on a bounded interval, \(I := [a,b]\), where \(f\) has the property \(f(x)f(y) < 0 \forall (x,y) \in [a,\alpha)\times(\alpha, b]\); we may assume, without loss of generality, that \(f(x) < 0 \forall x \in [a, \alpha]\).\\

The bisection method starts with initial bounds \(a_0 = a, b_0 = b\), where the initial approximation for the root is \(x_0 = \frac{1}{2}(a+b)\). We will consider pseudocode of the iteration process, that uses \(b_n - a_a < \tau\) or \(f(x_n) = 0\) as exit criteria. Here \(\tau\) is a tolerance threshold, and if the exit criteria is met it means that \(|x_n - \alpha| \le \frac{\tau}{2}\), while the other exit criteria means we have reached an exact solution.\\

%PCD%
\label{PCD_"General Bisection Method"}
\begin{lstlisting}[frame=single,mathescape,caption={General Bisection Method}]
  bisectionMethod($a \in \R, b \in (a, \infty), f \in \mathcal{C}[a,b], \tau \in \Rp$)
      $a_0 := a$
      $b_0 := b$
      $x_0 := \tfrac{1}{2}(a+b)$
	  $n := 0$
	  while $f(x_n) \neq 0$ AND $b_n - a_n > \tau$:
          if $f(x_n) < 0$:
              $a_{n+1} := x_n$
              $b_{n+1} := b_n$
          else:
              $a_{n+1} := a_n$
              $b_{n+1} := x_n$
          $n \mapsto n+1$
          $x_n := \tfrac{1}{2}(a_n + b_n)$
      return $x_n$
\end{lstlisting}\\
		
For our purposes we are trying to find the zero of \(f(x) = x^2 - N\), which is a strictly increasing function on \(\Rpz\). If \(N >= 1\), then \(\sqrt{N} \in [0, N]\), while \(N < 1 \implies \sqrt{N} \in [0, 1]\). It is obvious that our function has the required property, and thus we get the following method for finding the square root of \(N\):\\

%PCD%
\begin{lstlisting}[frame=single,mathescape,caption={Bisection Method for Square Roots},label={PCD_"Square Root Bisection Method"}]
  bisectionSquareRoot($N \in \Rpz, \tau \in \Rp$)
      $a_0 := 0$
      $b_0 := \max{1, N}$
      $x_0 := \tfrac{1}{2}(a_0 + b_0)$
      $n := 0$
      while $x_n^2 - N \neq 0$ AND $b_n - a_n > \tau$:
          if $x_n^2 - N < 0$:
              $a_{n+1} := x_n$
              $b_{n+1} := b_n$
          else:
              $a_{n+1} := a_n$
              $b_{n+1} := x_n$
          $n \mapsto n+1$
          $x_n := \tfrac{1}{2}(a_n + b_n)$
      return $x_n$
\end{lstlisting}\\

The implementation of this method is efficiently acheived in C using only addition, subtraction and multiplication by a constant. Before this method is implemented, however, we must first consider if and or when it converges to the correct answer. From an intuitive standpoint we would assume that if there is only one root in the interval, it would follow that we would converge to the root.

%THM%
\begin{Bisection Converges}
\label{THM_"Bisecton Converges"}
\(\lim_{n \to \infty} x_n = \sqrt{N}\) for Algorithm \ref{PCD_"Square Root Bisection Method"}
\end{Bisection Converges}
\begin{proof}
To prove this statement it suffices to prove that \(\sqrt{N} \in [a_n, b_n] \forall n \in \N\) and \(\lim_{n\to\infty} |x_n - \sqrt{N}| = 0\).\\

\textit{Claim 1:} \(\sqrt{N} \in [a_n, b_n] \forall n \in \N\)
\begin{subproof}\\
\(a_0 := 0 \implies a_0 \le \sqrt{N}\)\\
\(b_0 := \max\{1, N\} \implies b_0 \ge \sqrt{N}\)\\
Therefore it is obvious that \(\sqrt{N} \in [a_0, b_0]\)\\
Now suppose \(\sqrt{N} \in [a_n, b_n]\) for some \(n \in \N\)\\
It should be noted that \(a_n, b_n, x_n \in \Rpz \forall n \in \N\) as \(a_0, b_0 \in \Rpz\) and all the subsequent values are derived from these using only addition and multiplication by positive factors.\\
We then see that \(x_n := \frac{1}{2}(a_n + b_n)\), and we consider the two cases that \(x_n^2 - N \le 0\) or \(x_n^2 - N \ge 0\).\\
\begin{description}
\item[Case \(x_n^2 - N \le 0\):]\\
	\(a_{n+1} := x_n, b_{n+1} := b_n\)\\
	It is therefore obvious that \(\sqrt{N} \le b_{n+1}\).\\
	Now we see that \(x_n^2 - N \le 0 \implies x_n^2 \le N \implies x_n \le N\) as all the values are non-negative.\\
	Thus \(\sqrt{N} \in [a_{n+1}, b_{n+1}]\).\\
\item[Case \(x_n^2 - N \ge 0\):]\\
	\(a_{n+1} := a_n, b_{n+1} := x_n\)\\
	It is therefore obvious that \(\sqrt{N} \ge a_{n+1}\).\\
	Now we see that \(x_n^2 - N \ge 0 \implies x_n^2 \ge N \implies x_n \ge N\) as all the values are non-negative.\\
	Thus \(\sqrt{N} \in [a_{n+1}, b_{n+1}]\).\\
\end{description}
Hence \(\sqrt{N} \in [a_n, b_n] \implies \sqrt{N} \in [a_{n+1}, b_{n+1}] \forall n \in \N\)\\
As \(sqrt{N} \in [a_0, b_0]\) then we see that \(\sqrt{N} \in [a_n, b_n] \forall n \in \N\)
\end{subproof}\\

\textit{Claim 2:} \(\lim_{n\to\infty}|x_n - \sqrt{N}| = 0\)
\begin{subproof}\\
Let \(n \in \N\) be arbitrary.\\
As \(x_n := \frac{1}{2}(a_n + b_n)\) then we see that \(|a_n - x_n| = |b_n - x_n| = \frac{1}{2}(b_n - a_n)\).\\
Now as \(\sqrt{N} \in [a_n, b_n]\) it follows that \(|\sqrt{N} - x_n| \le \frac{1}{2}(b_n - a_n\).\\
As the modulas function is a mapping from \(\R\) to \(\Rpz\), it is clear that \(|\sqrt{N} - x_n|\) is bounded below by 0.\\
Now as for each \(n \in \N\), either \(a_{n+1} = x_n\) or \(b_{n+1} = x_n\), we see that \(b_{n+1} - a_{n+1} = \frac{1}{2}(b_n - a_n)\). Further we can see that \(b_n - a_n \ge 0 \forall n \in \N\) because \(b_n \ge a_n\).\\
Therefore the sequence of \(\frac{1}{2}(b_n - a_n)\) is a strictly decreasing sequence that is bounded below, by 0. Thus \(\lim_{n\to\infty} \frac{1}{2}(b_n - a_n) = 0\)\\ 
Therefore \(\lim_{n\to\infty} |x_n - \sqrt{N}| = \lim_{n\to\infty}\frac{1}{2}(b_n - a_n) = 0\)
\end{subproof}

By using our two claims above we see that \(\lim_{n\to\infty} x_n = \sqrt{N}\).
\end{proof}

The algorithm can be generalised to search for \(\sqrt{k}{N}\), where \(k \in [2,\infty) \cap \Z\). We can do this by using the integer power function discussed previously in section \ref{SEC#}. This gives the following algorithm:
 
%PCD%
\label{PCD_"Bisection Method for General Roots"}
\begin{lstlisting}[frame=single,mathescape,caption={Bisection Method for General Roots}]
  kRootBisectionMethod($N \in \Rpz, k \in [2, \infty) \cap \Z, \tau \in \Rp$)
      $a_0 := 0$
      $b_0 := \max{1, N}$
      $x_0 := \tfrac{1}{2}(a_0 + b_0)$
      $n := 0$
      while $\textrm{intPow}(x_n, k) - N \neq 0$ AND $b_n - a_n > \tau$:
          if $\textrm{intPow}(x_n, k) - N < 0$:
              $a_{n+1} := x_n$
              $b_{n+1} := b_n$
          else:
              $a_{n+1} := a_n$
              $b_{n+1} := x_n$
          $n \mapsto n+1$
          $x_n := \tfrac{1}{2}(a_n + b_n)$
      return $x_n$
\end{lstlisting}\\

The proof that this converges to the correct value is very similar to the proof for square roots.\\

We can now consider the accuracy that can be acheived by our algorithm, for our purposes we will be considering \(\sqrt{N}\), though the same applies for \(\sqrt{k}{N}\). We know that \(\sqrt{N} \in [a_n, b_n] \forall n \in \N\), and in particular we know that either \(\sqrt{N} \in [a_n, x_n]\) or \(\sqrt{N} \in [x_n, b_n] \forall n \in \N\); therefore we know that \(\epsilon_n := \left|x_n - \sqrt{N}\right| \le \tfrac{1}{2}(b_n - a_n) \forall n \in \N\). Then as we know that \(b_{n+1} - a_{n+1} = \tfrac{1}{2}(b_n - a_n)\), we know that \(\epsilon_n \le \tfrac{1}{2^n}(b_0 - a_0)\).\\

We can consider that \(\forall N \in \Rpz \exists (r,k) \in [\frac{1}{4},1)\times\Z : N = r \cdot 2^{2k}\); using this we know that \(\sqrt{N} = \sqrt{r} \cdot 2^k\). As we have the fixed initial bounds of \(a_0 = 0\) and \(b_0 = 1\), then if we are finding \(\sqrt{r}\) we know that \(\epsilon_n \le \tfrac{1}{2^n} \forall n \in \N\). Hence we can calculate the precision of our current estimate beforehand for any \(n \in \N\), and thus we can guarantee \(d\) significant digits of accuracy for \(r \in [\frac{1}{4}, 1)\).\\

To get this accuracy must find \(n \in \N\) such that \(\epsilon_n \le \tfrac{1}{10^d}\), to acheive this we must find \(n \in \N\) such that \(2^n \ge 10^d\). For example the following table indicates the required \(n\), required for certain significant digits of accuracy.

%TBL%
\begin{center}
	\begin{tabular}{|p{3cm}|p{3cm}|}
	\hline
	\(d\) & \(n : 2^n \ge 10^n\)\\
	\hline
	1 & 0\\\hline
	5 & 15\\\hline
	10 & 30\\\hline
	20 & 64\\\hline
	50 & 163\\\hline
	100 & 329\\\hline
	\end{tabular}
\end{center}

Now ususally finding \(r\) and \(k\) as above would be as hard as calculating the logorithm of \(N\); however due to the way that C stores real numbers as either \codeinline{double} or in the MPFR library, finding these values are actually fairly trivial. Both provide a functionality to find \((a, b) \in [\frac{1}{2}, 1)\times \Z : N = a\cdot2^b\), and from this we merely require a simple comparison and division by 2 if \(b\) is not even. This leads to the following algorithm, which has the above maximum number of iterations for a required accuracy:


%PCD%
\begin{lstlisting}[frame=single,mathescape,caption={Bisection Method for Square Roots with fixed bounds},label={PCD_"Square Root Bisection Method fixed bounds"}]
  bisectionSquareRoot($N \in \Rpz, \tau \in \Rp$)
      Let $(r, e) \in [\tfrac{1}{2}, 1) : N = r\cdot2^e$
      if $2 \nmid e$:
          $r \mapsto \tfrac{r}{2}$
          $e \mapsto e - 1$
      $a_0 := 0$
      $b_0 := 1$
      $x_0 := \tfrac{1}{2}(a_0 + b_0)$
      $n := 0$
      while $x_n^2 - N \neq 0$ AND $b_n - a_n > \tau$:
          if $x_n^2 - N < 0$:
              $a_{n+1} := x_n$
              $b_{n+1} := b_n$
          else:
              $a_{n+1} := a_n$
              $b_{n+1} := x_n$
          $n \mapsto n+1$
          $x_n := \tfrac{1}{2}(a_n + b_n)$
      return $x_n \cdot 2^{\tfrac{e}{2}}$
\end{lstlisting}\\

%SUB%
\subsection{Newton's Method for Square Roots}
\label{SUB_"Newton for Square Roots"}

\theoremstyle{plain}
\newtheorem{SRNM Right-hand Convergence}{Proposition}[subsection]
\newtheorem{SRNM NR1 and NR2}[SRNM Right-hand Convergence]{Proposition}
\newtheorem{SRNM NR3 for v3}[SRNM Right-hand Convergence]{Proposition}

If we consider $f(x) = x^2 - N$ then if $x^\ast$ is a solution to $f(x) = 0$ we see that $x^\ast = \sqrt{N}$. As $f'(x) = 2x$, then the Newton's Method, will give $x_{n+1} = x_n - \frac{x^2 - N}{2x}$, where $x_0$ is a given initial guess.\\

We can see that, in C, each iteration will calculate \codeinline{x = x - (x*x - N) / (2*x)}, which requires 5 operations; however if we re-arrange our equation, we instead get \(x_{n+1} = \frac{1}{2}\right(x_n + \frac{N}{x}\right)\). Implementing our new iterative formula we get \codeinline{x = 0.5 * (x + N/x)}, which now uses only 3 operations.\\

We can then use the following pseudocode as the basis of our implementaions of the Newton-Raphson Method for Square Roots:

%PCD%
\label{PCD_"Newton Square Root Basic"}
\begin{lstlisting}[frame=single,mathescape,caption={Basic Newton Method for Square Root}]
  NewtonSquareRoot($N \in \R, x_0 \in \R, \tau \in (0,1)$):
      $n := 0$
      loop:
          $x_{n+1} := \tfrac{1}{2}(x_n + \tfrac{N}{x_n})$
          $\delta_n := |x_{n+1} - x_n|$
          if $\delta_n \leq \tau$:
              return $x_{n+1}$
          $n \mapsto n + 1$
\end{lstlisting}

Next we want to consider our initial estimate \(x_0\); it is prudent to first consider when our initial estimate will converge to the correct root. By looking at a graph of the function, and in particular the tangents to the curve, it would seem reasonable to wonder if \(\lim_{n\to\infty} x_n = \sqrt{N}\).

%THM%
\begin{SRNM Right-hand Convergence}
\label{THM_"SRNM Right-Hand Convergence"}
If \(x_0 \in \right(\sqrt{N}, \infty\right)\) and \(\left\{x_n : n\in\N\right\}\) is a sequnence of approximations of \(\sqrt{N}\) found via the Newton-Raphson Method, as detailed above, then:
\[\lim_{n\to\infty} x_n = \sqrt{N}\]
\end{SRNM Right-hand Convergence}

%PRF%
\begin{proof}
Suppose \(x_n > \sqrt{N}\), then
\begin{align*}
	x_{n+1} &= \frac{1}{2}\left(x_n + \frac{N}{x_n}\right)\\
		  &< \frac{1}{2}\left(x_n + \frac{N}{\sqrt{N}}\right) 
		  		&\mathrm{as } \sqrt{N} < x_n \implies \frac{1}{x_n} <
				\frac{1}{\sqrt{N}}\\
		  &= \frac{1}{2}\left(x_n + \sqrt{N}\right)\\
		  &< \frac{1}{2}(2x_n)\\
		  &= x_n
\end{align*}
Therefore we see that \(\left\{x_k : k \in [n, \infty) \cap \Z\right\}\) is a strictly decreasing sequence.\\
Now suppose that \(x_n \ge \sqrt{N}\) and then, for a contradiction, assume that \(x_{n+1} < \sqrt{N}\). We then see that:
\begin{align*}
	& \frac{1}{2}\left(x_n + \frac{N}{x_n}\right) < \sqrt{N}\\
	\implies & x_n + \frac{N}{x_n} < 2\sqrt{N}\\
	\implies & x_n^2 + N < 2\sqrt{N}x_n\\
	\implies & x_n^2 - 2\sqrt{N}x_n + N < 0\\
	\implies & \left(x_n - \sqrt{N}\right)^2 < 0
\end{align*}
This is a contradiction as \(x_n, \sqrt{n} \in \R \implies \left(x_n - \sqrt{N}\right)^2 \ge 0\).\\
Therefore \(x_n \ge \sqrt{N} \implies x_{n+1} \ge \sqrt{N}\).\\
Hence if \(x_0 > \sqrt{N}\), then it follows that \(\{x_n : n \in \N\}\) is a strictly decreasing sequence that is bounded below. Therefore by an elementary result from limit theory, we see that \(\lim_{n\to\inft} x_n = \inf\{x_n : n \in \N\}\).
\end{proof}\\

The most obvious choice for \(x_0\) would be \(N\), but we see that \(N \in (0,1)\), then \(N < \sqrt{N}\). In this case, we could choose \(x_0 = 1\) for the case that \(N \in (0,1)\). Therefore we can choose 
\[x_0 := \left\{\begin{array}{lcl}N &: &N \in\left(1,\infty\right)\\1 &: &N \in (0,1)\end{array}\right.\]\\

In our choice of \(x_0\), we have so far left out the cases where \(N \in \{0, 1}\). In both of these case we already know the correct answer, namely \(\sqrt{N} = N\) provided \(N \in {0, 1}\). Therefore we can exclude them from our calculations, as we can pre-asses the value of \(N\), simply returning the correct answer if one of these cases is encountered.\\

This then leads to an updated version of the above pseudocode:\\

%PCD%
\label{PCD_"Newton Square Root v1"}
\begin{lstlisting}[frame=single,mathescape,caption={Basic Newton Method for Square Root}]
  NewtonSquareRoot($N \in \Rpz, \tau \in (0,1)$):
      if $N \in \{0, 1\}$:
          return $N$
      if $N > 1$:
          $x_0 := N$
      else:
          $x_0 := 1$
      $n := 0$
      loop:
          $x_{n+1} := \tfrac{1}{2}(x_n + \tfrac{N}{x_n})$
          $\delta_n := |x_{n+1} - x_n|$
          if $\delta_n \leq \tau$:
              return $x_{n+1}$
          $n \mapsto n + 1$
\end{lstlisting}

\TODO{Write up examination and implementation of this pseudocode}\\

An alternative would be to use the integer square root method discussed in Section \ref{SUB_"Digit by Digit Method"} to improve our initial choice of \(x_0\). We will start by showing, that for intervals \(I \subset \Rp\), the first two criteria for quadratic convergence of the Newton Raphson method are met.

%THM%
\begin{SRNM NR1 and NR2}
\label{THM_"SRNM NR1 and NR2}
If \(I \subset \Rp\) then \(NR_1\) and \(NR_2\) are satisfied for \(f(x) = x^2 - N\)
\end{SRNM NR1 and NR2}

%PRF%
\begin{proof}
\(f(x) = x^2 - N \implies f'(x) = 2x \implies f''(x) = 2\)\\
Now as \(x \in \Rp \forall x \in I\), then it is obvious that \(f'(x) > 0\)\\
Therefore \(f'(x) \neq 0 \forall x \in I\), and so \(NR_1\) is satisfied.\\
As \(f''(x)\) is a constant function, then it is continuous on all of \(\R\).\\
Hence \(f''(x)\) is continuous \(\forall x \in I\) and so \(NR_2\) is satisfied.
\end{proof}

Now the integer square root function will always produce a root that is at most a distance of \(1\) from \(\sqrt{N}\); therefore we can consider \(I = [\sqrt{N} - 1, \sqrt{N} + 1]\). Now if \(N \le 1\), then \(I \seubset \Rp\) and so we cannot guarantee the satisfaction of \(NR_1\). Therefore we can proceed with our analysis of the case that \(N > 1\).\\

If \(N > 1\) we need to find when we can satisfy \(NR_3\). First, we remember that \(M := \sup{\left|\tfrac{f''(x)}{f'(x)}\right| : x \in I}\) and \(\epsilon_0 := \left|x_0 - \sqrt(N)\right|\). Then to satisfy \(NR_3\), we must have that \(M\epsilon_0 < 1\).\\

We can guarantee that \(\epsilon_0 \le 1\) because \(x_0 \in I\) from the integer square root algortihm; therefore it suffices to find the situation where \(M < 1\). As both \(f'\) and \(f''\) are continuous and non-zero on \(I\) it follows that \(M = \sup{x^{-1} : x \in I} = (\sqrt{N} - 1)^{-1}\). We then see that:
\begin{displaymath}
	\begin{align*}
		M < 1 &\iff \sqrt{N} - 1 > 1\\
			  &\iff \sqrt{N} > 2\\
			  &\iff N > 4
	\end{align*}
\end{displaymath}

Therefore we can get the following new choice for \(x_0\), and thus new pseudocode:
\begin{displaymath}
	x_0 := \left\{\begin{array}{lcl}
		1 &: &N \in (0,1)\\
		N &: &N \in (1,4]\\
		intSqrt(N) &: &N \in (4, \infty)
	\end{array}\right.
\end{displaymath}

%PCD%
\label{PCD_"Newton Square Root v2"}
\begin{lstlisting}[frame=single,mathescape,caption={Basic Newton Method for Square Root}]
  NewtonSquareRoot($N \in \Rpz, \tau \in (0,1)$):
      if $N \in \{0, 1\}$:
          return $N$
      if $N < 1$:
          $x_0 := 1$
      else:
          if $N \le 4$:
              $x_0 := N$
          else:
              $x_0 := $ IntSqrt($N$)
      $n := 0$
      loop:
          $x_{n+1} := \tfrac{1}{2}(x_n + \tfrac{N}{x_n})$
          $\delta_n := |x_{n+1} - x_n|$
          if $\delta_n \leq \tau$:
              return $x_{n+1}$
          $n \mapsto n + 1$
\end{lstlisting}

\TODO{Write up examination of different versions tried, such as using \(x_0 = N\), etc...}\\

If we consider any \(N \in \Rpz\), then \(\exists a \in \left[\frac{1}{2}, 1\right), b \in \Z : N = a \times 2^b\). Finding this value would be a hard as finding the logarithm of \(N\) base 2, but due to the representation of numbers within C, both standard C and MPFR have functions that allow us to extract these two values with minimal computational expenditure.\\

This helps as we can then narrow our problem, to only finding \(\sqrt{a} : a \in \left[\frac{1}{2}, 1\right)\), and then calculating 
\begin{displaymath}
	\sqrt{N} = \sqrt{a} \times 2^{\left\lfloor \frac{b}{2} \right\rfloor} \times \alpha \ \mathrm{where}\  
	\alpha = \left\{
		\begin{array}{lcl}
			1 & : & b \in 2\Z \\
			\sqrt{2} & : & b \in \Zp\setminus2\Z \\
			\frac{1}{sqrt{2}} & : & b \in \Zn\setminus\2\Z
		\end{array}\right.
\end{displaymath}

We then get the following algorithm, which implements this:

%PCD%
\begin{lstlisting}[frame=single,mathescape,caption={Newton Method for Square Root v3},label={PCD_"Newton Method for Square Root v3"}]
  NewtonSquareRoot($N \in \Rpz, \tau \in (0,1)$):
      Let $(a, b) :\in \left[\tfrac{1}{2}, 1\right)\times\Z$ s.t. $N = a*2^b$
      $x_0 := 1$
      if $b \equiv 0 \textrm{mod}\ 2$:
          $\alpha := 1$
      else:
          if $b > 0$:
              $\alpha := \sqrt{2}$
          else:
              $\alpha := \tfrac{1}{\sqrt{2}}$
      $n := 0$
      loop:
          $x_{n+1} := \tfrac{1}{2}(x_n + \tfrac{a}{x_n})$
          $\delta_n := |x_{n+1} - x_n|$
          if $\delta_n \leq \tau$:
              return $\alpha\cdot x_{n+1} \cdot 2^{\left\lfloor\frac{b}{2}\right\rceil}$
          $n \mapsto n + 1$
\end{lstlisting}

We must first consider the fact that the algorithm requires the pre-calculation of both \(\sqrt{2}\) and \(\tfrac{1}{\sqrt{2}}\), to be able to calculate all values. However, it turns out we can use the algorithm itself to generate these values as \(2 = \tfrac{1}{2} \cdot 2^2\), and as the exponent of 2 is even then the algorithm does not require \(\sqrt{2}\) for this computation. Similarly \(\tfrac{1}{2} = \tfrac{1}{2} \cdot 2^0\), which again is an even exponent. We can thus run our algorithm to find an arbitrarily accurate values for \(\sqrt{2}\) and \(\tfrac{1}{\sqrt{2}}\) to allow us to run the algorithm for other values.\\

With this observation can then consider \(N \in \left[\tfrac{1}{2}, 1\right)\). As this is a small range and, as per our previous algorithm, we use an initial guess of \(x_0 = 1\), then we can prove that our algorithm will converge quadratically to \(\sqrt{N}\).

%THM%
\begin{SRNM NR3 for v3}
\label{THM_"NR3 for v3"}
Algorithm \ref{PCD_"Newton Method for Square Root v3"}, satisfies the criteria of Theorem \ref{THM_"Quad Conv Newton"}, and thus has quadratic convergence to \(\sqrt{N}\).
\end{SRNM NR3 for v3}
\begin{proof}
To fulfill the criteria of Theorem \ref{THM_"Quad Conv Newton"}, we must find and interval \(I := [\sqrt{N}-r, \sqrt{N} + r]\) for some \(r \ge \epsilon_0\).\\

Consider \(\epsilon_0 = |\sqrt{N} - x_0| = 1 - \sqrt{N}\). We see that as \(N \ge \frac{1}{2}\) then \(\sqrt{N} \ge \sqrt{2}^-1\), and thus \(\epsilon_0 \le 1 - \sqrt{2}^-1\). Let us have \(r := 1 - \frac{1}{\sqrt{2}}\), and \(I\) as defined above.\\

If we look at the lower bound of \(I\), then we see that:
\begin{displaymath}
\begin{align*}
\sqrt{N} - r &\ge \frac{1}{\sqrt{2}} - (1 - \frac{1}{\sqrt{2}})\\
	&= \frac{2}{\sqrt{2}} - 1\\
	&= \sqrt{2} - 1 \\
	&> 0
\end{align*}
\end{displaymath}

Therefore we see that \(I \subset \Rp\), and so by Proposition \ref{THM_"SRNM NR1 and NR2} we get that \(\mathrm{NR}_1\) and \(\mathrm{NR}_2\) ar satisfied. It then remains to show that \(\mathrm{NR}_3\) is satisfied on \(I\).\\

Now by the definition in Theorem \ref{THM_"Quad Conv Newton"}, we have that \(M = \sup\left\{\frac{1}{2}\left|\frac{f''(x)}{f'(y)}\right| : x, y \in I\right\}\). We know that \(I\) is bounded, \(f''(x) = 2\) and \(f'(x) = 2x\) meaning that \(\frac{1}{2}\left|\frac{f''(x)}{f'(y)}\right| = \frac{1}{f'(x)}\) as \(x \in \Rp\).\\ 

Therefore our problem is reduced to finding \(\max\left\{\frac{1}{2x} : x \in I\right\}\), which is equivalent to finding \(\min\{x : x \in I\} = \sqrt{N} - r\). Therefore by passing this information back up the chain we get that \[M = \frac{1}{2(\sqrt{N} - r)}\]\\

Then we see that:
\begin{displaymath}
\begin{align*}
M\epsilon_0 &= \frac{1 - \sqrt{N}}{2(\sqrt{N} - r)}\\
	&\le \frac{1 - \frac{1}{\sqrt{2}}}{2(\sqrt{N} - r)} 
		& \textrm{as } \sqrt{N} \ge \frac{1}{\sqrt{2}}\\
	&\le \frac{1 - \frac{1}{\sqrt{2}}}{2(\frac{1}{\sqrt{2}}-r)}
		& \textrm{as } \sqrt{N} \ge \frac{1}{\sqrt{2}}\\
	&= \frac{1 - \frac{1}{\sqrt{2}}}{2(\frac{2}{\sqrt{2}} - 1)}\\
	&= \frac{1 - \frac{1}{\sqrt{2}}}{2\sqrt{2}(1-\frac{1}{\sqrt{2}})}\\
	&= \frac{1}{2\sqrt{2}}\\
	&< 1 & \textrm{as } 2\sqrt{2} > 1
\end{align*}
\end{displaymath}

As we have confirmed that \(M\epsilon_0 < 1\), then we have confirmed that \(\mathrm{NR}_3\) is satisfied on \(I\), and so the algorithm converges quadratically to the desired root.
\end{proof}

Using the previous proposition we can, similar to our previous methods, consider how many iterations would be needed to reach a required tolerance. To start we consider that, as mentioned in the proof or Theorem \ref{THM_"Quad Conv Newton"}, that \(\epsilon_n \le (M\epsilon_0)^{2^n - 1}\epsilon_0\).\\

We know that \(M\epsilon_0 \le \frac{1}{2\sqrt{2}}\) and that \(\epsilon_0 \le 1 - \frac{1}{\sqrt{2}}\), giving:
\[\epsilon_n \le \left(\frac{1}{2\sqrt{2}}\right)^{2^n - 1}\left(1 - \frac{1}{\sqrt{2}}\right)\]

Thus if we want to acheive a tolerance of \(\epsilon_n \le \tau\), then it suffices to find \(n \in \N_0\) such that:
\[\left(\frac{1}{2\sqrt{2}}\right)^{2^n - 1} \le \tau\]

Then,
\[(2^n - 1)\log\left(\frac{1}{2\sqrt{2}}\right) \le \log\left(\frac{\tau}{1 - \frac{1}{\sqrt{2}}}\right)\]

By noting that \(\log(\frac{1}{a}) = - \log(a)\), then we get
\[(1-2^n)\log(2\sqrt{2}) \le \log\left(\frac{\tau}{1 - \frac{1}{\sqrt{2}}}\right)\]

Once this is rearranged we get the following inequality:
\[2^n \ge \frac{\log\left(\frac{2(\sqrt{2} - 1)}{\tau}\right)}{\log(2\sqrt{2})}\]

By taking logarithms again and re-arranging we get that
\[n \ge \frac{\log\left(\frac{\log\left(\frac{2(\sqrt{2} - 1)}{\tau}\right)}{\log(2\sqrt{2})}\right)}{\log(2)} = \log_2\left(\log_{2\sqrt{2}}\left(2\frac{\sqrt{2} - 1}{\tau}\right)\right)\]

Now for an example, suppose we want to know how many iterations we need to perform to find \(\sqrt{N}\) to within 10 decimal places, i.e. \(\tau = 10^-10 = 0.0000000001\). We remember that \(\sqrt{N} \in [\frac{1}{2}, 1)\), and then we will apply transformations to this value afterwards, therefore this is equivalent to finding 10 significant digits of accuracy for our square root (ignoring any loss of accuracy that may arrise from multiplications afterwards).\\

Now in this case we want to find \(n \in \N\) such that \(n \ge log_2(log_{2\sqrt{2}}(2\cdot10^{10}(\sqrt{2}-1)))\). Using Wolfram Alpha to calculate this value we get that we need \(n \ge 4.457144...\) and so we can take \(n = 5\). This means that we could modify our algorithm and implementation to do 5 fixed iterations of Newton's Method to guarantee at least 10 decimal places of accuracy.\\

In terms of efficiency versus accuracy tradeoff modifying the problem thus would improve it's efficiency by removing, now unneccesary, calculation and comparrison of \(\delta_n\) at each stage. However this does need a fixed guaranteed accuracy, and therefore such a program would no longer be suitable if we needed to calculate a square root accurate to 15 decimal places.\\

Below is a table that lists the minimum \(n \in \N\) such that \(n\) satisfies our inequality, where our tolerance is \(10^k\) for some \(k \in \N\). This will give us the maximum number of iterations that must be performed for the required accuracy.

%TBL%
\begin{center}
\begin{tabular}{|p{3cm}|p{3cm}|}
\hline
\(k : \tau = 10^k\) & \(n\)\\\hline
5 & 4 \\\hline
10 & 5 \\\hline
100 & 8 \\\hline
1,000 & 12 \\\hline
1,000,000 & 22\\\hline
\end{tabular}
\end{center}

%SUB%
\subsection{Newton's Inverse Square Root Method}
\label{SUB_"Newton's Inverse Square Root Method"}

\theoremstyle{plain}
\newtheorem{Inv Sqrt Quad Conv}{Proposition}[subsection]

As discussed in Section \ref{#SEC#}, computers are more efficient at multiplication over division. We would therefore prefer to find a way of utilising Neton's Method without having to perform any costly division operations.\\

If we consider \(f(x) = N - \frac{1}{x^2}\) then if \(x^\ast\) is a solution to \(f(x) = 0\) we see that \(x^\ast = \frac{1}{\sqrt{N}}\). As \(f'(x) = \frac{2}{x^3}\), then the Newton's Method, will give \[x_{n+1} = x_n - \frac{N - \frac{1}{x_n^2}}{\frac{2}{x_n^3}} = x_n\left(\frac{3}{2} - \frac{N}{2}x_n^2\right)\] where \(x_0\) is a given initial guess. As can be seen this algorithm requires no division if we multiply by real constants rather than the division implied above.\\

We can then consider that, similar to Algorithm \ref{PCD_"Newton Method for Square Root v3"}, any \(N\) can be represented as \(a \cdot 2^b\) where \(a \in \left[\tfrac{1}{2}, 1\right)\). This will, again allow us to narrow our problem to a known range of values, by using the following transormations.
\begin{displaymath}
\begin{align*}
N = a \cdot 2^b &\implies \tfrac{1}{N} = \tfrac{1}{a} \cdot 2^{-b}\\
	&\implies \tfrac{1}{\sqrt{N}} = \tfrac{1}{a}\cdot2^{\lfloor\frac{-b}{2}\rceil} \cdot \alpha
		&\alpha := \left\{
			\begin{array}{lcl}
				1 & : & b \equiv 0 \mod 2\\
				\sqrt{2} & : & b \equiv 1 \mod 2, b \in \Zn\\
				\frac{1}{\sqrt{2}} & : & b \equiv 1 \mod 2, b \in \Zp\\
			\end{array}\right.\\
	& \implies \sqrt{N} = N \cdot \tfrac{1}{\sqrt{a}} \cdot 2^{\lfloor\frac{-b}{2}\rceil} \cdot \alpha
\end{align*}
\end{displaymath}

Therefore we only need to calculate inverse square roots for values of \(N\) in the range \([\tfrac{1}{2}, 1)\). Thus giving us the following algorithm:\\

%PCD%
\begin{lstlisting}[frame=single,mathescape,caption={Newton Inverse Square Root Method},label={PCD_"Newton Inverse Square Root"}]
  NewtonInvSquareRoot($N \in \Rpz, \tau \in (0,1)$):
      Let $(a, b) :\in \left[\tfrac{1}{2}, 1\right)\times\Z$ s.t. $N = a*2^b$
      $x_0 := 1$
      if $b \equiv 0 \textrm{mod}\ 2$:
          $\alpha := 1$
      else:
          if $b > 0$:
              $\alpha := \tfrac{1}{\sqrt{2}}$
          else:
              $\alpha := \sqrt{2}$
      $n := 0$
      loop:
          $x_{n+1} := x_n(\tfrac{3}{2} + \tfrac{a}{2}x_n^2)$
          $\delta_n := |x_{n+1} - x_n|$
          if $\delta_n \leq \tau$:
              return $N\cdot\alpha\cdot x_{n+1} \cdot 2^{\left\lfloor\frac{-b}{2}\right\rceil}$
          $n \mapsto n + 1$
\end{lstlisting}

With this method we can once again consider it's convergence properties, in particular does it satisfy the criteria for quadratic convergence in Theorem \ref{THM_"Quad Conv Newton"}.

%THM%
\begin{Inv Sqrt Quad Conv}
\label{THM_"Inv Sqrt Quad Conv"}
Algorithm \ref{PCD_"Newton Inverse Square Root"} satisfies the criteria of Theorem \ref{THM_"Quad Conv Newton"}, and thus has quadratic convergence to \(\sqrt{N}\).
\end{Inv Sqrt Quad Conv}
\begin{proof}
We know that we only need to consider \(N \in [\frac{1}{2}, 1)\), and therefore \(\sqrt{N}^{-1} \in (1, \sqrt{2}]\). Also \(x_0 = 1\) and so we see that 
\[\epsilon_0 = |x_0 - \sqrt{N}^{-1}| = \sqrt{N}^{-1} - x_0 \le \sqrt{2} - 1\]

Now let \(r := \epsilon_0 = \sqrt{N} - 1\) and \(I := [\sqrt{N}^{-1} - r, \sqrt{N}^{-1}]\). If we consider the lower bound of I we see that \(\sqrt{N}^{-1} - (\sqrt{N}^{-1} - 1) = 1\), and in particular \(0 \notin I\).\\

Next we know that \(f(x) = N - x^{-2}\), and therefore we get \(f'(x) = 2x^{-3}\), \(f''(x) = -6x^{-4}\). It is obvious that \(\nexists x \in \R : f'(x) = 0\), which means that \(f'(x) \neq 0 \forall x \in I\) and so \(\mathrm{NR}_1\) is satisfied. Also as \(f''\) is only discontinuous at \(x = 0\) and \(0 \notin I\), then \(f''(x)\) is continuous \(\forall x \in I\), meaning this satisfies \(\mathrm{NR}_2\).\\

Now \(M = \sup\left\{\tfrac{1}{2}\left|\frac{2x^3}{6y^4}\right| : x, y \in I\rifht\}\), we can simplify the function we are trying to minimise to get \(\tfrac{1}{6}\frac{x^3}{y^4}\). It is obvious that in order to maximise this function we should find the largest possibe \(x\) and smallest possible \(y\), as both are positive. Hence by taking \(x = \sqrt{N}^{-1} + r\) and \(y = 1\), then \(M = \frac{1}{6}(2\sqrt{N}^{-1} - 1)^3 \le \frac{1}{6}(2\sqrt{2} - 1)^3\).\\

Now we consider \(M\epsilon_0\):\\

\begin{displaymath}
\begin{align*}
	M\epsilon_0 &=\frac{1}{6}(2\sqrt{N}^{-1} - 1)^3(\sqrt{N} - 1)\\
		&\le \frac{1}{6}(2\sqrt{2} - 1)^3(\sqrt{2} - 1)\\
		&\approx 0.42199376\ldots\\
		&< 1
\end{align*}
\end{displaymath}

Therefore as \(M\epsilon_0 < 1\) we have satisfied \(\mathrm{NR}_3\), and as such we have quadratic convergence of our method to \(\sqrt{N}^{-1}\).
\end{proof}
 
%SUB%
\subsection{Comparrison of Methods}

We have observed several methods that can be used to calculate Square Roots, and so now we will see how the methods compare to each other in practice. The exact root method that we first discussed is the hardest to compare to the other methods as it works in a very different manner to the others. For now we will merely observe that it is an inefficient method that can be will be shown to tae longer than the others.\\

Second we need to compare the different methods discussed for the Newton Square Root method. As they work in the same general method, we really only need to test the computation time of the different methods. To do this we will be testing 1000 values in the range \((0,1000)\) and will calculate each of these values 100000 times, accurate to within a tolerance of \(10^-1)\), for each method to give the most accurate results. The table below gives the calculated results:

{\fontfamily{pcr}\selectfont
%TBL%
\begin{center}
\begin{tabular}{|l|r|r|r|}
\hline
	&\codeinline{mpfr\_newton\_sqrt\_v1} &\codeinline{mpfr\_newton\_sqrt\_v2}
	&\codeinline{mpfr\_newton\_sqrt\_v3}\\\hline
	\textsf{Total time:} & 10.507s & 12.707s & 8.188s\\\hline
	\textsf{Average time:} & 0.010s & 0.012s & 0.008s\\\hline
	\textsf{Minimum time:} & 0.003s & 0.004s & 0.005s\\\hline
	\textsf{Maximum time:} & 0.016s & 0.021s & 0.016s\\\hline
\end{tabular}
\end{center}}

Here we see that our third method, as expected, is the fastest of the proposed methods and so we will use this method going forwards. One unexpected result is that the second method is actually slwer than the first, which is likely due to the extra conversions, comparrisons and method calls; this slows down the execution more than it is sped up by reduction in number of iterations required.\\

Now for the comparrison of methods we will be comparing modified versions of Algorithms \ref{PCD_"Square Root Bisection Method fixed bounds"}, \ref{PCD_"Newton Method for Square Root v3"} and \ref{PCD_"Newton Inverse Square Root"}, which will execute for a given number of steps, rather than testing for the approximate error. To do this we need to consider how many iterations each method needs  to reach a particular number of decimal places of accuracy.\\

We have seen the required number of iterations for a tolerance \(\tau = 10^{-k} : k \in \N\), for both the bisection and basic newton square root methods, and similar to the basic newton method, we can show that for the inverse newton method we are looking for \(n \in \N\) that satisfies the following inequality:

\[n > \log_2\left(\log_{\tfrac{1}{6}(\sqrt{2} - 1)(2\sqrt{2} - 1)^3}\left(\frac{\tau}{\sqrt{2} - 1}\right)\right) - 1\]

This gives the following table:

%TBL%
\begin{center}
\begin{tabular}{|p{3cm}|p{3cm}|p{3cm}|p{3cm}|}
\hline
\(k : \tau = 10^k\) & Bisection Method & Newton Method & Inverse Newton\\\hline
5 & 		16 &		4 &		4\\\hline
10 & 		33 &		5 &		5\\\hline
100 & 		332 &		8 &		9\\\hline
1,000 & 	3321 &		12 &	12\\\hline
1,000,000 & 3219280 &	22 &	22\\\hline
\end{tabular}
\end{center}

To show the above in action we have the table below which shows the convergence of all 3 methods to \(\sqrt{0.75} \approx 0.86602540378\), for differnct numbers of iterations \(n\) with the bold digits being those correct:

{\fontfamily{pcr}\selectfont
%TBL%
\begin{center}
\begin{tabular}{|c|l|l|l|}
\hline
\(n\) & \textrm{bisectSquareRoot(\(0.75, n\))}
	  & \textrm{NewtonSquareRoot(\(0.75, n\))}
	  & \textrm{NewtonInvSquareRoot(\(0.75, n\))}\\\hline
0 & \textbf{0.}500000000000000000
& \textbf{1}.000000000000000000
& \textbf{0.}750000000000000000\\\hline
1 & \textbf{0.}750000000000000000
& \textbf{0.8}75000000000000000
& \textbf{0.8}43750000000000000\\\hline
2 & \textbf{0.8}75000000000000000
& \textbf{0.8660}71428571428603
& \textbf{0.86}5173339843750000\\\hline
3 & \textbf{0.8}12500000000000000
& \textbf{0.86602540}5007363691
& \textbf{0.86602}4146705512976\\\hline
4 & \textbf{0.8}43750000000000000
& \textbf{0.866025403784438}596
& \textbf{0.86602540378}1701674\\\hline
5 & \textbf{0.8}59375000000000000
& \textbf{0.866025403784438}596
& \textbf{0.866025403784438}596\\\hline
6 & \textbf{0.86}7187500000000000
& \textbf{0.866025403784438}596
& \textbf{0.866025403784438}596\\\hline
\end{tabular}
\end{center}}

If we compare the methods so that they guarantee an accuracy of 10 decimal places, then we will be able to see their relative efficiency. In particular we will again be testing the three methods using 1000 values in the range \((0,1000)\), and calculating the square root of each of these values 10000 times for each method; further we will be including the the digit by digit method and the built-in C \codeinline{sqrt} function. The results calculated are present in the following table:

{\fontfamily{pcr}\selectfont
%TBL%
\begin{center}
\begin{tabular}{|l|r|r|r|r|r|}
\hline
	&\codeinline{root\_digits\_precise} 
	&\codeinline{bisect\_sqrt}
	&\codeinline{newton\_sqrt}
	&\codeinline{newton\_inv\_sqrt}
	&\codeinline{builtin\_sqrt}\\\hline
	\textsf{Total time:} & 227.620s & 2.520s & 1.028s & 0.646s & 0.072s
	\\\hline
	\textsf{Average time:} & 0.227s & 0.002s & 0.001s & 0.000s & 0.000s
	\\\hline
	\textsf{Minimum time:} & 0.160s & 0.002s & 0.000s & 0.000s & 0.000s
	\\\hline
	\textsf{Maximum time:} & 0.429s & 0.004s & 0.004s & 0.001s & 0.000s
	\\\hline
\end{tabular}
\end{center}}

Here we see the expected result that the digit by digit method is the least efficient method, taking two orders of magnitude more time than the second least efficient. We also see that while the two differnt newton methods are similar in time, and that even though they each performed the same number of iterations, the inverse square root method is the faster; this is due to the method having no division operations to perform. The quickest is of course the built-in \codeinline{sqrt} function from C, this is due to an implenetation that uses several low-level features of the C language to acheive the displayed level of performance.\\

In conclusion we can say that the best method that we have considered is Algorithm \ref{#ALG#} which has rapid convergence to the sought square root, while also having fast execution. However if we are in a situation where we require large numbers of digits of accuracy, and yet do not have a suitable floating point types large enough to store these values, then the digit by digit method can be used to get an aribitrary number of digits accuracy.

\section{Logarithms and Exponentials}

Exponentiation is the operation of calculating \(x^y\) where \(x\) and \(y\) are members of some field, for the purposes of this document we will be considering \(x, y \in \R\). This operation is widely used by many different branches of mathematics and industry, for example many real world phenomena can be modelled by exponentials\cite{ONL_ExpPres}; we therefore need to calculate \(x^y\) quickly and efficiently.\\

The first thing we consider is that \(x^y\) when \(x \in \mathbb{R}^{-}\) and \(y \in \R\setminus\Z\) is not well-defined on \(\R\), and requires consideration of the function on the complex plane. Due to this we will not be considering negative numbers to non-integer bases; in particular, unless stated otherwise, we will be assuming that \(x \in \Rpz\).\\

Now we also know that \(x^{-y} = \frac{1}{x^y}\) when \(y \in \R\), and as such we will also be restricting this section to the assumption that \(y \in \Rpz\). Further we consider the following facts: 

\[x^0 = 1\quad \forall\: x \in \Rpz\]
\[0^y = 0\quad \forall\: y \in \Rp\]

If we take out these known trivial cases then we can restrict this section to considering only \((x, y) \in (\Rp)^2\).\\

Now if we have \(y \in \Rp\) then it follows that \(\exists (a, b) \in \Zpz \times [0,1)\) such that \(y = a + b\). This allows us to use the identity that \(x^{m+n} = x^mx^n\) to consider the following two cases separately:

%EQN%
\begin{equation}
\label{EQN_"exp case 1"}
	x^a : a \in \Zpz
\end{equation}
\begin{equation}
\label{EQN_"exp case 2"}
	x^b : b \in [0,1)
\end{equation}

%SUB%
\subsection{Calculating \(x^a\)}

As we know that \(a \in \Zpz\), then we know that \(x^a = \underbrace{x\times \cdots \times x}_a\); i.e. the problem is equivalent to finding \(x\) multiplied with itself \(a\) times. As we are only dealing with \(a \in \Zpz\), then we will be considering \(x \in \R\) as we can calculate exponentials of negative numbers.\\

The naive way to go about calculating \(x^a\) is to simply perform the multiplication of \(x\) by itself \(a\) times. The algorithm for that can be seen below:

%PCD%
\begin{lstlisting}[caption={Naive integer exponentiation},label={PCD_"Naive int exp"}]
  naive_int_exp($x \in \R, a \in \Zpz$):
      $n := 0$
      $z := 1$
      while $n < a$:
          $z \mapsto x\cdot z$
      return $z$
\end{lstlisting}

This algorithm is very simple and has complexity of \(\bigO(a)\), which is a reasonably low complexity, but still has the chance to grow large as \(a\) grows. Instead we can consider a more informed approach, in particular we know that either \(2 \mid a\) or \(2 \nmid a\), which then gives us the following:
\:
\begin{displaymath}
	x^a = \left\{\begin{array}{lcl}
		(x^2)^{\tfrac{a}{2}} & : & 2 \mid a\\
		x \cdot (x^2)^{\tfrac{a-1}{2}} & : & 2 \nmid a
	\end{array}\right.
\end{displaymath}

We can use this fact to build a recursive method of calculating \(x^a\), where we repeatedly call the method from within itself. To ensure the method ends correctly we need to identify a base case for the recursion, i.e. where the process stops and returns the correct value. We can see that eventually the above will reach the point where \(a = 0\), in which case we know that \(x^0 = 1\); this will be the base case of our recursion.\\

We want to ensure that the algorithm will terminate, which we can do by seeing that it terminates when \(a = 0\) and then considering \(a \in \Zp\). Now if \(2 \mid a\) then \(\tfrac{a}{2} \in \Zp\) and also \(\tfrac{a}{2} < a\), similarly if \(2 \nmid a\) then \(\tfrac{a-1}{2} \in \Zpz\) because \(a \ge 1\) and also \(\tfrac{a-1}{2} < a\). Thus we see that the sequence produced by \(a \in \Zp\) is a strictly decreasing sequence that is bounded below by 0 and thus we must eventually reach 0, meaning the algorithm terminates.\\

Instead of a recursive algorithm that calls itself, the algorithm below is an iterative version which performs the same function:

%PCD%
\begin{lstlisting}[caption={Exponentiation by squaring},label={PCD_"exp by square"}]
  exp_by_squaring($x \in \R, a \in \Zpz$):
      $n := a$
      $z := 1$
      $\hat{x} := x$
      while $n > 0$:
          if $2 \nmid n$:
              $z \mapsto \hat{x} \times z$
              $n \mapsto n - 1$
          $\hat{x} \mapsto \hat{x}^2$
          $n \mapsto \tfrac{n}{2}$
      return $z$
\end{lstlisting}

This algorithm is much more efficient than Algorithm \ref{PCD_"Naive int exp"} due to the number of times the inner loop is executed. The inner loop drives \(a\) towards 0 by dividing by 2 each step, this means that as \(a = \bigO(2^{\log_2(a)})\), then this goal is achieved in only \(\log_2(a)\) loops. Therefore the complexity of this algorithm is \(\bigO(\log_2(a))\), which is an improvement upon the previous algorithm's complexity of \(\bigO(a)\).\\

To see this difference in efficiency in action the following table shows the times taken for each method when comparing 1000 different pairs of values \((x, a) \in [0, 10]\times([0,100]\cap\Z)\). With these values we calculated \(x^a\) using both methods 100000 times to get the following results:

%TBL%
{\fontfamily{pcr}\selectfont
\begin{center}
\begin{tabular}{|c|r|r|r|r|}
\hline
&\textsf{Total time:} & \textsf{Average time:} & \textsf{Minimum time:}
	&\textsf{Maximum time:}\\\hline
\codeinline{naive\_int\_exp}& 16.800s & 0.016s & 0.000s & 0.037s\\\hline
\codeinline{squaring\_int\_exp} & 2.593s & 0.002s & 0.000s & 0.004s\\\hline
\end{tabular}
\end{center}}

With this we will move on to further subsections as there are few improvements that can be made on an \(\bigO(\log_2(a))\) algorithm, particularly in this instance.

%SUB%
\subsection{Calculating \(x^b\)}

If we have \(b \in (0, 1)\), then we obviously can't use the our previous subsection for calculating \(x^y\). The most common way of calculating such exponentiation is by considering that \(x = e^{\ln(x)}\) and thus \(x^b = (e^{\ln(x)})^b = e^{b\ln(x)}\); however this now raises the problem of how to calculate both \(e^\alpha\) and \(\ln(\beta)\). The following will deal with how to calculate these values and thus use them in conjunction to calculate \(x^b\).\\

%SUB%
\subsection{Naive Method}

The mathematical constant \(e\) has been known since the early 1600s and was originally calculated by Jacob Bernoulli, and was studied by Leonhard Euler, where it appeared in Euler's Mechanica in 1736. While several possible equivalent definitions of \(e\) exist the most common such definition is that \(e := \lim_{n \to \infty}(1 + \tfrac{1}{n})^n\).\\

If we now consider the definition of \(e\) and also consider \(e^x\), then we can show that \(e^c = \lim_{n\to\infty}(1+\tfrac{x}{n})^n\). This gives us our first basic method of how to calculate \(e^x\):

%PCD%
\begin{lstlisting}[caption={Baisc Method for calculating \(e^x\)},label={PCD_"basic exp"}]
  basic_exp($x \in \R, n \in \N$)
      return $(1 + \tfrac{x}{n})^n$
\end{lstlisting}

If we consider \((1+\tfrac{x}{n})^n\) as a function of a continuous \(n\) then we can find the following derivation:

\begin{align*}
	\frac{d}{dn}\left[(1 + \tfrac{x}{n})^n\right]
		&= (1+\tfrac{x}{n})^n\frac{d}{dn}\left[n\ln(1+\tfrac{x}{n})\right]\\
	&= (1+\tfrac{x}{n})^n\left(\frac{d}{dn}[n]\ln(1+\tfrac{x}{n}) 
		+ n\frac{d}{dn}\left[\ln(1+\tfrac{x}{n})\right]\right)\\
	&=(1+\tfrac{x}{n})^n\left(\ln(1+\tfrac{x}{n}) 
		+ \frac{n}{1 + \tfrac{x}{n}}\frac{d}{dn}[1 + \tfrac{x}{n}]\right)\\
	&=(1+\tfrac{x}{n})^n\left(\ln(1+\tfrac{x}{n}) - \frac{x}{n + x}\right)\\
	&=\frac{(1+\frac{x}{n})^n}{x + n}((x + n)\ln(1 + \tfrac{x}{n}) - x)
\end{align*}

By the last line of this we can see that because \((x, n) \in (\Rp)^2\) then \(\ln(1 + \frac{x}{n}) > 0\) and thus we conclude that \((x+n)\ln(1+\frac{x}{n}) - x > 0\). Therefore we see that \(\frac{d}{dn}\left[(1+\frac{x}{n})^n\right] > 0\) for all \((x, n) \in \Rp^2\), and in particular this means that \((1+\frac{x}{n})^n < (1+\frac{x}{n+1})^{n+1}\ \forall\: n \in \N\).\\

One consequence of this is that \((1+\frac{x}{n})^n < e^x\ \forall\: n \in \N\), therefore we can define the error of algorithm \ref{PCD_"basic exp"} as \(\epsilon_N := |e^x - (1+\frac{x}{n})^n| = e^x - (1+\frac{x}{n})^n\). Now as \(\lim_{n\to\infty}(1+\frac{x}{n})^n = e^x\) then we see that \(\lim_{n\to\infty}\epsilon_n = 0\), and thus our algorithm is correct and valid for approximating \(e^x\).\\

Next we see that this method, while simple, approximates \(e^x\) very poorly. In particular the table below shows the approximation of \(e^{0.75}\) for different values of \(n\), where the bold digits are the correctly approximated digits.

%TBL%
{\fontfamily{pcr}\selectfont
\begin{center}
\begin{tabular}{|l|l|}
\hline
\(n\) & \textsf{Approximation of \(e^{0.75}\)}\\\hline
1 & 1.800000000000000044\\\hline
10 & \textbf{2.}158924997272786787\\\hline
100 & \textbf{2.2}18468215957572747\\\hline
1000 & \textbf{2.22}4829248807374831\\\hline
10000 & \textbf{2.225}469716120127850\\\hline
100000 & \textbf{2.2255}33806810873500\\\hline
1000000 & \textbf{2.225540}216319864358\\\hline
10000000 & \textbf{2.225540}857275162929\\\hline
100000000 & \textbf{2.22554092}1370736781\\\hline
1000000000 & \textbf{2.22554092}7780294606\\\hline
\end{tabular}
\end{center}}

With this table we see that the method very poorly approximates \(e^x\), requiring a very large \(n\) to get just a few digits of accuracy. While this does not require more calculations from the method, requiring this large a value of \(n\) can lead to inaccuracies in the implementation of the algorithm using \codeinline{double} data types in C.\\

In general there are better methods of approximating \(e^x\) and also \(\ln(x)\), which while requiring more calculations are much more accurate than the most basic method presented here.

%SUB%
\subsection{Taylor Series Method}

\theoremstyle{plain}
\newtheorem{nat log dif}{Proposition}[subsection]
\newtheorem{log convergence}[nat log dif]{Proposition}

If we take the elementary result from calculus that \(\frac{d}{dx}e^x = e^x\), then we can calculate the Maclaurin series of \(e^x\). By the definition of a Maclaurin series we know that the series expansion of \(e^x\) about 0 is 

\[\sum_{k=0}^\infty \frac{d^{(k)}}{dx^k}[e^x](0)\frac{x^k}{k!}\]

As \(\frac{d^{(k)}}{dx^k}[e^x] = e^x\ \forall\: k \in \Zpz\) and \(e^0 = 1\) then we see that the series becomes

\[\sum_{k=0}^\infty \frac{x^k}{k!}\]

Using this we see that \(e^x \approx \sum_{k=0}^n \frac{x^k}{k!}\), which gives the following method for approximating \(e^x\):

%PCD%
\begin{lstlisting}[caption={Taylor Method for calculating \(e^x\)},label={PCD_"taylor exp"}]
  taylor_exp($x \in \R, n \in \Zpz$)
      $t = 1$
      $z = 1$
      $k = 1$
      while $k < n$:
          $t \mapsto \frac{t\cdot x}{n}$
          $z \mapsto z + t$
          $k \mapsto k + 1$
      return $z$
\end{lstlisting}

This allows us to calculate \(e^x\) more efficiently, and we can see that the error of the approximation is easy \(\epsilon_n := |e^x - \sum_{k=0}^n\frac{x^k}{k!}| \le \frac{|x|^{n+1}}{(n+1)!}\) for all \(n \in \Zpz\). While we can't guarantee the size of \(x\) in general we will consider \(x \in (0,1)\) for the purposes of analysing this function.\\

As \(x \in (0,1)\) then it follows that \(x < 1\) and thus we can see that \(\epsilon_n < \frac{1}{n!}\ \forall\: n \in \Zpz\). Using this we can see that to use our method such that the error is at most \(\tau_d := 10^{-d}\), then we need to find \(n \in \Zpz : \frac{1}{n!} < \tau_d\). The table below shows some values for \((n, d)\) pairs such that \(n\) is the smallest positive integer such that \(\frac{1}{n!} < \tau_d\):

%TBL%
\begin{center}
\begin{tabular}{|l|l|}
\hline
\(d \in \N\) 
	& \(\textrm{arg}\min\left\{n \in \N : n! > 10^d\right\}\)\\\hline
1 & 4\\\hline
10 & 14 \\\hline
100 & 70 \\\hline
1000 & 450\\\hline
\end{tabular}
\end{center}

Therefore we can guarantee 100 digits of accuracy with an input of \(n \ge 70\) and 1000 digits of accuracy with \(n \ge 450\), this is much less than our previous method where an input of \(n = 1000\) only gave 2 decimal places of accuracy.\\

The inverse of the function \(z = e^x\) is the logarithm function \(\ln(z) = x\), which we can again consider for Taylor Series expansion. First we will show the result from calculus that \(\frac{d}{dx}[\ln(x)] = \frac{1}{x}\):

%THM%
\begin{nat log dif}
\[\frac{d}{dx}[\ln(x)] = \frac{1}{x}\]
\end{nat log dif}
\begin{proof}
We will prove this from the first principles using the definition that \(\frac{d}{dx}[f(x)] = \lim_{h\to 0} \frac{f(x + h) - f(x)}{h}\)

\begin{align*}
	\frac{d}{dx}[\ln(x)] 
		&= \lim_{h\to 0}\frac{\ln(x + h) - \ln(x)}{h}\\
		&= \lim_{h\to 0}\frac{\ln(1 + \frac{h}{x})}{h}\\
		&= \lim_{h\to 0}\ln\left((1+\frac{h}{x})^{\frac{1}{h}}\right)
\end{align*}

If we let \(u := \frac{h}{x}\), then we get that \(ux = h\) and \(\frac{1}{h} = \frac{1}{ux}\). Also \(\lim_{h\to 0}u = 0\), and so we get the following:

\begin{align*}
	\frac{d}{dx}[\ln(x)]
		&= \lim_{u\to 0}\ln((1+u)^{\frac{1}{ux}})\\
		&= \frac{1}{x}\lim_{u\to 0}\ln((1+u)^{1/u})\\
\end{align*}

If we now let \(n := \frac{1}{u}\) and consider that \(\lim_{u\to 0} n = \infty\), then our derivative becomes:

\begin{align*}
	\frac{d}{dx}\ln(x) 
		&= \frac{1}{x}\lim_{n\to\infty}\ln((1 + \frac{1}{n})^n)\\
		&= \frac{1}{x}\ln(\lim_{n\to\infty}(1+\frac{1}{n})^n)\\
		&= \frac{1}{x}\ln(e) &\textrm{by the definition of \(e\)}\\
		&= \frac{1}{x}
\end{align*}
\end{proof}

Now we know that \(\frac{d^k}{dx^k}[\frac{1}{x}] = (-1)^kk!x^{-k-1}\), and thus we can build up a Taylor Series expansion. In this case, rather than centring the series about \(x=0\) for a Maclaurin series we can instead centre the series around \(x=1\) which gives the following series expansion for \(\ln(x)\):

\begin{align*}
	\sum_{k=0}^\infty \frac{\frac{d^k}{dx^k}[\ln(x)](1)}{k!}(x-1)^k
		&=\ln(1) + \sum_{k=1}^\infty\frac{\frac{d^{k-1}}{dx^{k-1}}
			[x^{-1}](1)}{k!}(x-1)^k\\
		&=\sum_{k=1}^\infty\frac{[(-1)^{k-1}(k-1)!x^{-k}](1)}
			{k!}(x-1)^k\\
		&=\sum_{k=1}^\infty\frac{(-1)^{k-1}}{k}(x-1)^k\\
		&=-\sum_{k=1}^\infty\frac{(1-x)^k}{k}
\end{align*}
				
We know that \(\ln(x) = -\sum_{k=1}^\infty\frac{(1-x)^k}{k}\) when the series \(\sum_{k=1}^\infty\frac{(1-x)^k}{k}\) converges. We thus need to know when the sum converges.

%THM%
\begin{log convergence}
The series \(\sum_{k=1}^\infty\frac{(1-x)^k}{k}\) converges when \(x \in (0,2)\).
\end{log convergence}
\begin{proof}
Let \(a_k := \frac{(1-x)^k}{k}\). We will proceed by using the ratio test to show when the series converges absolutely. The test states that the series converges when \(\lim_{k\to\intfy} \left|\frac{a_{k+1}}{a_k}\right| < 1\).\\

Now we can consider the following derivation:

\begin{align*}
	\lim_{k\to\infty}\left|\frac{a_{k+1}}{a_k}\right|
		&=\lim_{k\to\infty}\left|\frac{\frac{1}{k+1}(1-x)^{k+1}}
			{\frac{1}{k}(1-x)^k}\right|\\
		&=\lim_{k\to\infty}\left|\frac{k}{k+1}(1-x)\right|\\
		&=|1-x|\lim_{k\to\infty}\left|\frac{k}{k+1}\right|\\
		&=|1-x|
\end{align*}

Therefore our series converges when:
\begin{align*}
	|1-x| < 1 &\iff -1 < 1 - x < 1\\
		&\iff -1 < x - 1 < 1\\
		&\iff 0 < x < 2
\end{align*}

Hence \(\sum_{k=1}^\infty \frac{(1-x)^k}{k}\) converges when \(x \in (0, 2)\).
\end{proof}

Now as we can't know if \(x \in (0,2)\) then we can consider that \(\forall\: x \in \Rp\:\: \exists\: (a, b) \in [\frac{1}{2}, 1)\times\Z : x = a\cdot2^b\); thus we see that \(\ln(x) = \ln(a\cdot2^b) = b\ln(2) + \ln(a)\). As previously noted in Section \ref{SUB_"Code and Computers used"} this operation, while theoretically complex, is simple to calculate for most computers by how the represent floating point values.\\

With this we can then use the following method to approximate \(\ln(x)\) by the Taylor polynomial \(-\sum_{k=1}^n\frac{(1-x)^k}{k}\):

%PCD%
\begin{lstlisting}[caption={Taylor Method for calculating \(\ln(x)\)},label={PCD_"taylor log"}]
  taylor_nat_log($x \in \Rp, n \in \N$):
      Find $(a, b) \in [\tfrac{1}{2}, 1)\times \Z$ such that $x = a\cdot2^b$
      $y := 1 - a$
      $t := y$
      $z := y$
      $k := 1$
      while $k < n$:
          $t \mapsto t\cdot y$
          $z \mapsto z + \tfrac{t}{k}$
          $k \mapsto k + 1$
      return $b\ln(2) - z$
\end{lstlisting}

The first thing to consider for the above method is how to calculate \(\ln(2)\). It is not possible to directly calculate \(\ln(2)\) using the above algorithm as \(2 = \frac{1}{2}\cdot2^2\), however \(\frac{1}{2} = \frac{1}{2}\cdot2^0\) and so we do not need to know \(\ln(2)\) to calculate \(\ln(\frac{1}{2})\). We can see that \(\ln(2) = -\ln(\frac{1}{2})\), and so we can calculate our constant value \(\ln(2)\) to be used in the algorithm by using the algorithm itself.\\

Now similar to previous Taylor approximations the final error of our approximation using the above method is \(\epsilon_n := |\ln(x) - \textrm{taylor\_log(}x,n\textrm{)}|\). As the next term of the approximation would be \(\tfrac{(1-x)^n}{n}\), then we know that \(\epsilon_n \le \left|\tfrac{(1-a)^n}{n}\right|\); further we know that \(a \in [\tfrac{1}{2}, 1)\) and thus \(\epsilon_n < \tfrac{1}{2^nn}\).\\

Using this approximation we can see that if we wish to guarantee \(d\) decimal places of accuracy then it suffices to find \(n \in \N\) such that \(\tfrac{1}{2^nn} < 10^{-d} \implies 2^nn > 10^d\). As \(n \in \N\) then \(2^n < 2^nn\) and so we merely need to find \(n \in \N\) such that \(2^n > 10^d\) to guarantee \(d\) decimal places of accuracy. Some example values are included in the table below:\\

%TBL%
\begin{center}
\begin{tabular}{|l|l|}
\hline
\(d \in \N\) 
	& \(\textrm{arg}\min\left\{n \in \N : 2^n > 10^d\right\}\)\\\hline
1 & 4\\\hline
10 & 34\\\hline
100 & 333 \\\hline
1000 & 3322\\\hline
\end{tabular}
\end{center}

As we now have Taylor methods for approximating both \(e^x\) and \(\ln(x)\), then we can use the two to derive a Taylor method of calculating \(x^y\) and \(\log_x(y)\). To start we will consider \(x^y = e^{y\ln(x)}\) and \(x = a\cdot2^b\), giving the solution as \(x^y = e^{y(b\ln(2) + \ln(a))}\). Similarly we note that \(\log_x(y) = \frac{\ln(y)}{\ln(x)}\), and if we consider that \(x = a\cdot2^b\) and \(y = c\cdot2^d\), then we see that \(\log_x(y) = \frac{d\ln(2) + \ln(c)}{b\ln(2) + \ln(a)}\). Below are the Taylor methods for approximating these functions:

%PCD%
\begin{lstlisting}[caption={Taylor Method for calculating \(x^y\) and \(\log_x(y)\)},label={PCD_"taylor pow/log"}]
  taylor_log($x \in \Rp, y \in \Rp, n \in \N$):
      $a :=$ taylor_nat_log($y, n$)
      $b :=$ taylor_nat_log($x, n$)
      return $\tfrac{a}{b}$
  
  taylor_pow($x \in \Rp, y\in \R, n \in \N$):
      $a :=$ taylor_nat_log($x, n$)
      $a \mapsto y \cdot a$
      return taylor_exp($a, n$)
\end{lstlisting}

To test the convergence of the Taylor methods above we are going to test calculations of \(7.3^{4.8}\), \(7.3^{-4.8}\), \(0.21^{4.8}\), \(7.3^{0.21}\), \(\log_{7.3}(4.8)\), \(\log_{0.21}(4.8)\) and \(\log_{7.3}(0.21)\). These values are calculated for several different values of \(n\) with the bold digits representing the correct values in the tables below:

%TBL%
{\fontfamily{pcr}\selectfont
\begin{center}
\begin{tabular}{|l|l|l|l|l|}
\hline
\(n\)&\(7.3^{4.8}\)&\(7.3^{-4.8}\)&\(0.21^{4.8}\)&\(7.3^{0.21}\)\\\hline
1 & \textbf{1}.0000000000& \textbf{1.000}0000000& \textbf{1.00}00000000& \textbf{1.}0000000000\\\hline
2 & \textbf{10}.561319400& \textbf{-}8.561319400& \textbf{-}6.422212933& \textbf{1.}4183077237\\\hline
3 & \textbf{5}6.076838311& \textbf{36}.990949511& \textbf{2}1.518877680& \textbf{1.5}046585363\\\hline
4 & \textbf{2}00.85920964& \textbf{-1}07.8118783& \textbf{-}48.47602784& \textbf{1.51}67171202\\\hline
5 & \textbf{54}6.24576990& \textbf{2}37.58122696& \textbf{8}2.710783892& \textbf{1.51}79778747\\\hline
6 & \textbf{1}205.3726532& \textbf{-}421.5471761& \textbf{-}113.8668463& \textbf{1.5180}831956\\\hline
7 & \textbf{2}253.5829747& \textbf{6}26.66342673& \textbf{1}31.57101558& \textbf{1.518090}5223\\\hline
8 & \textbf{3}682.4131809& \textbf{-8}02.1668232& \textbf{-1}31.0877429& \textbf{1.5180909}589\\\hline
9 & \textbf{53}86.6141612& \textbf{90}2.03416303& \textbf{1}14.86315726& \textbf{1.51809098}16\\\hline
10 & \textbf{7}193.4074522& \textbf{-9}04.7591286& \textbf{-}89.85299062& \textbf{1.5180909827}\\\hline
\cdots&\cdots&\cdots&\cdots&\cdots\\\hline
20 & \textbf{139}01.238666& \textbf{-11}.00988984& \textbf{-0.}092958315& \textbf{1.5180909827}\\\hline
\cdots&\cdots&\cdots&\cdots&\cdots\\\hline
40 & \textbf{13929.955484}& \textbf{0.00007178}62& \textbf{0.0005580236}& \textbf{1.5180909827}\\\hline
\cdots&\cdots&\cdots&\cdots&\cdots\\\hline
80 & \textbf{13929.955484}& \textbf{0.0000717877}& \textbf{0.0005580236}& \textbf{1.5180909827}\\\hline
\end{tabular}
\end{center}}

As we can see in the table the \textrm{taylor\_pow} does not converge perfectly, and may even diverge from the correct value for small values of \(n\); however we see that the methods do converge for large values of \(n\). This behaviour is due to the values being outside the restrictions used in the analysis of the functions.

%TBL%
{\fontfamily{pcr}\selectfont
\begin{center}
\begin{tabular}{|l|l|l|l|}
\hline
\(n\)&\(\log_{7.3}(4.8)\)&\(\log_{0.21}(4.8)\)&\(\log_{7.3}(0.21)\)\\\hline
1 & \textbf{0.}8431178860& \textbf{-1.}086107266& \textbf{-0.}776274970\\\hline
2 & \textbf{0.}8431178860& \textbf{-1.}086107266& \textbf{-0.}776274970\\\hline
3 & \textbf{0.}8045021618& \textbf{-1.}025878600& \textbf{-0.7}84207957\\\hline
4 & \textbf{0.7}938608884& \textbf{-1.}011309817& \textbf{-0.7}84982875\\\hline
5 & \textbf{0.7}906472231& \textbf{-1.0}07102721& \textbf{-0.785}071082\\\hline
6 & \textbf{0.789}6173849& \textbf{-1.00}5776909& \textbf{-0.7850}82036\\\hline
7 & \textbf{0.789}2739993& \textbf{-1.00}5337682& \textbf{-0.78508}3473\\\hline
8 & \textbf{0.789}1562591& \textbf{-1.005}187460& \textbf{-0.785083}668\\\hline
9 & \textbf{0.789}1150494& \textbf{-1.005}134935& \textbf{-0.7850836}95\\\hline
10& \textbf{0.789}1003970& \textbf{-1.005}116266& \textbf{-0.78508369}9\\\hline
\cdots & \cdots & \cdots & \cdots \\\hline
50& \textbf{0.7890920869}& \textbf{-1.005105681}& \textbf{-0.785083699}\\\hline
\end{tabular}
\end{center}}

This shows that \textrm{taylor\_log} converges better than \textrm{taylor\_exp}, however part of this is due to the values tested having magnitudes close to \(1\). Answers with a larger or smaller magnitudes tend to converge slower, which can be seen in the table for \textrm{taylor\_exp}. The value that had best convergence in the \textrm{taylor\_exp} table had an answer of about \(1.5\) and all other values tested had answers that were several orders of magnitude different from \(1\).

%SUB%
\subsection{Hyperbolic Series Method}

There are more efficient series which can be used to find \(\ln\), which converge quicker than the Taylor approximation. One such method is to consider the Hyperbolic Trigonometric function \(\tanh\). We start by considering the definition that \(\tanh(x) := \frace^x - e^{-x}}{e^x + e^{-x}}\), and then find a formula for \(\tanh^{-1}(x)\):

\begin{align*}
z = \frac{e^x - e^{-x}}{e^x + e^{-x}} 
	&\implies z = \frac{e^{2x} - 1}{e^{2x} + 1}\\
	&\implies ze^{2x} + z = e^{2x} - 1\\
	&\implies e^{2x}(1 - z)=1+z\\
	&\implies e^{2x} = \frac{1+z}{1-z}\\
	&\implies e^{x} = \left(\frac{1+z}{1-z}\right)^{\frac{1}{2}}\\
	&\implies x = \frac{1}{2}\ln\left(\frac{1+z}{1-z}\right)
\end{align*}

Using this we can see that \(2\tanh^{-1}\left(\frac{x-1}{x+1}\right) = \ln(x)\), and we can use the Taylor Expansion of \(\tanh^{-1}\) to approximate \(\ln\).\\

Now to attain the Taylor series for \(\tanh^{-1}(x)\) we can use the same method as when we calculated the Taylor series for \(\ln\). The exact calculations are omitted, but the end result is that we get that:

\[\tanh^{-1}(x) = \sum_{n=0}^\infty\frac{x^{2n+1}}{2n+1}\quad \forall\: x \in \Rp\]

And thus by using this series we get the result that:

\[\ln(x) = 2\sum_{n=0}^\infty\frac{1}{2n+1}\left(\frac{x-1}{x+1}\right)^{2n+1}\quad \forall\: x \in \Rp\]

The implementation of this is similar to previous implementations of series approximations of a function and is detailed below:

%PCD%
\begin{lstlisting}[caption={Hyperbolic seies method for \(\ln\)},label={PCD_"hyperbolic ln"}]
  hyperbolic_nat_log($x \in \Rp, n \in \Zpz$):
      $a := \tfrac{x - 1}{x+1}$
      $b := y^2$
      $c := a$
      $k := 0$
      while $k \le n$:
          $a \mapsto a\cdot b$
          $c \mapsto c + \tfrac{a}{2k+1}$
          $k \mapsto k + 1$
      return $2\cdot c$
\end{lstlisting}

Using this we see that if we have \(\epsilon_n := |\ln(x) - \textrm{hyperbolic\_nat\_log(}x, n\textrm{)}|\), then we know that \(\epsilon_n \le \frac{1}{2n + 3}\left|\frac{x-1}{x+1}\right|^{2n+3}\). If we consider restricting our calculations to \(x \in [\tfrac{1}{2}, 1)\) by using the  same calculations as shown for algorithm \ref{PCD_"taylor log"}, then we can see that \(|x - 1| \le \frac{1}{2}\) and \(|x+1| \ge \frac{3}{2}\); therefore \(\epsilon_n \le \frac{1}{3^{2n+3}(2n+3)}\).\\

By considering the final simplification that \(\epsilon_n < \frac{1}{3^{2n+3}}\), then if we wish to have \(\epsilon_n < \tau \in \Rp\) it suffices to find \(n \in \N\) such that \(\frac{1}{3^{2n+3}} < \tau\). In particular we consider when \(\tau = 10^{-d}\) which will guarantee \(d\) decimal places of accuracy, below is a table showing the smallest \(n \in \N\) that guarantees \(d\) decimal places of accuracy:

%TBL%
\begin{center}
\begin{tabular}{|l|l|}
\hline
\(d \in \N\) 
	& \(\textrm{arg}\min\left\{n \in \N : 3^{2n+3} > 10^d\right\}\)\\\hline
1 & 1\\\hline
10 & 8\\\hline
100 & 104\\\hline
1000 & 1047\\\hline
\end{tabular}
\end{center}

As can be seen in the table, fewer iterations are needed to approximate \(\ln(x)\) to the same degree of accuracy using hyperbolic series as when using the Taylor series. Further, the calculations performed each iteration are very similar in complexity, both being \(\bigO(1)\), and thus we can expect that algorithm \ref{PCD_"hyperbolic ln"} will execute faster than \ref{PCD_"taylor log"}.

%SUB%
\subsection{Continued fractions}

\theoremstyle{plain}
\newtheorem{equiv cont frac}{Proposition}[subsection]
\newtheorem{odd even conv}[equiv cont frac]{Proposition}

Another method for evaluating \(e^x\) is the use of continued fractions, which are a way of approximating real functions by a rational number\cite{BOK_ContFrac} with a recursive structure. Such fractions have been studied for many years and can be used to rationally approximate functions. Some examples of continued fractions for real numbers are\cite[][266]{BOK_ContFrac}:

\begin{displaymath}
\begin{array}{c@{\hspace{4em}}c}
e = 2 + \cfrac{1}{1 +
	    \cfrac{1}{2 +
		\cfrac{1}{1 +
		\cfrac{1}{1 +
		\cfrac{1}{4 + \ddots} } } } }&
\pi = 3 + \cfrac{1}{7 + 
		  \cfrac{1}{15 +
		  \cfrac{1}{1 + 
		  \cfrac{1}{292 +
		  \cfrac{1}{1 + \ddots} } } } }
\end{array}
\end{displaymath}

In general a continued fraction for a number \(x \in \R\) has the form:

%EQN%
\begin{equation}
\label{EQN_"general cont frac"}
	b_0 + \cfrac{a_1}{b_1 + 
		  \cfrac{a_2}{b_2 +
		  \cfrac{a_3}{b_3 + \ddots} } }
\end{equation}

As the writing of continued fractions in the above manner takes up a lot of room and has a degree of ambiguity we will use the following notation:

%EQN%
\begin{equation}
\label{EQN_"cont frac notation"}
	\mathbf{K}_{n=1}^\infty \frac{a_n}{b_n} := \cfrac{a_1}{b_1 +
		  					   				   \cfrac{a_2}{b_2 +
							   				   \cfrac{a_3}{b_3 + 
											   \cfrac{a_4}{b_4 + \ddots}}}}
\end{equation}

Therefore we can re-write Equation \ref{EQN_"general cont frac"} as \(b_0 + \mathbf{K}_{n=1}^\infty \frac{a_n}{b_n}\).\\

One of the most useful formulas regarding continued fractions was formulated by Leonhard Euler\cite[][Ch.~18]{BOK_EulContFrac}, and deals with the sum \(a_0 + a_0a_1 + a_0a_1a_2 + \cdots + (a_0 \cdots a_n) = \sum_{i=0}^n(\prod_{j=0}^ia_j)\). The formula derived by Euler is known as Euler's Continued Fraction Formula and is as follows:

%EQN%
\begin{equation}
\label{EQN_"euler cont frac"}
\begin{align*}
	\sum_{i=0}^n\left(\prod_{j=0}^ia_j\right) 
		= \mathbf{K}_{i=0}^n\frac{\alpha_i}{\beta_i} \textrm{ where } 
			&\alpha_i := \left\{
				\begin{array}{lcl}
					a_0&:&i=0\\
					-a_i&:&i \in [1,n]\cap\Z
				\end{array}\right.\\
			&\beta_i := \left\{
				\begin{array}{lcl}
					1&:&i=0\\
					1+a_i&:&i\in[1,n]\cap\Z
				\end{array}
\end{align*}
\end{equation}

Many Taylor series have a structure that is compatible with equation \ref{EQN_"euler cont frac"} and so can be approximated by a continued fraction in this way. In particular we are looking at \(e^x = \sum_{k=0}^n \frac{x^n}{n!}\) where we note that we can re-write the series as \(e^x = 1 + \sum_{i=1}^n (\prod_{j=1}^i \frac{x}{j})\) and therefore by using Euler's Continued Fraction Formula we see that:

\begin{align}
e^x &= 1 + \cfrac{x}{1 -
		   \cfrac{\frac{1}{2}x}{1 + \frac{1}{2}x - 
		   \cfrac{\frac{1}{3}x}{\ddots - 
		   \cfrac{\frac{1}{n-1}x}{1 + \frac{1}{n-1}x - 
		   \frac{\frac{1}{n}x}{1 + \frac{1}{n}x} } } } }
		   \label{EQN_"exp cont frac 1"}\\
	&= 1 + \mathbf{K}_{i=1}^n\frac{\alpha_i}{\beta_i} &\textrm{ where }
		&\alpha_i := \left\{\begin{array}{lcl}
			x&:&i=1\\
			-\frac{1}{i}x &:& i \in [2, n]\cap\Z
			\end{array}\right.\nonumber\\
	&	&&\beta_i := \left\{\begin{array}{lcl}
			1 &:& i=1\\
			1 + \frac{1}{i}x &:& i \in [2,n]\cap\Z
			\end{array}\right.\nonumber
\end{align}

We want to simplify the above equation to remove the fractional coefficients. If we consider multiplying \(\alpha_1\) by some constant \(c_1\), then to have an equivalent fraction we would have to multiply it's denominator by \(c_1\); in practice this means multiplying \(\beta_1\) and \(\alpha_2\) by \(c_1\). We can suppose that we could continue in a similar manner for constants \(c_2, c_3, \ldots\) multiplying \(\alpha_2, \alpha_3, \ldots\).

%THM%
\begin{equiv cont frac}
\label{THM_"equiv cont frac"}
If we have a continued fraction \(b_0 + \mathbf{K}_{i=1}^n\frac{a_i}{b_i}\) and constants \((c_i : i \in [1, n] \cap \Z)\), then:
\[b_0 + \mathbf{K}_{i=1}^n\frac{a_i}{b_i} = b_0 + \mathbf{K}_{i=1}^n\frac{c_{i-1}c_ia_i}{c_ib_i}\]
where \(c_0 := 1\), for any \(n \in \N\).
\end{equiv cont frac}
\begin{proof}
We will proceed by induction on \(n \in \N\).\\
\begin{description}
\item[\textrm{H\((n)\):}] \(b_0 + \mathbf{K}_{i=1}^n\frac{a_i}{b_i} = b_0 + \mathbf{K}_{i=1}^n\frac{c_{i-1}c_ia_i}{c_ib_i}\)
\item[\textrm{H\((1)\):}]
	\begin{align*}
		b_0 + \frac{c_0c_1a_1}{c_1b_1} 
			&= b_0 + \frac{c_1a_1}{c_1b_1} &\textrm{as \(c_0 = 1\)}\\
			&= b_0 + \frac{a_1}{b_1} &\textrm{as required}
	\end{align*}
\item[\textrm{H\((n)\) \(\implies\) H\((n+1)\):}]
	\begin{align*}
		b_0 + \mathbf{K}_{i=1}^{n+1}\frac{c_{i-1}c_ia_i}{c_ib_i}
			&= b_0 + \left(\mathbf{K}_{i=1}^n
				\frac{c_{i-1}c_ia_i}{c_ib_i}\right)_+
				\frac{c_nc_{n+1}a_{n+1}}{c_{n+1}b_{n+1}}\\
			&= b_0 + \left(\mathbf{K}_{i=1}^n
				\frac{c_{i-1}c_ia_i}{c_ib_i}\right)_+
				c_n\frac{a_{n+1}}{b_{n+1}}\\
	\end{align*}

Now let us define \(b'_i\) as:
\begin{displaymath}
\left\{
	\begin{array}{lcl}
		b_n + \frac{a_{n+1}}{b_{n+1}} &:& i = n\\
		b_i &:& i \neq n
	\end{array}
\right.
\end{displaymath}

Therefore we can continue and see that:
	\begin{align*}
		b_0 + \mathbf{K}_{i=1}^{n+1}\frac{c_{i-1}c_ia_i}{c_ib_i}
			&= b_0 + \mathbf{K}_{i=1}^n\frac{c_{i-1}c_ia_i}{c_ib'_i}\\
			&= b_0 + \mathbf{K}_{i=1}^n \frac{a_i}{b'_i}
				&\textrm{by H\((n)\)}\\
			&= b_0 + \mathbf{K}_{i=1}^{n+1}\frac{a_i}{b_i}
	\end{align*}
\end{description}
\end{proof}

Using this proposition we can see that if we have the sequence \((c_1, c_2, \ldots, c_n)\) defined as \(c_i = i\) and apply it to our sequence for \(e^x\) we get that:

\begin{align}
e^x &= 1 + \cfrac{x}{1 -
		   \cfrac{x}{2 + x - 
		   \cfrac{2x}{\ddots - 
		   \cfrac{(n-1)x}{n-1 + x - 
		   \frac{x}{n + x} } } } } \label{EQN_"exp cont frac 1 simp"}\\
	&= 1 + \mathbf{K}_{i=1}^n\frac{\alpha_i}{\beta_i} &\textrm{ where }
		&\alpha_i := \left\{\begin{array}{lcl}
			x&:&i=1\\
			-(i-1)x &:& i \in [2, n]\cap\Z
			\end{array}\right.\nonumber\\
	&	&&\beta_i := \left\{\begin{array}{lcl}
			1 &:& i=1\\
			x + i &:& i \in [2,n]\cap\Z
			\end{array}\right.\nonumber
\end{align}

This is a much simpler continued fraction, but evaluating it would still be computationally expensive due to the repeated division operations; to get around this we can consider what are known as the convergents of a continued fraction. It is obvious that if we use a continued fraction to approximate some value \(z\) by the continued fraction \(b_0 + \mathbf{K}_{i=1}^n \frac{a_i}{b_i}\), then there are some \(A_n, B_n \in \N\) such that \(z = \frac{A_n}{B_n}\).\\

To start we will define \(A_{-1} := 1\) and \(B_{-1} := 0\), and consider the case when \(n = 0\); for this case \(z = b_0\), which means that \(A_0 = b_0\) and \(B_0 = 1\). For the case when \(n = 1\) we have \(z = b_0 + \frac{a_1}{b_1}\), which when rearranged is \(z = \frac{b_0b_1 + a_1}{b_1}\). This means that \(A_1 = b_0b_1 + a_1 = b_1A_0 + a_1A_{-1}\) and \(B_1 = b_1 = b_1B_0 + a_1B_{-1}\), and for the case when \(n = 2\) we get the similar result that \(A_2 = b_0b_1b_2 + a_2b_0 + a_1b_2 = b_2A_1 + a_2A_0\) and \(B_2 = b_1b_2 + a_2 = b_2 B_1 + a_2B_0\).\\

It is actually true that this relationship continues for all \(n \in \N\), and thus we get what are known as the Fundamental Recurrence Formulas for continued fractions:

\begin{displaymath}
\begin{array}{rcl@{\hspace{2em}}rclc}
	A_{-1} &=& 1 &B_{-1} &=& 0\\
	A_0 &=& b_0 &B_0 &=& 1\\
	A_{n+1} &=& b_{n+1}A_n + a_{n+1}A_{n-1}
	&B_{n+1}&=& b_{n+1}B_n + a_{n+1}B_{n-1} & \forall\ n \in \Zpz
\end{array}
\end{displaymath}

Using this and our simplified continued fraction for \(e^x\) we can use the following method to approximate \(e^x\) by using a continued fraction up to \(a_n,b_n\) where \(n \ge 2\):
 
%PCD%
\begin{lstlisting}[caption={Continued fraction for \(e^x\)},label={PCD_"cont exp v1"}]
  cont_frac_exp($x \in \R, n \in \N$):
      $A_1 := x + 1$
      $B_1 := 1$
      $A_2 := x^2 + 2x + 2$
      $B_2 := 2$
      $a := -x$
      $b := 2 + x$
      $k := 2$
      while $k < n$:
          $a \mapsto a - x$
          $b \mapsto b + 1$
          $A_{k+1} := bA_k + aA_{k-1}$
          $B_{k+1} := bB_k + ab_{k-1}$
          $k \mapsto k + 1$
      return $\tfrac{A_k}{B_K}$
\end{lstlisting}

One observation of the above algorithm, when implemented on a computer, is that if we pre-generate \(b_i\) and \(a_i\) for \(i \in [2, n]\cap \Z\) then the calculations of \(A_i\) and \(B_i\) are independent. This means that, if supported by the computer, both \(A_i\) and \(B_i\) could be computed in parallel. This may allow an implementation of the algorithm to be more efficient than one that computes the function in sequence.\\

While continued fractions are useful for approximating functions, it is difficult to evaluate the error of their output analytically. One result is that if \(a_n = 1\ \forall\: n \in \N\) when approximating some value \(z\), then \(|z - \tfrac{A_n}{B_n}| \le \frac{1}{|B_{n+1}B_n|}\)\cite{BOK_ContFrac}. If we transform the continued fraction of \(e^x\) into this form by using Proposition \ref{THM_"equiv cont frac"}, then we get that:

\begin{displaymath}
	e^x = 1 + \cfrac{1}{(-\tfrac{1}{x}) + 
			  \cfrac{1}{(-\frac{2+x}{2x}) +
			  \cfrac{1}{(-\frac{3+x}{3x}) + \ddots } } }
\end{displaymath}

By using a computer to implement the calculations for a test value of \(x = 1\), we see that \(\frac{1}{B_5B_6} = 0.009131261889664\ldots\) and \(\frac{1}{B_{10}B_{11}} = 0.000041307209877\ldots\); thus we can guarantee two decimal place of accuracy with \textrm{cont\_frac\_exp(\(1, 5\))} and 4 with \textrm{cont\_frac\_exp(\(1, 10\))}. However if we instead have \(x = 14\) then \(\frac{1}{B_{10}B_{11}} = 0.314711263190806\ldots\) and convergence is similarly poor for negative values.\\
 
Further computations show that convergence of \(x \in (0, 1)\) is better than the convergence when \(x=1\), and thus we can use the identities and conversions to ensure good convergence. In particular if \(x \in \Rn\) then we can calculate the reciprocal of \textrm{cont\_frac\_exp(\(-x, n\))} and if \(x \in (1,\infty)\) we use the identity that \(x = a\cdot2^b\); with this we see that \(e^x = (e^a)^{2^b}\) and \(2^b \in \Zp\).\\

As \(a \in (0,1)\) and \(2^b \in \Zp\) then we can calculate \(z = 2^a\) using algorithm \ref{PCD_"cont exp v1"}. Then we can calculate \(z^{2^b}\) using algorithm \ref{PCD_"exp by square"}, to find our approximation of \(e^x\). Performing the calculation in this way allows us to use the our continued fraction method to guarantee fast convergence, and the \(\bigO(1)\) integer exponential algorithm to guarantee the correct approximation without increasing the algorithmic complexity of the calculations by more than a constant factor.\\

With this restriction in place we know that algorithm \ref{PCD_"cont exp v1"} converges at least as quickly as it does for \(x = 1\), and thus we can use its convergence to guarantee the convergence of our method. Below is a table that shows the minimum \(n\) needed to achieve the associated \(d\) decimal places of accuracy:

%TBL%
\begin{center}
\begin{tabular}{|l|l|}
\hline
\(d\) & Minimum \(n\) to guarantee \(d\) decimal places of accuracy\\\hline
1 & 2 \\\hline
10 & 22 \\\hline
100 & 235 \\\hline
1000 & 2386 \\\hline
\end{tabular}
\end{center}

An alternative continued fraction for \(e^x\) that arises from the generalized hyper geometric series\cite{BOK_JoTh80} is:

\begin{align}
e^x &= \cfrac{1}{1 -
	   \cfrac{x}{1 +
	   \cfrac{x}{2 -
	   \cfrac{x}{3 +
	   \cfrac{2x}{4 -
	   \cfrac{2x}{5 +
	   \cfrac{3x}{6 - \ddots} } } } } } } \label{EQN_"exp cont frac 2"}\\
	&= \mathbf{K}_{i=1}^n\frac{\alpha_i}{\beta_i} &\textrm{ where } 
		&\alpha_i := \left\{\begin{array}{lcl}
			1 &:& i = 1\\
			-x &:& i = 2\\
			(-1)^{i-1}\lfloor\frac{i-1}{2}\rfloor x&:&i\in[3,\infty)\cap\Z
			\end{array}\right.\nonumber\\
	&  &&\beta_i := \left\{\begin{array}{lcl}
			1 &:& i=1\\
			i-1 &:& i \in [2, \infty) \cap\Z
			\end{array}\right.\nonumber
\end{align}

Due to the \((-1)^{i-1}\lfloor\frac{i-1}{2}\rfloor\) factor in the definition of \(\alpha_i\) it is more efficient to perform two updates each step rather than one. Below is the implementation of this method:

%PCD%
\begin{lstlisting}[caption={Continued fraction for \(e^x\) version 2},label={PCD_"cont exp v2"}]
  cont_frac_exp_v2($x \in \R, n \in \N$):
      $A_1 := 1$
      $B_1 := 1$
      $A_2 := 1$
      $B_2 := 1 - x$
      $a := 1$
      $b := 1$
      $k := 2$
      while $k < n$:
          $a \mapsto xa$
          $b \mapsto b + 1$
          $A_{k+1} := bA_k + aA_{k-1}$
          $B_{k+1} := bB_k + ab_{k-1}$
          $k \mapsto k + 1$
          $b \mapsto b + 1$
          $A_{k+1} := bA_k - aA_{k-1}$
          $B_{k+1} := bB_k - ab_{k-1}$
          $k \mapsto k + 1$
      return $\tfrac{A_k}{B_K}$
\end{lstlisting}

The fraction needed to analyse this method is again found by using proposition \ref{THM_"equiv cont frac"}, and is:

\begin{displaymath}
	\cfrac{1}{1 +
	\cfrac{1}{-\frac{1}{x} +
	\cfrac{1}{-2 + 
	\cfrac{1}{\frac{3}{x} +
	\cfrac{1}{2 + 
	\cfrac{1}{-\frac{5}{x} +
	\cfrac{1}{-2 + \ddots} } } } } } }
\end{displaymath}

By implementing this we get similar results to above, particularly there is rapid convergence for \(x \in (0, 1)\). Further the convergence of values in \(x \in (0,1)\) is more rapid than \(x = 1\) and so we can use the convergence of \(x = 1\) as an upper bound of our method. Below is the table showing the smallest \(n \in \N\) needed to ensure \(d\) decimal places of accuracy for some \(d \in \N\):

\begin{center}
\begin{tabular}{|l|l|}
\hline
\(d\) & Minimum \(n\) to guarantee \(d\) decimal places of accuracy\\\hline
1 & 4 \\\hline
10 & 12 \\\hline
100 & 61 \\\hline
1000 & 405 \\\hline
\end{tabular}
\end{center}

As can be seen the convergence of \ref{EQN_"exp cont frac 2"} appears to be significantly faster than that of \ref{EQN_"exp cont frac 1 simp"} and one might be satisfied by this, however an even better solution exists.\\

As the fraction \ref{EQN_"exp cont frac 2"} can be shown to converge for all values of \(x\) to \(e^x\) then we can consider the even and odd convergents. The even convergents of the sequence are \(\frac{A_0}{B_0}, \frac{A_2}{B_2}, \frac{A_4}{B_4}, \ldots\), while the odd convergents are \(\frac{A_1}{B_1}, \frac{A_3}{B_3}, \frac{A_5}{B_5}, \ldots\). As \(\lim_{n\to\infty} \frac{A_n}{B_n}=e^x\), then \(\lim_{n\to\infty}\frac{A_{2n}}{B_{2n}} = \lim_{n\to\infty}\frac{A_{2n+1}}{B_{2n+1}} = e^x\); the following proposition\cite[][86]{BOK_ContFrac} gives an explicit form for the odd and even convergents.

%THM%
\begin{odd even conv}
\label{THM_"odd even conv"}
If \(z = \mathbf{K}_{i=1}^\infty \frac{a_i}{1}\), then the limit of the odd convergent of \(z\) is:
\begin{displaymath}
	x_{\textrm{odd}}=a_1 - \cfrac{a_1a_2}{1+a_2+a_3 -
						   \cfrac{a_3a_4}{1+a_4+a_5 -
						   \cfrac{a_5a_6}{1+a_6+a_7 - \ddots} } }
\end{displaymath}
while the limit of the even convergent is:
\begin{displaymath}
	x_{\textrm{even}}=\cfrac{a_1}{1+a_2 -
					  \cfrac{a_2a_3}{1+a_3+a_4 -
					  \cfrac{a_4a_5}{1+a_5+a_6 - \ddots} } }
\end{displaymath}
\end{odd even conv}
\begin{proof}
Omitted
\end{proof}

If we apply proposition \ref{THM_"equiv cont frac"} to \ref{EQN_"exp cont frac 2"}, to achieve the form \(\mathbf{K}_{i=1}^\infty \frac{a_i}{1}\) then we end up with the following fraction:

\begin{equation}
	e^x = \cfrac{1}{1 +
		  \cfrac{-x}{1 +
		  \cfrac{\frac{1}{2}x}{1 +
		  \cfrac{-\frac{1}{6}x}{1 +
		  \cfrac{\frac{1}{6}x}{1 +
		  \cfrac{-\frac{1}{10}x}{1 +
		  \cfrac{\frac{1}{10}x}{1 + 
		  \cfrac{-\frac{1}{14}x}{1 + \ddots} } } } } } } }
\end{equation}

Now if we apply proposition \ref{THM_"odd even conv"} to the above fraction we see that:

\begin{equation}
	e^x = 1 + \cfrac{x}{1 - x + \frac{1}{2}x +
			  \cfrac{\frac{1}{12}x^2}{1 - \frac{1}{6}x + \frac{1}{6}x +
			  \cfrac{\frac{1}{60}x^2}{1 -\frac{1}{10}x +\frac{1}{10}x +
			  \cfrac{\frac{1}{140}x^2}{\ddots} } } }
\end{equation}

Finally by simplifying and applying proposition \ref{THM_"equiv cont frac"} one more time we reach the following continued fraction for \(e^x\):

\begin{align}
e^x &= 1 + \cfrac{2x}{2-x+
		   \cfrac{x^2}{6 +
		   \cfrac{x^2}{10+
		   \cfrac{x^2}{14+\ddots} } } } \label{EQN_"exp cont frac 3"}\\
	&= 1 + \mathbf{K}_{i=1}^\infty \frac{\alpha_i}{\beta_i}
		&\textrm{ where }
		&\alpha_i := \left\{\begin{array}{lcl}
			2x &:& i = 1\\
			x^2 &:& i \in [2, \infty)\cap\Z
			\end{array}\right.\nonumber\\
	&&	&\beta_i := \left\{\begin{array}{lcl}
			2 - x &:& i = 1\\
			4i - 2 &:& i \in [2, \infty)\cap \Z
			\end{array}\right.\nonumber
\end{align}

If we implement this method by using the Fundamental Recurrence Formula then we get the following:

%PCD%
\begin{lstlisting}[caption={Continued fraction for \(e^x\) version 3},label={PCD_"cont exp v3"}]
  cont_frac_exp_v3($x \in \R, n \in \N$):
      $A_0 := 1$
      $B_0 := 1$
      $A_1 := 2 + x$
      $B_1 := 2 - x$
      $a := x^2$
      $b := 2$
      $k := 1$
      while $1 < n$:
          $b \mapsto b + 4$
          $A_{k+1} := bA_k + aA_{k-1}$
          $B_{k+1} := bB_k + ab_{k-1}$
          $k \mapsto k + 1$
      return $\tfrac{A_k}{B_K}$
\end{lstlisting}

As with the previous two continued fraction methods of approximating \(e^x\) we can apply proposition \ref{THM_"equiv cont frac"} to \ref{EQN_"exp cont frac 3"} to find the following equivalent continued fraction:

\begin{displaymath}
	1 + \cfrac{1}{\frac{1}{x} - \frac{1}{2} +
		\cfrac{1}{\frac{12}{x} + 
		\cfrac{1}{\frac{5}{x} +
		\cfrac{1}{\frac{28}{x} + \ddots } } } }
\end{displaymath}

Again a computer was used to evaluate \(B_k\) of the above fraction, which gave the expected results of quick convergence for \(x \in (0,1)\) and more rapid convergence for \(x \in (0,1)\) than \(x = 1\). Using this the table below was generated to show the minimum \(n \in \N\) that guarantees \(d\) digits of accuracy:

%TBL%
\begin{center}
\begin{tabular}{|l|l|}
\hline
\(d\) & Minimum \(n\) to guarantee \(d\) decimal places of accuracy\\\hline
1 & 2 \\\hline
10 & 6 \\\hline
100 & 30 \\\hline
1000 & 202 \\\hline
\end{tabular}
\end{center}

This has the fastest theoretical convergence of the three methods, and thus is expected to perform the best.

\subsection{Comparison of Methods}

We have introduced several methods for calculating both logarithms and exponentials in this chapter, and considered their theoretical convergence; we now look at a direct comparison of the methods as implemented in C.\\

The first consideration is which values to use while comparing methods. While all the methods converge for all values, or can be made to by using transformations of the inputs and outputs, most methods converge best for small values. Therefore values being tested will typically be in the range of \([0.5, 1)\).\\

The first methods to be compared here are the versions of the continued fraction method discussed previously. Below we have the outputs of different versions of the program for different values of \(n\), with the bold digits being the correctly approximated digits.


%TBL%
{\fontfamily{pcr}\selectfont
\begin{center}
\begin{tabular}{|l|l|l|l|}
\hline
\(n\) & \codeinline{cont\_frac\_exp\_v1} 
	&\codeinline{cont\_frac\_exp\_v2} 
	&\codeinline{cont\_frac\_exp\_v3}\\\hline
1 & \textbf{1}.9449999999999& 3.3333333333333& \textbf{2.0}769230769230\\\hline
2 & \textbf{2.0}021666666666& 3.3333333333333& \textbf{2.013}2689987937\\\hline
3 & \textbf{2.01}21708333333& \textbf{2.0}769230769230& \textbf{2.01375}43842848\\\hline
4 & \textbf{2.013}5714166666& \textbf{2.0}054200542005& \textbf{2.01375270}42253\\\hline
5 & \textbf{2.0137}348180555& \textbf{2.013}2689987937& \textbf{2.01375270747}44\\\hline
6 & \textbf{2.01375}11581944& \textbf{2.0137}906192914& \textbf{2.0137527074704}\\\hline
7 & \textbf{2.013752}5879565& \textbf{2.01375}43842848& \textbf{2.0137527074704}\\\hline
8 & \textbf{2.013752}6991603& \textbf{2.013752}6161232& \textbf{2.0137527074704}\\\hline
9 & \textbf{2.01375270}69445& \textbf{2.01375270}42253& \textbf{2.0137527074704}\\\hline
10 & \textbf{2.0137527074}399& \textbf{2.013752707}6056& \textbf{2.0137527074704}\\\hline
\end{tabular}
\end{center}}

As can be seen here the first two methods have similar convergence, however despite having a very poor theoretical convergence the first method converges better than the second version. Further, it is obvious that the third method has the fastest convergence, and thus should be the one to use in further comparisons.\\

Now we can compare the speed of the Taylor and continued fraction methods of calculating exponential values. For this we will use 1000 values in the range \([\frac{1}{2}, 1)\) and calculate each 100000 times to compare the speed of the method. We will be using values of \(n\) which guarantee 10 decimal places of accuracy, in particular \(n = 14\) for \codeinline{taylor\_exp} and \(n = 6\) for \codeinline{cont\_frac\_exp\_v3}.\\

The results of the tests run on my computer are included in the table below alongside those for the built in \codeinline{exp} function in \codeinline{math.h}:

%TBL%
{\fontfamily{pcr}\selectfont
\begin{center}
\begin{tabular}{|l|r|r|r|r|}
	\hline
	&\textsf{Total time:} & \textsf{Average time:} & \textsf{Minimum time:}
	&\textsf{Maximum time:}\\\hline
	\codeinline{taylor\_exp} & 12.430s & 0.012s & 0.012s & 0.019s\\\hline
	\codeinline{cont\_frac\_exp} & 4.741s & 0.004s & 0.004s & 0.007s\\\hline
	\codeinline{builtin\_exp} & 2.608s & 0.002s & 0.002s & 0.004s\\\hline
\end{tabular}
\end{center}}

This shows that the continued fractions method of evaluating exponential functions is almost three times as efficient as the standard Taylor series method. However both fall short of the built in method, despite the hyperbolic series method being a close second. This is likely due to a lower level implementation of the exponential function with various highly efficient programming practices implemented to optimize the code execution speed.\\

However one consideration is that if we instead test values in the range \([-5, 50]\), then while both \codeinline{taylor\_exp} and \codeinline{cont\_frac\_exp} have similar results the total time for \codeinline{cont\_frac\_exp} becomes {\fontfamily{pcr}\selectfont 9.347s}. This discrepancy is due to the additional calculations needed by \codeinline{cont\_frac\_exp} so that it evaluates only values in the range \([\frac{1}{2}, 1)\) for a quicker convergence.\\

The two methods discussed to evaluate \(\ln\) have their convergence for different values of \(n\) shown below, where they are approximating the value 0.7, with the bold digits representing the correctly approximated digits:

%TBL%
{\fontfamily{pcr}\selectfont
\begin{center}
\begin{tabular}{|l|l|l|}
\hline
\(n\) & \codeinline{taylor\_nat\_log} 
	&\codeinline{hyperbolic\_nat\_log}\\\hline
1 & \textbf{-0.3}00000000000& \textbf{-0.3566}04925707\\\hline
2 & \textbf{-0.3}00000000000& \textbf{-0.35667}3383305\\\hline
3 & \textbf{-0.3}45000000000& \textbf{-0.3566749}06089\\\hline
4 & \textbf{-0.35}4000000000& \textbf{-0.35667494}2973\\\hline
5 & \textbf{-0.356}025000000& \textbf{-0.3566749439}13\\\hline
6 & \textbf{-0.356}511000000& \textbf{-0.356674943938}\\\hline
7 & \textbf{-0.3566}32500000& \textbf{-0.356674943938}\\\hline
8 & \textbf{-0.3566}63742857& \textbf{-0.356674943938}\\\hline
9 & \textbf{-0.35667}1944107& \textbf{-0.356674943938}\\\hline
10 & \textbf{-0.356674}131107& \textbf{-0.356674943938}\\\hline
\end{tabular}
\end{center}}

We can see here that the hyperbolic method converges a lot faster than the Taylor method; one particular note is that the hyperbolic series accurately approximates the first 12 decimal places of \(\ln(0.7)\) accurately in just 6 iterations while the Taylor method only achieves 6 decimal places after 10 iterations.\\

To further test the two methods the table below shows the timings of calculating 1000 values in the range \([0.02, 50]\), each of which will be calculated to 10 decimal places 100000 times by each method. To guarantee 10 decimal places of accuracy with \codeinline{taylor\_log} we can use \(n = 34\) and \(n = 8\) for \codeinline{hyperbolic\_log}, below is the table that displays the results alongside the results for the built in \codeinline{log} function:

%TBL%
{\fontfamily{pcr}\selectfont
\begin{center}
\begin{tabular}{|l|r|r|r|r|}
	\hline
	&\textsf{Total time:} & \textsf{Average time:} & \textsf{Minimum time:}
	&\textsf{Maximum time:}\\\hline
	\codeinline{taylor\_log} & 22.247s & 0.022s & 0.021s & 0.026s\\\hline
	\codeinline{hyperbolic\_log} & 7.742s & 0.007s & 0.007s & 0.009s\\\hline
	\codeinline{builtin\_log} & 3.438s & 0.003s & 0.003s & 0.005s\\\hline
\end{tabular}
\end{center}}

Here we can see that the hyperbolic method of approximating \(\ln(x)\) is the better of the two methods discussed, around three times faster in execution. While the built in function is, as to be expected, the fastest executing function, \codeinline{hyperbolic\_log} is not far behind, implying that \codeinline{builtin\_log} may use an optimized version of \codeinline{hyperbolic\_log}.\\

Finally we get to comparing the general exponential, \(x^y\), and logarithm, \(log_x(y)\), functions. First we will test the convergence of the two variations of the \(log_x(y)\) function for different values of \(n\), using \((x,y) = (1.5, 15)\):

%TBL%
{\fontfamily{pcr}\selectfont
\begin{center}
\begin{tabular}{|l|l|l|}
\hline
\(n\) & \codeinline{taylor\_log} 
	&\codeinline{hyperbolic\_log}\\\hline
1 & \textbf{6.}1155499597569& \textbf{6.678}4758082659\\\hline
2 & \textbf{6.}1155499597569& \textbf{6.6788}677803210\\\hline
3 & \textbf{6.}5747854684469& \textbf{6.678873}4950163\\\hline
4 & \textbf{6.6}587865280763& \textbf{6.67887358}57263\\\hline
5 & \textbf{6.67}48050386470& \textbf{6.6788735872}409\\\hline
6 & \textbf{6.678}0194099644& \textbf{6.678873587267}1\\\hline
7 & \textbf{6.678}6895339230& \textbf{6.6788735872675}\\\hline
8 & \textbf{6.6788}331533503& \textbf{6.6788735872675}\\\hline
9 & \textbf{6.6788}645711387& \textbf{6.6788735872675}\\\hline
10 & \textbf{6.67887}15529216& \textbf{6.6788735872675}\\\hline
\end{tabular}
\end{center}}

This table clearly demonstrates that \codeinline{hyperbolic\_log} converges faster to the correct value than \codeinline{taylor\_log} as expected. The table below shows the convergence of \codeinline{taylor\_pow} and \codeinline{improved\_pow} for the input of \((x, y) = (1.115, 15)\):

%TBL%
{\fontfamily{pcr}\selectfont
\begin{center}
\begin{tabular}{|l|l|l|}
\hline
\(n\) & \codeinline{taylor\_pow} 
	&\codeinline{improved\_pow}\\\hline
1 & \textbf{1}.0000000000000& \textbf{5.}7430173458025\\\hline
2 & \textbf{4}.7597077083991& \textbf{5.11}63939774264\\\hline
3 & \textbf{5.}9158698156503& \textbf{5.118}5134154921\\\hline
4 & \textbf{5.}6528248111124& \textbf{5.1182}823832710\\\hline
5 & \textbf{5.}3825631874287& \textbf{5.11826}88605223\\\hline
6 & \textbf{5.}2383576844918& \textbf{5.118267}9322812\\\hline
7 & \textbf{5.1}703487304639& \textbf{5.11826786}73534\\\hline
8 & \textbf{5.1}401274582517& \textbf{5.118267862}7291\\\hline
9 & \textbf{5.1}272612067887& \textbf{5.1182678623}951\\\hline
10 & \textbf{5.1}219353833296& \textbf{5.1182678623}708\\\hline
\cdots&\cdots&\cdots\\\hline
20 & \textbf{5.11826}84126550& \textbf{5.1182678623688}\\\hline
\cdots&\cdots&\cdots\\\hline
30 & \textbf{5.118267862}4756& \textbf{5.1182678623688}\\\hline
\cdots&\cdots&\cdots\\\hline
40 & \textbf{5.118267862368}9& \textbf{5.1182678623688}\\\hline
\end{tabular}
\end{center}}

Both of these methods for the general exponential function have slow convergences, though the improved method does converge faster. This implies that there may be a more efficient method for approximating \(x^y\).\\

Next we will consider the actual speed of both the logarithm and exponential functions presented. One note is that C does not have a general \(\log\) function for an arbitrary base in \codeinline{math.h}, and so to implement this we will use \codeinline{log(y)/log(x)} for the built in logarithm. All of the functions will have values of \((x, y) \in (0,2]\times[0,3)\) and will use a value of \(n\) sufficient to calculate their answer accurate to 10 decimal places. Below are the calculations for 1000 random values calculated 10000 times for each method:

%TBL%
{\fontfamily{pcr}\selectfont
\begin{center}
\begin{tabular}{|l|r|r|r|r|}
	\hline
	&\textsf{Total time:} & \textsf{Average time:} & \textsf{Minimum time:}
	&\textsf{Maximum time:}\\\hline
\codeinline{taylor\_log} & 4.750s & 0.004s & 0.004s & 0.008s\\\hline
\codeinline{hyperbolic\_log} & 1.589s & 0.001s & 0.001s & 0.002s \\\hline
\codeinline{builtin\_log} & 0.690s & 0.000s & 0.000s & 0.000s \\\hline
\codeinline{taylor\_pow} & 6.956s & 0.006s & 0.006s & 0.007s \\\hline
\codeinline{improved\_pow} & 2.456s & 0.002s & 0.002s & 0.003s \\\hline
\codeinline{builtin\_pow} & 0.787s & 0.000s & 0.000s & 0.001s \\\hline
\end{tabular}
\end{center}}

Again we see that the methods that we showed to be theoretically superior, do in fact have superior execution speeds; however our methods still fail to match the execution speed of those built into C.\\

Overall we can conclude that if we were to want to implement calculating logarithms of a number then the hyperbolic series method is the best choice discussed, while the best choice for evaluating exponentials is the continued fraction method.\\

The special case of the exponentiation by squaring is worth considering in the case where a computer only supports integers. This is because the algorithm will still work for integer only values, while most of the others will not, and has a computational complexity of \(\bigO(1)\).



\section{Preliminary References}
http://math.exeter.edu/rparris/peanut/cordic.pdf \\
Inside your Calculator by Gerald R Rising \\
Wolfram Alpha \\

\appendix
%\Section{Code}
\lstset{basicstyle=\ttfamily,
		language=C,
		backgroundcolor=\color{cBg},
		basicstyle=\footnotesize,
		commentstyle=\color{cCm},
		frame=L,
		keywordstyle=\color{cKw},
		showstringspaces=false,
		stringstyle=\color{cSt},
		tabsize=2}
\renewcommand{\thelstlisting}{}
\renewcommand{\lstlistingname}{File}

In this appendix I list the entirety of the code which implements the algorithms discussed in the body of this document.

\subsection{General Code}
General Utilities File:
\lstinputlisting[caption={utilities.c}]{../Code/utilities.c}

Trigonometric Utilities File:
\lstinputlisting[caption={trig\_utilities.c}]{../Code/trig_utilities.c}

Header Files for Utilities:
\lstinputlisting[caption={utilities.h}]{../Code/utilities.h}
\lstinputlisting[caption={trig\_utilities.h}]{../Code/trig_utilities.h}

Makefile for the project:
\lstinputlisting[caption={makefile},language=make]{../Code/makefile}

\subsection{Trigonometric Code}
Code for Geometric Trigonometric Functions:
\lstinputlisting[caption={geometric\_trig.c}]{../Code/geometric_trig.c}

Header files for trigonometric functions:
\lstinputlisting[caption={geometric\_trig.h}]{../Code/geometric_trig.h}

\subsection{Square Root Code}
Code for Exact Square Root Metods:
\lstinputlisting[caption={exact\_root.c}]{../Code/exact_root.c}

Code for the Bisection Methods:
\lstinputlisting[caption={bisect\_root.c}]{../Code/bisect_root.c}

Code for Newton Square Root Methods:
\lstinputlisting[caption={newton\_root.c}]{../Code/newton_sqrt.c}

Code for Newton Inverse Square Root Methods:
\lstinputlisting[caption={newton\_inv\_sqrt.c}]{../code/newton_inv_sqrt.c}

Header files for Square Root Code:
\lstinputlisting[caption={exact\_root.h}]{../Code/exact_root.h}
\lstinputlisting[caption={bisect\_root.h}]{../Code/bisect_root.h}
\lstinputlisting[caption={newton\_root.h}]{../Code/newton_root.h}
\lstinputlisting[caption={newton\_inv\_sqrt.h}]{../Code/newton_inv_sqrt.h}


\end{document}
