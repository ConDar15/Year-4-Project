\documentclass[12pt]{article}

\usepackage[utf8]{inputenc}
\usepackage{amsthm}
\usepackage{amsmath}
\usepackage{amssymb}
\usepackage[a4paper, margin=2.5cm]{geometry}
\usepackage{natbib}
\usepackage{chngcntr}
\usepackage{color}
\usepackage{listings}
\usepackage{caption}

\definecolor{cBg}{RGB}{240,240,250}
\definecolor{cCm}{RGB}{0, 105, 0}
\definecolor{cSt}{RGB}{155, 0, 0}
\definecolor{cKw}{RGB}{0, 0, 105}

\newcommand{\TODO}[1]{\textcolor{red}{TODO: #1}}
\newcommand{\N}{\mathbb{N}}
\newcommand{\R}{\mathbb{R}}
\newcommand{\Rp}{\mathbb{R}^+}
\newcommand{\Rpz}{\mathbb{R}^+_0}
\newcommand{\Z}{\mathbb{Z}}
\newcommand{\Zp}{\mathbb{Z}^+}
\newcommand{\Zn}{\mathbb{Z}^-}
\newcommand{\Zpz}{\mathbb{Z}^+_0}
\newcommand{\bigO}{\mathcal{O}}
\newcommand{\codeinline}[1]{\colorbox{cBg}{\texttt{#1}}}

\newenvironment{subproof}[1][\proofname]{%
	\renewcommand{\qedsymbol}{$\blacksquare$}%
	\begin{proof}[#1]%
	}{\end{proof}}

\newenvironment{codelisting}[1]{%
\lstset{basicstyle=\ttfamily,%
		language=C,%
		backgroundcolor=\color{cBg},%
		basicstyle=\footnotesize,%
		commentstyle=\color{cCm},%
		frame=L,%
		keywordstyle=\color{cKw},%
		showstringspaces=false,%
		stringstyle=\color{cSt},%
		tabsize=2%
		caption=#1}%
\renewcommand{\lstlistingname}{Implementation}%
\begin{lstlisting}}{}

\setlength{\parindent}{0pt}
%\counterwithin{figure}{section}
%\counterwithin{lstlisting}{subsubsection}
\renewcommand{\familydefault}{\sfdefault}
\renewcommand{\lstlistingname}{Algorithm}
\lstset{basicstyle=\rmfamily,mathescape}
\bibpunct{[}{]}{;}{s}{,}{,}

\AtBeginDocument{%
	\counterwithin*{lstlisting}{section}
	\counterwithin*{lstlisting}{subsection}
	\counterwithin*{lstlisting}{subsubsection}
	\renewcommand{\thelstlisting}{%
		\ifnum\value{subsection}=0
			\thesection.\arabic{lstlisting}%
		\else
			\ifnum\value{subsubsection}=0
				\thesubsection.\arabic{lstlisting}%
			\else
				\thesubsubsection.\arabic{lstlisting}%
			\fi
		\fi
	}%
}

%\includeonly{./Sections/Introduction}
%\includeonly{./Sections/Root_Functions}
%\includeonly{}

\begin{document}

\author{Jake Darby}
\title{Accuracy vs Efficiency of Numerical Methods \\ \large How to program a Calculator}
\date{}
\maketitle

\begin{abstract}
\begin{center}
This document will discuss and analyse various numerical methods for computing functions commonly found on calculators. The aim of this paper is to, for each set of functions, compare and contrast several algorithms in regards to their effiency and accuracy.
\end{center}
\end{abstract}

\iffalse
\newpage
\tableofcontents
\newpage
\fi

%SEC%
\section{Introduction}
\label{SEC_"Introduction"}
For many thousands of years all calculations that a Human might want performing had to be done by hand. For simple calculations such as addition, subtraction and multiplication this was not such an issue, but as society evolved we wanted to know the answer to increasingly hard questions. The greeks saught to find a value for $\pi$, and ended up with the bounds that $\frac{223}{71} < \pi < \frac{22}{7}$, which while sufficient for their needs is not sufficient for ours in the modern era. \\

At the same time many functions were being studied to find solutions, often arising from practical concerns. For instance finding the square root of any arbitrary number has been important to architects since the time of the ancient Babylonian mathematics. Similarly long studied have been the periodic trigonometric functions due to their relation to triangles, or exponential functions due to their use in finance such as interest on loans.\\

The difficulty of these methods is that there is typically no simple way of getting an exact answer, if in fact one is available. Over time methods were developed that would allow a person to calculate an approximate answer to their problem, given enough time and patience. Such methods tended to be long and tedious work, which even lead to the profession of a human computer from the early 17\textsuperscript{th} century until the 20\textsuperscript{th} century; who would be hired to perform these long tedious calculations.\\

By the time of the Renaissance period people had started to build early mechanical calculators to help in these endeavours. Such calculators were typically capable of only addition and subtraction, which could be used to implement multiplication and division if one so wished. Later these machines became more elaborate, capable of multiple simple functions, or designed to perform one more complicated function. A famous example would be Charles Babbage's difference engine which was a large mechanical calculator that would tabulate polynomial functions developed in the early 1800s.\\

Finally in the 20\textsuperscirpt{th} century electronic computers were created and soon replaced both mechanical and human calculators. Such electronic machines had many benefits over both their human and mechanical counterparts, and soon it became common place to use electronic computers to perform mathematical computations. Nowadays computers have become faster and smaller, and the average person's phone outstrips the entire computing power of NASA during the Apollo missions.\\

However despite the speed of the calculations these modern computers still need to be instructed in how to evaluate the functions asked of it. This document will take some common functions that any calculator will answer in the blink of an eye accurate to around 10 significant digits, and explore how they may be computed. In particular this document will be comparing the speed at which these computations can be performed versus the accuracy of their results.\\

%SUB%
\subsection{Code and Computers used}
\label{SUB_"Code and Computers used}
During this project I will be discussing the implementation of various algorithms. I will be implementing these algorithms in the C programming language, using the C11 standard.\\

I chose the C programming language to implement my algorithms in because, once it compiles to binary machine code, the programs produced tend to be very efficient. This is partly due to the low-level of C programming, having relatively close control over direct CPU actions; however this does come at the cost of losing higher functionality that many other programming languages offer. A second reason for the efficiency is due to C's long history, originally being developed int 1969-1970, which has lead to several very efficient compilers.\\

I will be implementing most programs using C's built in primitive types, typically \codeinline{int}, \codeinline{unsigned int} and \codeinline{double}. On a computer an \codeinline{int} is an integer that can represent both positive and negative bits using twos compliment, this gives an \codeinline{int} using \(n\) bits a minimum value of \(-2^n\) and a maximum value of \(2^n-1\). Typically a computer will store an \codeinline{int} as 32 bits, though some computers may use more or less bits. An \codeinline{unsigned int} is very similar to an \codeinline{int}, but does not represent negative values, and thus an \codeinline{unsigned int} of \(n\) bits has a minimum value of \(0\) and a maximum value of \(2^{n+1}-1\).\\

If an integer of a specific number of bits is needed then the header \codeinline{stdint.h} may be used which defines \codeinline{int\_N} and \codeinline{uint\_N} which repsectively represent \codeinline{int} of N bits and \codeinline{unsigned int} of N bits; The typical values of N are 8, 16, 32 and 64.\\

In C a \codeinline{double} is a floating point representation of a real value, that typically follows the IEEE 754 standard for double-precision binary floating points. This standard has:
\begin{itemize}
\item 1 bit for the sign of the number, \(s\)
\item 11 bits for the exponent, \(e\)
\item 52 bits for the significand, \(b = b_0b_1b_2\dots b_{51}\)
\item A value that is understood to be:
	\[(-1)^s\left(1 + \sum_{i=1}^{52}b_{52-i}2^{-i}\right) \times 2^{e-1023
}\]
\end{itemize}

This gives a \codeinline{double} value a precision of around 15-17 significant decimal digits. While this is good for most applications, there are applications where we may want even more precision than this. To solve this I will be implementing certain algorithms using the GNU Multiple Precision Arithmetic Library (reffered to as GMP) as well as GNU MPFR Library(reffered to as GMP), wich was built upon GMP to correct and optimise the original. These libraries allow the use of arbitrary precision real values, given enough memory space, as well as integers longer than C's standard integer types can hold.\\

An important point to note that will be useful later on is that due to the storage structure of C's \codeinline{double}s and the MPFR \codeinline{mpfr\_t}s which also use a floating point represtntation. In the storage of the significand both data types work such that the value of \(b\) is in the range \([\frac{1}{2}, 1)\). This is usefull as it means that if we have a stored value \(x\), then it is very easy to extract \(\alpha\in[\frac{1}{2}), \beta \in \Z\) such that \(x = \alpha\cdot2^\beta\). The value of this observation will be in restricting the range over which functions need to be evaluated later in the document.\\

I will be compiling and testing all of my code on a benchmark machine running a light version of Ubuntu 14.04, using the GNU C Compiler. The specifications of the machine, that may impact perrformance are:
\begin{itemize}
\item An Intel i5-4690K processor running at 4GHz. This processor uses a 64 bit architecture.
\item 8Gb of DDR3 RAM
\item A modern chipset on the motherboard
\end{itemize}

%SEC%
\section{General Definitions and Theorems}
\label{SEC_"General Definitions and Theorems"}
This section will discuss those definitions and theorems that will be applicable and refferenced later in the document.

%SUB%
\subsection{Methods}

In this document we will look at various functions, such as root functions, trigonometric functions, among others. Despite the variety of functions being analysed there are several methods that are useful for more than one function, we will discuss these methods below.

%SUBSUB%
\subsubsection{Newton-Raphson Method}
\label{SUBSUB_"Newton-Raphson Method"}
\theoremstyle{definition}
\newtheorem{Newton Method}{Definition}[subsubsection]

The Newton-Raphson Method is named after Sir Isaac Newton and Joseph Raphson. It is a method that takes a continuously differentiable function \(f\) and it's derivative \(f'\), as well as an initial guess \(x_0\), to create successively more accurate solutions to \(x\) where \(f(x) = 0\).\\

The specific definition of the Newton-Raphson method that I will be using in this document is below:

%DEF%
\begin{Newton Method}
\label{DEF_"Newton-Raphson Method"}
Given \(f \in \mathcal{C}(\R)\), \(f'\) being the derivative of \(f\), and \(x_0 \in \R\); then we define:
\begin{displaymath}
	x_{n+1} := x_n - \frac{f(x)}{f'(x)} \forall n \in \N
\end{displaymath}
\end{Newton Method}

The hope is that, if we start with a good initial guess, that the method will converge to some \(x \in \R : f(x) = 0\).\\

We derive the above method in the following way. If we have an approximation \(x_n \in \R : f(x_n) \approx 0\) we consider the tangent to \(f(x)\) at \(x_n\), which is given by \(y = f'(x_n)(x-x_n) + f(x_n)\). If we set \(y = 0\) and solve for \(x\) we get that \[x = x_n - \frac{f(x)}{f'(x)}\]

The iterative formula follows by letting \(x_{n+1} := x\) in the above.\\

\TODO{Re-arrange into a better order}

\TODO{Refferences}

\TODO{Possibly include discussions of pros and cons}

%SUBSUB%
\subsubsection{Taylor Series Expansion}
\label{SUBSUB_"Taylor Series"}
\theoremstyle{definition}
\newtheorem{Taylor Series}{Definition}[subsubsection]
\newtheorem{Taylor Polynomial}[Taylor Series]{Definition}

The Taylor Series formulation was created by Brook Taylor in 1715, based off of the work of Scottish mathematician James Gregory. The Taylor Series describes a method of representing a given function by a polynomial, where any finite number of terms will give an approximation to the function itself.

%DEF%
\begin{Taylor Series}
\label{DEF_"Taylor Series"}
Given \(f : \R \to \R\) which is infinitely differentiable at \(a \in \R\), we define the Taylor Series of \(f\) at \(a\) to be:
\[\sum_{n=0}^{\infty} \frac{f^{(n)}(a)}{n!}(x-a)^n\]
\end{Taylor Series}

We can then, from our definition of a Taylor Series, define a polynomial that will approximate our function \(f\) at \(x \in \mathcal{I} \subset \R\)

%DEF%
\begin{Taylor Polynomial}
\label{DEF_"Taylor Polynomial"}
Given \(f : \R \to \R\) which has a Taylor Series of
\( \sum_{n=0}^\infty c_n x^n \), we define the Taylor Polynomial of degree \(N \in \N\) to be
\[ p_N(x) := \sum_{n=0}^N c_n x^n = c_0 + c_1 x + c_2 x^2 + \dotsb + c_N x^N\]
\end{Taylor Polynomial}

Some examples of simple Taylor Series are:
\begin{align*}
\frac{1}{1-x} &= \sum_{n=0}^\infty x^n &\forall x \in {({-1},1)}\\
\end{align*}

\TODO{Eloquate this section}

\TODO{Discuss when a Taylor Series is not equal to it's function}

\TODO{Describe intervals for which Taylor Series converge}

\TODO{Create more examples of Taylor Series that are not used later in the document}

\TODO{Possibly prove Taylor's Theorem}

%SUBSUB%
\subsubsection{CORDIC}
\label{SUBSUB_"CORDIC"}
\TODO{Describe the CORDIC Algorithm in general}

\TODO{Let other sections define specific changes to CORDIC}

%SUB%
\subsection{Errors}
\label{SUB_"Error Definitions"}
\theoremstyle{definition}
\newtheorem{Absolute Error}{Definition}[subsection]
\newtheorem{Relative Error}[Absolute Error]{Definition}
\newtheorem{Iteration Error}[Absolute Error]{Definition}

Errors are very useful in this document for discussing convergence and measuring the accuracy of a particular method. There are two distinct types of error that we are interested in:

%DEF%
\begin{Absolute Error}
\label{DEF_"Absolute Error"}
If we have a value \(v\) and it's approximation \(\tilde{v}\), then the absolute error is
\[ \epsilon := \left| v - \tilde{v} \right| \]
\end{Absolute Error}

The absolute error will be useful in guaranteeing a certain level of accuracy that a given implementation of a method will give. Uses of absolute error in the document will use \(\epsilon\) as their absolute error variable.

%DEF%
\begin{Relative Error}
\label{DEF_"Relative Error"}
If we have a value \(v\) and it's approximation \(\tilde{v}\), then the relative error is
\[ \eta := \frac{\epsilon}{\left| v \right|} = \left| 1 - \frac{\tilde{v}}{v} \right|\]
\end{Relative Error}

The relative error will also be useful in guaranteeing a certain level of accuracy that a given implementation of a method will give, particularly when the absolute error varies with the size of the input. Uses of relative error in the document will use \(\eta\) as their absolute error variable.\\

As the previous two errors are hard or impossible to accurately estimate, in a practical manner, then we want an error estimate that we can calculate in an algorithm. Thus we define the iteration error, as the absolute difference of consecutive iterations.

%DEF%
\begin{Iteration Error}
\label{DEF_"Iteration Error"}
If we have the sequence \(\left(x_n\right)_{n \in \N}\), then the iteration error is defined as:
\[ \delta_n := \left|x_n - x_{n-1}\right| \]
\end{Iteration Error}

For our practical purposes we can note that it is almost impossible to calculate \(\epsilon_n\) exactly, while \(\delta_n\) is easy to calculate and tends to be a good enough approximation of \(\epsilon_n\); particularly if the convergence is rapid.\\ 

%SUB%
\subsection{Convergence}
\label{SEC_"Convergence"}

\thoremstyle{definition}
\newtheorem{Uniform Convergence}{Definition}[subsection]
\newtheorem{Rate of Convergence}[Uniform Convergence]{Definition}

\theoremstyle{remark}
\newtheorem{Uniform Convergence R1}{Remark}[Uniform Convergence]

\theoremstyle{plain}
\newtheorem{Uniform Convergence Thm}{Theorem}[subsection]
\newtheorem{Quad Convergence of Newton}[Uniform Convergence Thm]{Theorem}

In this section we consider what it means for a sequence to converge to a limit value, and some results that will be useful in future chapters.

%DEF%
\begin{Uniform Convergence}
\label{DEF_"Uniform Convergence"}
A sequence \((x_n \in \R: n \in \N)\) converges to \(x\) uniformly if \[\forall \tau \in \Rpz \exists N \in \N : \epsilon_n := |x - x_n| < \tau \forall n \in [N, \infty) \cap \Z\]
\end{Uniform Convergence}

%RMK%
\begin{Uniform Convergence R1}
\label{RMK_"Uniform Convergence R1"}
We will typically use the notation that \(\lim_{n \to \infty} |x_n - x| = 0\), to denote that \((x_n : n \in \N)\) converges to \(x\).
\end{Uniform Convergence R1}

%THM%
\begin{Uniform Convergence Thm}
\label{THM_"Uniform Convergence Thm"}
\((x_n \in \R : n \in \N)\) converges to \(x\) uniformly if and only if \[\forall \tau \in \Rpz \exists N \in \N : |x_n - x_m| < \tau \forall m, n \in [N, \infty) \cap \Z\]
\end{Uniform Convergence Delta}
\begin{proof}
For \(\implies\):\\
Suppose that \((x_n : n \in \N)\) converges to \(x\) uniformly. Then \(\forall \tau \in \Rpz \exists N \in \N : |x_n - x| < \tau \forall n \in [N, \infty) \cap \Z\).\\
Thus suppose \(N \in \N\) is such that \(|x_n - x| < \frac{\tau}{2} \forall n \in [N, \infty) \cap \Z\).\\
Then if \(n, m \ge N\) we see that \[|x_n - x_m| \le |x_n - x| + |x_m - x| \le \tau\]\\

For \(\Leftarrow\):\\
Omitted for brevity.
\end{proof}

\begin{Rate of Convergence}
If \((x_n \in \R : n \in \N)\) is a sequence that converges to \(x\), then it is said to converge:
\begin{itemize}
\item Linearly if \(\lambda \in \Rp\) and \[\lim_{n\to\infty}\frac{|x_{n+1} - x|}{|x_n - x|} = \lambda\]
\item Quadratically if \(\lambda \in \Rp\) and \[\lim_{n\to\infty}\frac{|x_{n+1} - x|}{|x_n - x|^2} = \lambda\]
\item Order \(\alpha \in \Rpz\) if \(\lambda \in \Rp\) and \[\lim_{n\to\infty}\frac{|x_{n+1} - x|}{|x_n - x|^\alpha} = \lambda\]
\end{itemize}
\end{Rate of Convergence}

For the following proof we will have \(\epsilon_n := |x^\ast - x_n|\).

%THM%
\begin{Quad Convergence of Newton}
\label{THM_"Quad Conv Newton"}
Let \(f\) be a twice differentiable function, \(x^\ast\) be a solution to \(f(x) = 0\) and \((x_n : n \in \N)\) be a sequence produced by the Newton-Raphson Method from some initial point \(x_0\). If the following are satisfied, then \((x_n : n \in \N_0)\) converges quadratically to \(x^\ast\):
\begin{description}

\item[\(\textrm{NR}_1\):]
\begin{math}
	f'(x) \neq 0 \forall x \in I := [x^\ast - r, x^\ast + r], \ \mathrm{where}\ r \in \left[\left|x^\ast - x_0\right|, \infty\right)
\end{math}

\item[\(\textrm{NR}_2\):]
\begin{math}
	f''(x) \ \textrm{is continuous}\  \forall x \in I
\end{math}

\item[\(\textrm{NR}_3\):]
\begin{math}
	M\left|\epsilon_0\right| < 1 \ \mathrm{where}\ M := \sup\left\{\left|\frac{f''(x)}{f'(x)}\right| : x \in I\right\}\\
\end{math}
\end{description}
\end{Quad Convergence of Newton}

\begin{proof}
By Taylor's Theorem with Lagrange Remainders we have that
\begin{displaymath}
	0 = f(x^\ast) = f(x_n) + (x^\ast - x_n)f'(x_n) + \frac{1}{2}
		(x^\ast - x_n)^2f''(y_n)
\end{displaymath}
where \(0 < |x^\ast - y_n| < |x^\ast - x_n|\).\\

Then we get the following derivation:
\begin{displaymath}
\begin{align*}
	&f(x_n) + (x^\ast - x_n)f'(x_n) = 
		-\frac{1}{2}(x^\ast - x_n)^2f''(y_n)\\
	\implies &(\frac{f(x_n)}{f'(x_n)} - x_n) + x^\ast =
		-\frac{1}{2}\frac{f''(y_n)}{f'(x_n)}(x^\ast - x_n)^2
		&\textrm{as} \ \textrm{NR}_3 \implies f'(x_n) \neq 0\\
	\implies &x^\ast - x_{n+1} = 
		-\frac{1}{2}\frac{f''(y_n)}{f'(x_n)}(x^\ast - x_n)^2\\
	\implies &\epsilon_{n+1} =
		\frac{1}{2}\left|\frac{f''(y_n)}{f'(x_n)}\right|\epsilon_n^2
		&\textrm{by taking absolute values}
\end{align*}
\end{displaymath}
As \(\textrm{NR}_2\) holds then \(M\) exists and is positive, and thereforewe have:
\[\epsilon_n \le M\epsilon_{n-1}^2 \le M^{2^n - 1}\epsilon_0^{2^n}\]

We now aim to show that we have convergence, i.e. \(\lim_{n \to \infty} x_n = x^\ast\); to do this it suffices to show that \(\lim_{n\to\infty}\epsilon_n = 0\).\\

Consider the sequence \((z_n := M^{2^n - 1}\epsilon_0^{2^n} : n \in \N_0)\). We know that \(0 \le \epsilon_n \le z_n \forall n \in \N_0\), so it then follows that if \(\lim_{n \to \infty}z_n = 0\), then \(\lim_{n \to \infty}\epsilon_n = 0\) by the Squeeze Theorem \ref{#BIBREF#}.\\

Now as \(M\epsilon_0 < 1\) by \(\textrm{NR}_3\), then we see that:

\begin{displaymath}
\begin{align*}
\lim_{n\to\infty}z_n 
	&= \lim_{n\to\infty}(M\epsilon_0)^{2^n - 1}\epsilon_0\\ 
	&= \epsilon_0\lim_{n\to\infty}(M\epsilon_0)^{2^n - 1}\\
	&= \epsilon_0\cdot0
		&\textrm{because } \lim_{n\to\infty}r^{m_n} = 0 \textrm{where }
			|r| < 1, m_n \ge n \forall n \in \N\\
	&= 0
\end{align*}
\end{displaymath}

Now to show that this sequence converges quadratically we see that \(\epsilon_{n+1} = \frac{1}{2}\left|\frac{f''(y_n)}{f'(x_n)}\right|\epsilon_n^2\), and therefore \(\frac{\epsilon_{n+1}}{\epsilon_n^2} = \frac{1}{2}\left|\frac{f''(y_n)}{f'(x_n)}\right|\).\\

Because \(|x^\ast - y_n| < |x^\ast - x_n|\) and \(\lim_{n\to\infty}x_n = x^\ast\), then it follows that \(\lim_{n\to\infty}y_n = x^\ast\). Therefore we see that
\[\lim_{n\to\infty}\frac{\epsilon_{n+1}}{\epsilon_n} = \frac{1}{2}\left|\frac{f''(x^\ast)}{f'(x^\ast)}\right| \in \Rp\]

Hence as the above limit exists and is positive then the sequence is quadratically convergent.
\end{proof}

%SUB%
\subsection{Efficiency and Accuracy Metrics}
\label{SUB_"Efficiency and Accuracy Metrics"}


%SUB%
\section{Division and Multiplication}
\label{SUB_"Division and Multiplication"}
\subsection{Introduction}
Though the idea of Division and Multiplication can seem fairly simple, particularly from an abstract pure mathematical point of view, these calculations can be computationally difficult. This section will show a few algorithms designed to calculated these values, and discuss their implementation and efficiency.

\subsection{Multiplication}
\subsubsection{Basic Methods}
\subsubsection{Advanced Methods}
\subsubsection{Analysis}

\subsection{Division}
\subsubsection{Basic Methods}
\subsubsection{Advanced Methods}
\subsubsection{Analysis}
\TODO{Section to be filled out, no work currently done on this section}

%SEC%
\section{Trigonometric Functions}

This section will focus on trigonometric functions, which are commonly used cyclic functions. These functions have been studied for hundreds of years, and can be challenging to calculate. We will discuss several methods of calculating them below before comparing methods.

\TODO{Extend and Eloquate introduction}

%SUB%
\subsection{Calculating \pi}
\label{SUB_"Calculating pi"}

Several of the methods in this section require that we already know the value of \(\pi\), for example when we are applying several trig identities. Here we will briefly discuss several methods for calculating the value of \(\pi\), so that we may use this value in later subsections.\\

The first method to consider is the method used by ancient mathematicians, such as the Greeks and Chinese. We know that if the radius of the circle is \(\frac{1}{2}\), then the circumference of the circle is \(\pi\), and the value is between the perimeters of the inner and outer polygon perimeters. The internal perimeter is \(p_n = n\sin(\frac{\pi}{n})\) and the external perimeter is \(P_n = n\tan(\frac{\pi}{n})\).

%FIG%
\begin{figure}[!ht]
	\label{FIG_"Pi Diagram 1"}
	\caption{Ancient method of calculating \(\pi\)}
	\centering
	\includegraphics[width=0.5\textwidth]{"./Diagrams/Pi Diagram 1"}
\end{figure}

As we know the values of \(\tan(\frac{\pi}{6})\) and \(\sin(\frac{\pi}{6}\), then we can calculate \(P_6\) and \(p_6\). It has be shown that \(P_{2n} = \frac{2p_nP_n}{p_n + P_n}\) and \(p_{2n} = \sqrt{p_nP_{2n}}\), which allows us to create an iterative method to approximate \(\pi\), by taking the mid-point of the successive polygon perimeters.\\

Other common historical methods for approximating \(\pi\) are to use infinite series. One such method uses the series expansion of \(\tan^{-1}\), which is discussed in detail below, where \(\tan^{-1}(1) = \frac{\pi}{4}\). This gives the following approximation using \(N\) terms:

%EQN%
\begin{equation}
\label{EQN_"Tan pi Series"}
\pi = 4\sum_{n=0}^{N} \frac{(-1)^n}{2n+1} = \sum_{n=0}^N \frac{8}{(4n+1)(4n+3)}
\end{equation}

This sequence converges very slowly, with sublinear convergence, to the correct value. More modern methods have typically revolved around finding more rapidly converging infinite series, examples include Ramanujan's series:

\begin{equation}
\frac{1}{\pi} = \frac{2\sqrt{2}}{9801}\sum_{n=0}^\infty \frac{(4n)!(1103 + 26390n)}{(k!)^n396^{4n}}
\end{equation}

or the Chudnovsky algorithm:

\begin{equation}
\frac{1}{\pi} = 12\sum_{n=0}^\infty \frac{(-1)^n(6n)!(13591409 + 545140134n)}{(3n)!(n!)^3640320^{3n + \frac{3}{2}}}
\end{equation}

This final series is extremely rapidly convergant to the value of \(\frac{1}{\pi}\), for example just the first term gives \(\pi\) accurate to 13 decimal places while we can get \(\pi\) accurate to 1000 decimal places with summing just 71 terms. Compared to Equation \ref{EQN_"Tan pi Series} which takes the summation of 500 terms to acheive the same 1000 digits of accuracy.\\

To get large degrees of accuracy for \(\pi\) is extremely computer intensive and using the \codeinline{mpfr} requires the number of bits of precision and number of terms to be set. This makes calculating \(\pi\) to a large number of decimal places, for example 1000000, computationally infeasible on a regular home computer. Therefore for our purposes we will use the precalculated value of \(\pi\) to 1000000 decimal places as listed on \texttt{http://www.exploratorium.edu/pi/pi\_archive/Pi10-6.html}

%SUB%
\subsection{Geometric Method}
\label{SUB_"Trig Geometric Method"}

\theoremstyle{plain}
\newtheorem{Geo Trig Prop 1}{Proposition}[subsection]
\newtheorem{Geo Trig Prop 2}[Geo Trig Prop 1]{Proposition}
\newtheorem{Geo Trig Prop 3}[Geo Trig Prop 1]{Proposition}

The first method I will be discussing is a method based on geometric properties that are derived on a circle, and we will start by considering values of \(\cos\) in the range \([0, \frac{\pi}{2}]\). To do this we will consider the following figure of the unit circle:

%FIG%
\begin{figure}[!ht]
	\label{FIG_"Geometric Trig 1"}
	\caption{Diagram showing angles to be dealt with}
	\centering
	\includegraphics[width=0.5\textwidth]{"./Diagrams/Geometric Trig Diagram 1"}
\end{figure}

Here theta will be given in radians, and we can note that the labelled arc has length \(\theta\) due the formula for the circumference of a circle. By using the following derivation we can find a formula for \(\theta\) in terms of \(s\):

\begin{displaymath}
\begin{align*}
	s^2 &= \sin^2\theta + (1 - \cos\theta)^2\\
	    &= (\sin^2\theta + \cos^2\theta) + 1 - 2\cos\theta\\
		&= 2 - 2 \cos\theta 
			&\mathrm{By using } \sin^2\theta + \cos^2\theta = 1\\
	\cos\theta &= 1 - \frac{s^2}{2}
\end{align*}
\end{displaymath}

We will now consider a second diagram which will allow us to calculate an approximate value of \(s\).

%FIG%
\begin{figure}[!ht]
	\label{FIG_"Geometric Trig 2"}
	\caption{Diagram detailing how to calculate \(s\)}
	\centering
	\includegraphics[width=0.5\textwidth]{"./Diagrams/Geometric Trig Diagram 2"}
\end{figure}

We will first note that by an elementary geometry result we can know that the angle \(ABC\) is a right-angle; also we can consider that \(h\) is an approximation of \(\tfrac{\theta}{2}\), which will become relevant later. Now because \(AC\) is a diameter of our circle then it's length is 2 and thus, by utilising Pythagarus' Theorem, we get that the length of \(AB\) is \(\sqrt{AC^2 - BC^2} = \sqrt{4 - h^2}\).\\

From here we consider the area of triangle \(ABC\), which can be calculated as \(\frac{1}{2}\cdot h\cdot\sqrt{4-h^2}\) and as \(\frac{1}{2}\cdot2\cdot\frac{s}{2}\); by equating these two, squaring both sides and re-arranging we get that \(s^2 = h^2(4 - h^2)\). Now we have the basis for a method that will allow us to calculate \(\cos\theta\).\\

To complete our method we will consider introducing a new line that is to \(h\) what \(h\) is to \(s\) as shown in the diagram below:

%FIG%
\begin{figure}[!ht]
	\label{FIG_"Geometric Trig 3"}
	\caption{Detailing the recursive steps}
	\centering
	\includegraphics[width=0.5\textwidth]{"./Diagrams/Geometric Trig Diagram 3"}
\end{figure}

It is easy to see that if we repeat the steps above we get that \(h^2 = \hat{h}^2(4 - \hat{h}^2)\), and it also follows that \(\hat{h} \approx \frac{\theta}{4}\). Using this we can take an initial guess of \(h_0 := \frac{\theta}{2^k}\), for some \(k \in \N\), and then calculate \(h_{n+1}^2 = h_n^2(4 - h_n^2)\) where \(n \in [0, k] \cap \Z\); finally we calculate \(\cos\theta = 1 - \frac{h_k^2}{2}\), giving the following algorithm:
  
%PCD%
\begin{lstlisting}[numbers=left,frame=single,mathescape,caption={Geometric calculation of \(\cos\)},label={PCD_"Geometric Cos"}]
  geometric_cos($\theta \in [0, \frac{\pi}{2}], k \in \N$)
      $h_0 := \tfrac{\theta}{2^k}$
      $n := 0$
      while $n < K$:
          $h_{n+1}^2 := h_n^2\cdot(4 - h_n^2)$
          $n \mapsto n + 1$
      return $1 - \tfrac{h_k^2}{2}$
\end{lstlisting}\\

Now we can use the above pseudocode to calculate any trigonometric function value by using various trigonometric identities. First we  suppose \(\theta \in \R\), then we can repeatedly apply the identity \(\cos\theta = \cos(\theta \pm 2\pi)\) to either add or subtrack \(2\pi\) until we have a value \(\theta' \in [0, 2pi)\). Once we have this value we can utilise the following assignment to calculate \(\cos\theta\):

\begin{displaymath}
	\cos\theta = \left\{ \begin{array}{lcl}
			\cos\theta' & : & \theta' \in [0, \frac{\pi}{2}]\\
			-\cos(\pi - \theta') & : & \theta' \in [\frac{\pi}{2}, \pi]\\
			-\cos(\theta' - \pi) & : & \theta' \in [\pi, \frac{3\pi}{2}]\\
			\cos(2\pi - \theta') & : & \theta' \in [\frac{3\pi}{2}, 2\pi)
		\end{array}\right.
\end{displaymath}

Using Algorithm \ref{PCD_"Geometric Cos"} we can also easily calculate both \(\sin\theta\) and \(\tan\theta\), by further use of trigonometric identities. In particular we note that \(\sin\theta = \cos(\theta - \frac{\pi}{2}\) and \(\tan\theta = \frac{\sin\theta}{\cos\theta}\). Hence we can now calculate the trigonometric function value of any angle.\\

We now wish to analyse the error of our approximation for \(\cos\), as the other methods have errors that are derivative of the error for approximating \(\cos\). Now Figure \ref{FIG_"Geometric Trig 4"} shows an arc of a circle which creates chord \(x\), with this we will be able to calculate the exact length of the chord and thus work on the error of our approximations.\\

%FIG%
\begin{figure}[!ht]
	\caption{Diagram to find actual arc approximation}
	\label{FIG_"Geometric Trig 4"}
	\centering
	\includegraphics[width=0.5\textwidth]{"./Diagrams/Geometric Trig Diagram 4"}
\end{figure}

To start we will note that \(\phi = \frac{\pi - \theta}{2} = \frac{\pi}{2} - \frac{\theta}{2}\), and then by using the Sine Law we get 
\[\frac{x}{\sin\theta} = \frac{1}{\sin\phi} \implies x = \frac{\sin\theta}{\sin\phi}\]

Now we can recall the double angle formula for \(\sin\), which gives \(\sin\theta = 2\sin\frac{\theta}{2}\cos\frac{\theta}{2}\), and also \(\sin\phi = \cos\frac{\theta}{2}\). This allows us to see that
\[x = \frac{2\sin\frac{\theta}{2}\cos\frac{\theta}{2}}{\cos\frac{\theta}{2}} = 2\sin\tfrac{\theta}{2}\]

Therefore we see that \(h_n\) is approximating the chord length associated with angle \(\theta2^{n-k}\), and thus \(\epsilon_n = |h_n - 2\sin(\theta2^{n-k-1})|\). Now as \(h_0 =\theta2^{-k} \approx 2\sin(\theta2^{-k-1})\) then if follows that \(\exists \phi\) such that \(h_0 = 2\sin(\phi2{-k-1})\), from this we can see that \(\phi = 2^{k+1}\sin^{-1}(\theta2^{-k-1})\). We will uses these facts to prove a couple of propositions.

%THM%
\begin{Geo Trig Prop 1}
\label{THM_"Geo Trig Prop 1"}
\(h_n = 2\sin(\phi2^{n-k-1}) \forall n \in [0, k] \cap \Z\) where \(\phi := 2^{k+1}\sin^{-1}(\theta2^{-k-1})\).
\end{Geo Trig Prop 1}
\begin{proof}
Proceed by induction on \(n \in [0, k]\cap\Z\).\\
\begin{description}
\item [\textrm{H\((n)\):}] \(h_n = 2\sin(\phi2^{n-k-1})\)
\item [\textrm{H\((0)\):}] 
	\begin{displaymath}
		\begin{align*}
			2\sin(\phi2^{-k-1}) &= 2\sin(\sin^{-1}(\phi2^{-k-1}))\\
								&= \theta2^{-k}\\
								&= h_0 & \textrm{by definition of } h_0
		\end{align*}
	\end{displaymath}
\item [\textrm{H\((n)\) \(\implies\) H\((n+1)\):}]
	\begin{displaymath}
		\begin{align*}
			h_{n+1} &= h_n\sqrt{4-h_n^2}\\
					&= 2\sin(\phi2^{n-k-1})\sqrt{4-4\sin^2(\phi2^{n-k-1})}
						&\textrm{by H\((n)\)}\\
					&= 4\sin(\phi2^{n-k-1})\cos(\phi2^{n-k-1})\\
					&= 2\sin(\phi2^{n-k})
						&\textrm{by the use of double angle formulas}
		\end{align*}
	\end{displaymath}
\end{description}
\end{proof}

%THM%
\begin{Geo Trig Prop 2}
\label{THM_"Geo Trig Prop 2"}
\(h_n > 2\sin(\theta2^{n-k-1}) \forall n \in [0,k] \cap \Z\)
\end{Geo Trig Prop 2}
\begin{proof}
We start by considering the expansion of the exact value of \(h_n\).
\begin{displaymath}
	\begin{align*}
		h_n &= 2\sin(\phi2^{n-k-1})\\
			&= 2\sin(2^{n-k-1}(2^{k+1}\sin^{-1}(\theta2^{-k-1})))\\
			&= 2\sin(2^n\sin^{-1}(\theta2^{-k-1}))\\
			&= 2\sin(\theta2^{n-k-1} + \tfrac{1}{6}\theta^32^{n-3k -3} 
				+ \bigO(2^{-5k}))
				& \textrm{Detailed in section \ref{#REF#}}
	\end{align*}
\end{displaymath}

Now as we know that \(n \le k\), then it follows that \(\theta2^{n-k-1} \le \tfrac{1}{2}\theta\).\\

Also as \(\theta \le \tfrac{\pi}{2}\) we know that \(\theta2^{n-k-1} \le \tfrac{\pi}{4}\).\\

We can also show that \(\tfrac{1}{6}\theta^32^{n-3k-3} + \bigO(2^{-5k}) \le \tfrac{\pi}{4}\), though the proof is ommited here for brevity; therefore we see that \(\phi2^{n-k-1} \le \tfrac{\pi}{2}\), and obviously that \(\phi2^{n-k-1} > \theta2^{n-k-1}\).\\

Hence, as \(\sin\) is an increasing function in the range \([0, \tfrac{\pi}{2}]\), we conclude that \[h_n = 2\sin(\phi2^{n-k-1}) > 2\sin(\theta2^{n-k-1})\].
\end{proof}

With these two propositions we can now consider the error of our approximation of \(\cos\). First we will prove the following proposition regarding the error of the approximation of \(s\):

%THM%
\begin{Geo Trig Prop 3}
\label{THM_"Geo Trig Prop 3"}
If \(\epsilon_n := |h_n - 2\sin(\theta2^{n-k-1})| \forall n \in [0,k] \cap \Z\), then \(\epsilon_k < 2^k\epsilon_0)\).
\end{Geo Trig Prop 3}
\begin{proof}
\(\epsilon_n = h_n - 2\sin(\theta2^{n-k-1})\) as \(h_n > 2\sin(\theta2^{n-k-1})\) by Proposition \ref{THM_"Geo Trig Prop 2"}.\\

Now we see that:
\begin{displaymath}
\begin{align*}
	\epsilon_{n+1} &= h_{n+1} - 2\sin(\theta2^{n-k})\\
		&= h_n\sqrt{4-h_n^2} 
			- 4\sin(\theta2^{n-k-1})\cos(\theta2^{n-k-1})\\
\end{align*}
\end{displaymath}

If we consider the equation \(\alpha\beta - \gamma\delta = (\alpha - \gamma) + \alpha(\beta - 1) - \gamma(\delta - 1)\) and apply it to our current formula we get:

\begin{displaymath}
\begin{align*}
	\epsilon_{n+1} &= (h_n - 2\sin(\theta2^{n-k-1})) 
						+ h_n(\sqrt{4 - h_n^2} - 1)
						- 2\sin(\theta2^{n-k-1})(2\cos(\theta2^{n-k-1}) - 1)\\
		&= \epsilon_n + h_n(\sqrt{4 - h_n^2} - 1)
			-2\sin(\theta2^{n-k-1})(2\cos(\theta2^{n-k-1}) - 1)\\
		&= 2\epsilon_n + h_n(\sqrt{4 - h_n^2} - 2)
			-2\sin(\theta2^{n-k-1})(2\cos(\theta2^{n-k-1}) - 2)\\
		&= 2\epsilon_n + h_n(\sqrt{4 - h_n^2} - 2)
			+2\sin(\theta2^{n-k-1})(2 - 2\cos(\theta2^{n-k-1}))\\
		&< 2\epsilon_n + h_n(\sqrt{4 - h_n^2} - 2\cos(\theta2^{n-k-1}))\\
		&< 2\epsilon_n + h_n(\sqrt{4 - 4\sin^2(\theta2^{n-k-1})}
			- 2\cos(\theta2^{n-k-1}))\\
		&= 2\epsilon_n + h_n(2\cos(\theta2^{n-k-1}) 
			- 2\cos(\theta2^{n-k-1}))\\
		&= 2\epsilon_n
\end{align*}
\end{displaymath}

The inequalities in the above derivation arrise from the fact that \(h_n > 2\sin(\theta2^{n-k-1})\) by Proposition \ref{THM_"Geo Trig Prop 2"}.\\

Hence as we now know that \(\epsilon_{n+1} < 2\epsilon_n\), we then see that \(\epsilon_n < 2^n\epsilon_0\). Therefore we prove our statement that
\[\epsilon_k < 2^k\epsilon_0\]
\end{proof}

Obviously \(\epsilon_k = |h_k - s|\), and we can now use this to find the error of our final answer. First we will start by letting \(\mathcal{C} := 1-\tfrac{1}{2}h_k^2\) and note that analytically \(cos\theta = 1 - \tfrac{1}{2}s^2\). Therefore we will now consider \(\epsilon_{\mathcal{C}} = |\mathcal{C} - \cos(\theta)|\):

\begin{displaymath}
\begin{align*}
	\epsilon_{\mathcal{C}} &= | 1 - \frac{h_k^2}{2} - 1 + \frac{s^2}{2}|\\
		&=\frac{1}{2}|h_k^2 - s^2|\\
		&=\frac{1}{2}|h_kh_k 
			- 2\sin(\frac{\theta}{2})2\sin(\frac{\theta}{2})|\\
		&=\frac{1}{2}(h_kh_k 
			- 2\sin(\frac{\theta}{2})2\sin(\frac{\theta}{2})
			&\textrm{as \(2\sin(\frac{\theta}{2}) < h_k\)}\\
		&=\frac{1}{2}(2\epsilon_k + h_k(h_k-2) - 2\sin(\frac{\theta}{2})
			(2\sin(\frac{\theta}{2}) - 2)\\
		&<\frac{1}{2}(2\epsilon_k + h_k(h_k - 2\sin(\frac{\theta}{2})))\\
			&=\frac{1}{2}(2+h_k)\epsilon_k\\
		&=\frac{1}{2}(2 + 2\sin(\frac{\phi}{2}))\epsilon_k\\
		&=(1 + \sin(\frac{\phi}{2}))\epsilon_k\\
		&\le2\epsilon_k
\end{align*}
\end{displaymath}

As \(\epsilon_{\mathcal{C}} \le 2\epsilon_k\), then by Proposition \ref{THM_"Geo Trig Prop 3"} we see that \(\epsilon_{\mathcal{C}} < 2^{k+1}\epsilon_0\). Now to consider \(\epsilon_0\) we first observe that \(\epsilon_0 =\theta2^{-k} - 2\sin{\theta2^{-k-1}\), and therefore we can conclude that:

\[\epsilon_{\mathcal{C}} < 2\theta - 2^{k+2}\sin(\theta2^{-k-1})\]

This looks like an error that may infact grow exponentially large as \(k \to \infty\), due to the multiplication by \(2^{k+2}\). However if we instead consider the series expansion of \(\sin(x)\), shown in Section \ref{SUB_"Taylor Series Trig"} to be \(\sin(x) = x - \frac{1}{3!}x^3 + \frac{1}{5!}x^5 - \cdots\), and substitute that into our equation we see that:

\begin{align*}
	\epsilon_{\mathcal{C}} &< 2\theta - 2^{k+2}(\theta2^{-k-1} 
		- \frac{1}{3!}\theta^32^{-3k-3} 
		+ \frac{1}{5!}\theta^52^{-5k-5} - \cdots)\\
	&= 2\theta - 2\theta + \frac{1}{3}\theta^32^{-2k-1}
		- \frac{1}{5!}\theta^52^{-4k-3} + \cdots\\
	&= \frac{1}{3}\theta^32^{-2k-1} - \frac{1}{5!}\theta^52^{-4k-1}
		+ \cdots
\end{align*}

Now obviously the last line tends towards zero as \(k\) tends to infinity, due to it being a formula of order \(\bigO(2^{-2k-1})\). Therefore we know that \(\forall \tau \in \Rp \exists \mathcal{K} \in \N : \epsilon_{\mathcal{C}, k} < \tau \forall k \in [\mathcal{K}, \infty) \cap \Z\). In particular, if we then wish to calculate \(\cos\theta\) accurate to \(N\) decimal places then we are looking to find \(k \in \N\) such that:

\[2\theta - 2^{k+2}\sin(\theta2^{-k-1}) < 10^{-N} \implies 2^{k+2}\sin(\theta2^{-k-1}) > 2\theta - 10^{-N}\]

For an example of the above in action we will be taking \(\theta = 0.5\). The table below shows the minimum \(k \in \N\) to guarantee \(N\) digits of accuracy in the result:

%TBL%
\begin{center}
\begin{tabular}{|p{3cm}|p{3cm}|}
	\hline
	\(N\) & \(k\)\\
	\hline
	5 & 6\\\hline
	10 & 14\\\hline
	50 & 80\\\hline
	100 & 163\\\hline
	1000 & 1658\\\hline
\end{tabular}
\end{center}

As can be seen the value of \(k\) required to acheive \(N\) digits of accuracy increases roughly linearly when \(\theta = 0.5\). Testing for other values of \(\theta\) reveals them to have similar required values for \(k\), at least within the same order of each other.\\

Another consideration for Algorithm \ref{PCD_"Geometric Cos"} is that we could "run it in reverse" to attain an algorithm for the inverse cosine function. To start take line 7 which is \(\mathcal{C} = 1 - \frac{1}{2}h_k^2\), which can be re-arranged to give \(h_k^2 = 2 - 2\mathcal{C}\), where we know \(\mathcal{C}\) as our initial value.\\

Line 5 is a little more difficult, but by re-arranging we see that \(h_n^4 - 4h_n^2 + h_{n+1}^2 = 0\), which can be solved via the quadratic formula to give \(h_n^2 = 2 \pm \sqrt{4 - h_{n+1}^2}\). Now we can make the observation that if \(x \in \Rpz\), then \(\cos^{-1}(-x) = \pi - \cos^{-1}(x)\) and so we can restric our algorithm to only consider \(x \in [0,1]\). With this we know that \(\theta \in [0,\frac{\pi}{2}]\), and thus \(h_k \le \sqrt{2}\). Therefore as \(h_{n+1} > h_n \forall n \in [0,k-1]\cap\Z\) we see that \(h_n^2 \le 2 \forall n \in [0,k]\cap\Z\). This allows us to ascertain that to reverse Line 5 we perform \(h_n^2 = 2 - \sqrt{4 - h_{n+1}^2}\).\\

Finally line 2 is reversed by returning the value \(2^kh_0\); therefore we get the following algorithm for \(\cos^{-1}(x)\) where \(x \in [0,1]\):

%PCD%
\begin{lstlisting}[numbers=left,frame=single,mathescape,caption={Geometric calculation of \(\cos^{-1}\)},label={PCD_"Geometric aCos"}]
  geometric_aCos($x \in [0,1], k \in \N$)
      $h_k := 2 - 2x$
      $n := k-1$
      while $n \ge 0$:
          $h_n^2 := 2 - \sqrt{4 - h_{n+1}^2}$
          $n \mapsto n - 1$
      return $2^kh_0$
\end{lstlisting}\\

Similar to the regular trigonometric functions we can use trigonometric identities to calculate the inverse trigonometric functions from \(\cos^{-1}\). To start we recall that \(\cos^{-1}(-x) = -\cos(x)\) where \(x \in [0,1]\), then we can use the identities that \(\sin^{-1}(x) = \frac{\pi}{2} - \cos^{-1}(x)\) and \(\tan^{-1}(x) = \sin^{-1}(\frac{x}{\sqrt{x^2 + 1}})\).\\

If we suppose that all operations in the method are accurately computed then Algorithm \ref{PCD_"Geometric aCos"} is a computation with high accuracy. This is because there is no initial guess, such as in Algorithm \ref{PCD_"Geometric Cos"}, and so the only introduction of error is assuming that \(2^kh_0 \approx \theta\). However as we discuss in detail in Section \ref{#SEC#}, calculating square roots is not a simple task and thus will introduce error to the method in general; therefore the accuracy of the method is roughly as accurate as our method of calculating square roots.

%SUB%
\subsection{Taylor Series}
\label{SUB_"Taylor Series Trig"}

If we consider our definition of a McClaurin Series from Section \ref{#SUB#}, we can use this to approximate our Trigonometric Functions. Consider first \(\cos\theta\), for which we know that \(\frac{d}{d\theta}\cos\theta = - \sin\theta\); it then follows that \(\frac{d^2}{d\theta^2}\cos\theta = -\cos\theta\), \(\frac{d^3}{d\theta^3} \cos\theta = \sin\theta\) and \(\frac{d^4}{d\theta^4} \cos\theta = \cos\theta\).\\

If we let \(f(x) = \cos x\) and use the known values \(\cos(0) = 1\) and \(\sin(0) = 0\), then we see that:

\begin{displaymath}
	f^{(n)}(0) = \left\{
		\begin{array}{lcl}
			1 &:& 4 \mid n\\
			0 &:& 4 \mid n-1\\
			-1 &:& 4 \mid n-2\\
			0 &:& 4 \mid n-3
		\end{array}\right.
\end{displaymath}

By simplifying this by ommitting the \(0\) coefficient terms we get the following series:

%EQN%
\begin{equation}
\label{EQN_"Cos Series Formula"}
\sum_{n=0}^\infty \frac{(-1)^n}{(2n)!}x^{2n} = 1 - \frac{x^2}{2!} + \frac{x^4}{4!} - \cdots
\end{equation}

By using similar working we can get that the series associated with \(\sin\(x)\):

%EQN%
\begin{equation}
\label{EQN_"Sin Series Formula"}
\sum_{n=0}^\infty \frac{(-1)^n}{(2n+1)!}x^{2n+1} = x - \frac{x^3}{3!} + \frac{x^5}{5!} - \cdots
\end{equation}

Before we go any further we need to consider when Equations \ref{EQN_"Cos Series Formula"} and \ref{EQN_"Sin Series Formula"} converge to their respective functions. To do this we will use the ratio test for series as defined in \ref{#SEC#}, using Equation \ref{EQN_"Cos Series Formula"} we see that

\begin{displaymath}
	\begin{align*}
		L_{\mathcal{C}} &= \lim_{n \to \infty} \left| 
			\frac{a_{n+1}}{a_n} \right|\\
		&= \lim_{n \to \infty} \left| 
			\frac{\frac{(-1)^{n+1}}{(2n+2)!}x^{2n+2}}
				{\frac{(-1)^n}{(2n)!}x^{2n}} \right|\\
		&=\frac{(2n)!}{(2n+2)!}|x|^2\\
		&=\frac{1}{(2n+2)(2n+1)}|x|^2
	\end{align*}
\end{displaymath}

Now it is easy to see that, \(L_{\mathcal{C}} = 0\) for all values of \(x\) as the fractional component decreases as \(n\) increases and \(|x|^2\) is a constant. Therefore we can conclude that Equation \ref{EQN_"Cos Series Formula"} converges to \(\cos(x)\) for all values of \(x\). We can use a very similar deduction to show that Equation \ref{EQN_"Sin Series Formula"} converges to \(\sin(x)\) for all values of \(x\).\\

The above means that \(\cos\) and \(\sin\) can be approximated using Taylor Polynomials, in particular for a given \(N \in \N\):
\begin{displaymath}
\begin{array}{rcl}
	\cos x \approx \sum_{n=0}^N \frac{(-1)^n}{(2n)!}x^{2n}
	& \textrm{and}
	&\sin x\approx \sum_{n=0}^N \frac{(-1)^n}{(2n+1)!}x^{2n+1}
\end{array}
\end{displaymath}

This allows us to create the following two methods for computing \(\cos x\) and \(\sin x\):

%PCD%
\begin{lstlisting}[numbers=left,frame=single,mathescape,caption={Taylor computation of \(\cos\) and \(\sin\)},label={PCD_"Taylor Cos/Sin"}]
  taylor_cos($x \in \R, N \in \N$)
      $\mathcal{C} := 0$
      $n := 0$
      while $n < N$:
          $\mathcal{C} \mapsto \mathcal{C} + (-1)^n\cdot\tfrac{1}{(2n)!}x^{2n}$
          $n \mapsto n+1$
      return $\mathcal{C}$
  
  taylor_sin($x \in \R, N \in \N$)
      $\mathcal{S} := 0$
      $n := 0$
      while $n < N$:
          $\mathcal{S} \mapsto \mathcal{S} + (-1)^n\cdot\tfrac{1}{(2n+1)!}x^{2n+1}$
          $n \mapsto n+1$
      return $\mathcal{S}$
\end{lstlisting}

As these two methods are obviously very similar and the fact that \(\sin(x) = \cos(x - \frac{\pi}{2})\), we will continue by examining only the taylor method for approximating \(\cos\). We will assume that any calculations for \(\sin\) are transformed into a problem of finding a \(\cos\) value.\\

It should be noted that this \(\cos\) algorithnm is particularly inefficient to calculate on a computer implementation; this is primarily due to the way in which the update of \(\mathcal{C}\) is calculated each loop.\\

In each loop we are calculating \(x^{2n}\), which has a naieve complexity of \(\bigO(2n)\). However what we are actually calculating \(x^{2(n-1)}\cdot x^2\) and thus if we store the values of \(x^{2(n-1)}\) and \(x^2\), the complexity of this step drops to \(\bigO(1)\). Similarly we are also calculating \(\tfrac{1}{(2n)!}\) in each loop which, by the same logic, is \(\tfrac{1}{2(n-1)!} \cdot \tfrac{1}{(2n)(2n-1)}\), and we can use the same storage and update method as for \(x^{2n}\).\\

As another step towards optimizing the algorithm we can start with an initial value of \(\mathcal{C} = 1\), and then perform two updates of \(\mathcal{C}\) each loop until we reach or surpass \(N\). This saves calculating \((-1)^n\) each loop, by explicitly performing two different calculations. Implementing all of the above gives us the following two updated methods:

%PCD%
\begin{lstlisting}[numbers=left,frame=single,mathescape,caption={Taylor computation of \(\cos\) optimised},label={PCD_"Taylor Cos opt"}]
  taylor_cos($x \in \R, N \in \N$)
      $\mathcal{C} := 1$
      $x_2 := x^2$
      $a := 1$
      $b := 1$
      $n := 1$
      while $n < N$:
          $a \mapsto a \cdot \tfrac{1}{(2n-1)(2n)}$
          $b \mapsto b \cdot x_2$
          $\mathcal{C} \mapsto \mathcal{C} - a\cdot b$
          $a \mapsto a \cdot \tfrac{1}{(2n+1)(2n+2)}$
          $b \mapsto b \cdot x_2$
          $\mathcal{C} \mapsto \mathcal{C} + a\cdot b$
          $n \mapsto n+2$
      return $\mathcal{C}$
\end{lstlisting}

As the next term of the polynomial is known definitively then we can see that it is very easy to calculate the error of our approximation. We see that 
\begin{displaymath}
\begin{align*}
	\epsilon_N &= |\cos(x) - \mathrm{taylor\_cos(x,N)}|\\
		&= \bigO(|x|^{N'+1}) &\textrm{where } N' \textrm{ is the smallest}\\
		&&\textrm{odd integer such that } N'\ge N\\
		&\le \frac{1}{(2(N'+1))!} |x|^{N'+1}\\
		&\le \frac{1}{(2(N+1))!} |x|^{N+1}
\end{align*}
\end{displaymath}

If we place bounds on the value of \(\cos\) calculated as in Section \ref{SUB_"Trig Geometric Method"}, then we know that \(|x| \le \frac{\pi}{2}\), and thus we get the following bound for the error of our approximation:

\[\epsilon_N \le \frac{\pi^{N' + 1}}{2^{N'+1}(2(N'+1))!}\]

Thus if we find \(N \in \N\) such that \(\frac{\pi^{N}+1}{2^{N+1}(2(N+1)!)} < \tau \in \R+\) then we know that \(\epsilon_N < \tau\). If we consider \(\tau = 10^k\), then we can find \(N \in \N\) such that our approximation is accurate to \(k\) decimal places. Below is a table which details some values of \(k\) and the corresponding minimum \(N\) to guarantee \(k\) decimal places of accuracy:

%TBL%
\begin{center}
\begin{tabular}{|p{3cm}|p{3cm}|}
	\hline
	\(k\) & \(N\)\\
	\hline
	5 & 4\\\hline
	10 & 7\\\hline
	50 & 21\\\hline
	100 & 36\\\hline
	1000 & 233\\\hline
\end{tabular}
\end{center}

Now for \(\tan x\) we can either calculate both \(\sin x\) and \(\cos x\) using \mathrm{taylor\_cos(x,N)} and divide the resulting value, or we can calculate \(\tan x\) directly using a Taylor expansion.\\

In calculating the McClaurin series for \(\tan x\) we start by letting \(\tan x = \sum_{n=0}^\infty a_nx^n\), and then noting that as \(\tan x\) is an odd series then it's McClaurin series only contains non-zero coefficients for odd powers of \(x\); therefore we get that \(\tan x = \sum_{n=0}^\infty a_{2n+1}x^{2n+1} = a_1x + a_3x^3 + a_5x^5 + \cdots\).\\

Next we consider that \(\frac{d}{dx} \tan x = 1 + \tan^2 x\), and knowing the McClaurin series form of \(\tan x\) we get the following:

\begin{displaymath}
\begin{align*}
	\sum_{n=0}^\infty (2n+1)a_{2n+1}x^{2n} &= 1 + 
		(\sum_{n=0}^\infty a_{2n+1}x^{2n+1})^2\\
	&= 1 + a_1^2x^2 + (2a_1a_3)x^4 + (2a_1a_5 + a_3^2)x^6 + \cdots
\end{align*}
\end{displaymath}

Considering the co-efficients of powers on the right hand side of the above equation we see that \(2a_1a_3 = a_1a_3 + a_3_a_1 = a_1a_{4-1} + a_3a_{4-3}\) and \(2a_1a_5 + a_3^2 = a_1a_5 + a_3a_3 + a_5a_1 = a_1a_{6-1} + a_3a_{6-3} +a_5a_{6-5}\). This indicates that our general form for the co-efficient of \(2n\) on the right hand side is \(\sum_{k=1}^n a_{2k-1}a_{2n - 2k + 1}\), and thus returning to our equation we get

\[a_1 + \sum_{n=1}^\infty (2n+1)a_{2n+1}x^{2n} = 1 + \sum_{n=1}^\infty(\sum_{k=1}^n a_{2k-1}a_{2n-2k+1})x^{2n}\]

Using this we conclude that \(a_1 = 1\) and \(a_{2n+1} = \frac{1}{2n+1}\sum_{k=1}^n a_{2k-1}a_{2n-2k+1} \forall n \in \N\). We can note immediately that the calculation of any previous co-efficients will provide no help in calculating later co-efficients and so the entire sum must be calculated each loop, while also storing each co-efficient already calcualted.\\

This means that the complexity to calculate co-efficient \(a_{2n+1}\) is \(\bigO(n)\) and will be the \(n^\text{th}\) such calculation, making the complexity of calculating \(n\) co-efficients to be \(\bigO(n^2)\). Comparing this to the \mathrm{taylor\_cos} method we see that to calculate up to \(n\) co-efficients of both \(\cos\) and \(\sin\) has complexity \(\bigO(n)\). Therfore it is more efficient to calculate \(\tan\) by calculating both \(\cos\) and \(\sin\) using Algorithm \ref{PCD_"Taylor Cos Opt"}, and performing division than directly using Taylor Polynomial approximation.\\

We would also like to be able to calculate the inverse trigonometric functions using this method, which means we need to find our McClaurin series of the inverse trigonometric functions. The simplest of these is \(\tan^{-1}\), where we start by recalling that \(\frac{d}{dx} \tan^{-1} x = \frac{1}{1+x^2}\) and then by intergrating both sides we get:

\begin{displaymath}
\begin{align*}
	\tan^{-1} x &= \int \frac{1}{1+x^2} dx\\
		&= \int (1 - (-x^2))^-1 dx\\
		&= \int \sum_{n=0}^\infty (-x^2)^n dx &\textrm{by Equation \ref{#EQN#}}\\
		&= \int \sum_{n=0}^\infty (-1)^nx^{2n} dx\\
		&= c + \sum_{n=0}^\infty \frac{(-1)^n}{2n+1}x^{2n+1}
\end{align*}
\end{displaymath}

As \(\tan^{-1} (0) = 0\) then we see that \(c = 0\) and thus gives us the following formula for \(\tan^{-1}\):

\[\tan^{-1} x = \sum_{n=0}^\infty \frac{(-1)^n}{2n+1}x^{2n+1}\]

Now due to the restrictions from Equation \ref{#EQN#} the above is only valid for \(x \in [-1, 1]\), but we know that the domain of \(\tan^{-1}\) is \(x \in \R\). To fix this we will first recognise that \(\tan^{-1}(-x) = -\tan^{-1}(x)\), so we can restric our problem to \(x \in \Rpz\). Now if we take the double angle formula for \(\tan\):

\[\tan(\alpha + \beta) = \frac{\tan(\alpha) + \tan(\beta)}{1 - \tan(\alpha)\tan(\beta)}\]

By substituting \(\alpha = \tan^{-1}(x)\) and \(\beta = \tan^{-1}(x)\) into the above then we get

\[\tan^{-1}(x) + \tan^{-1}(y) = \tan^{-1}\left(\frac{x + y}{1 - xy}\right)\]

Using this, suppose we are looking for \(\tan^{-1}(z)\) where \(z \in (1, \infty)\) and let \(y = 1\), then \(\tan^{-1}(y) = \frac{\pi}{4}\). We can then re-arrange the equation \(z = \frac{x + 1}{1 - x}\) to get \(x = \frac{z - 1}{z + 1}\); finally as \(z > 1\), then \(0 < x < 1\). This allows us to calculate:

\[\tan^{-1}(z) = \frac{\pi}{4} + \tan^{-1}\left(\frac{z-1}{z+1}\right)\]

In the above the calculated value is in the range \([0, 1]\) and so it is valid to use a Taylor polynomial using our McClaurin series above. This gives the following method

%PCD%
\begin{lstlisting}[numbers=left,frame=single,mathescape,caption={Taylor Method for \(\tan^{-1}\)},label={PCD_"Taylor aTan"}]
  taylor_aTan($x \in [0,1], N \in \N$)
      $\mathcal{T} := 0$
      $x_2 := x^2$
      $y := x$
      $n := 0$
      while $n < N$:
          $\mathcal{T} \mapsto \mathcal{T} + \tfrac{1}{2n+1}y$
          $y\mapsto y\cdot x_2$
          $\mathcal{T} \mapsto \mathcal{T} - \tfrac{1}{2n+2}y$
          $y\mapsto y\cdot x_2$
          $n \mapsto n + 2$
      return $\mathcal{T}$
\end{lstlisting}

Similar to Algorithm \ref{PCD_"Taylor Cos Opt"} the error of Algorithm \ref{PCD_"Taylor aTan"} is easy to calculate. We see that 

\begin{displaymath}
\begin{align*}
	\epsilon_N &= |\tan^{-1}(x) - \mathrm{taylor\_aTan(x,N)}|\\
		&\le \frac{1}{2N + 3}|x|^{2N+3}\\
		&\le \frac{1}{2N + 3} &\textrm{as } x \le 1
\end{algin*}
\end{displaymath}

The next function we will consider is \(\sin^{-1}\), which starts it's derivation in much the same way as \(\tan^{-1}\). First we start by recalling that \(\frac{d}{dx} \sin^{-1}(x) = (1 - x^{2})^{-\frac{1}{2}}\), then by taking integrals of both sides we get the following derivation:

\begin{displaymath}
\begin{align*}
	\sin^{-1}(x) &= \int (1 - x^{2})^{-\frac{1}{2}} dx\\
		&= \int \sum_{n=0}^\infty \binom{-\frac{1}{2}}{n} (-x^2)^n\\
		&= c + \sum_{n=0}^\infty (-1)^n 
			\left(\prod_{k=1}^n \frac{-\tfrac{1}{2} - k + 1}{k}\right)
			\frac{x^{2n+1}}{2n+1}\\
		&= c + \sum_{n=0}^\infty \frac{(-1)^n}{n!(2n+1)} 
			\left(\prod_{k=1}^n \tfrac{1}{2} - k\right)
			x^{2n+1}\\
		&= c + \sum_{n=0}^\infty \frac{(-1)^{2n}}{n!(2n+1)}
			\left(\prod_{k=1}^n \frac{2k - 1}{2}\right)
			x^{2n+1}\\
		&= c + \sum_{n=0}^\infty \frac{1}{n!(2n+1)2^n}
			\left(\prod_{k=1}^n 2k - 1\right)
			x^{2n+1}\\
		&= c + \sum_{n=0}^\infty \frac{1}{n!(2n+1)2^n}
			(1\times3\times5\times\cdots\times(2n-1))
			x^{2n+1}\\
		&= c + \sum_{n=0}^\infty \frac{1}{n!(2n+1)2^n} \times
			\frac{1\times2\times3\times\cdots\times(2n)}{2\times4\times\cdots\times(2n)}x^{2n+1}\\
		&= c + \sum_{n=0}^\infty \frac{(2n)!}{(n!)^2(2n+1)4^n}x^{2n+1}
\end{align*}
\end{displaymath}

As \(\sin^{-1}(0) = 0\) then we see that \(c=0\). Because the above is valid for \(x \in (-1,1)\), and we know the values of \(\sin^{-1}(-1)\) and \(\sin^{-1}(1)\), then we can have the following method for evaluating \(\sin^{-1}\):

%PCD%
\begin{lstlisting}[numbers=left,frame=single,mathescape,caption={Taylor Method for \(\sin^{-1}\)},label={PCD_"Taylor aSin"}]
  taylor_aSin($x \in [-1,1], N \in \N$)
      if $x = 1$:
          return $\tfrac{\pi}{2}$
      if $x = -1$:
          return $-\tfrac{\pi}{2}$
      $\mathcal{S} := x$
      $x_2 := x^2$
      $y := x$
      $a := 1$
      $b := 1$
      $c := 1$
      $n := 1$
      while $n < N$:
          $a \mapsto 2n\cdot(2n-1)\cdot a$
          $b \mapsto n^2 \cdot b$
          $c \mapsto 4\cdot c$
          $y \mapsto x_2 \cdot y$
          $\mathcal{S} \mapsto \mathcal{S} + \tfrac{a}{b\cdot c \cdot(2n+1)}\cdot y$
          $n \mapsto n + 1$
      return $\mathcal{S}$
\end{lstlisting}

The error for this method is similar to the \(\tan^{-1}\) method, in that \(\epsilon_N \le \frac{(2(N+1))!}{((N+1)!)(2N+1)4^{N+1}}\). Finally we note that \(\cos^{-1}(x) = \tfrac{\pi}{2} - \sin^{-1}(x)\), and thus can be calculated from a value calculated with Algorithm \ref{PCD_"Taylor aSin"}.

%SUB%
\subsection{CORDIC}
\label{SUB_"CORDIC"}

\theoremstyle{plain}
\newtheorem{Cordic Gamma Property}{Proposition}[subsection]
\newtheorem{Cordic Accuracy}[Cordic Gamma Property]{Proposition}

CORDIC is an algorithm that stands for COrdinate Rotation DIgital Computer and can be used to calculate many functions, including Trigonometric Values. The CORDIC algorithm works by utilising Matrix Rotations of unit vectors. This algorithm is less accurate than some other methods but has the advantage of being able to be implemented for fixed point real numbers in efficient ways using only addition and bitshifting.\\

CORDIC works by taking an initial value of
\begin{math}
	\mathbf{x}_0 = \left( 
		\begin{array}{c}
			1 \\
			0
		\end{array} \right)
\end{math}
which can be rotated through an anti-clockwise angle of $\gamma$ by the matrix
\begin{displaymath}
	\left( \begin{array}{cc}
		\cos{\gamma} & -\sin{\gamma} \\
		\sin{\gamma} &  \cos{\gamma}
	\end{array} \right)
	= \frac{1}{\sqrt{1 + \tan{\gamma}^2}} \left( \begin{array}{cc}
		1 & -\tan{\gamma} \\
		\tan{\gamma} & 1
	\end{array} \right)
\end{displaymath}

By taking taking smaller and smaller values of $\gamma$ we can create an iterative process to find $\mathbf{x}_n$ which converges, for a given $\beta \in (-\frac{\pi}{2}, \frac{\pi}{2})$, to
\begin{displaymath}
	\left( \begin{array}{c}
		\cos{\beta}\\
		\sin{\beta}
	\end{array} \right)
\end{displaymath}

To do this we repreately add and subtract our values for \(\gamma\) from \(\beta\) to bring it as close to 0 as possible. For our purposes we wish to have a sequence \((\gamma_k : k \in [0, n] \cap \Z)\) which will allow us to construct all angles in the range \((-\frac{\pi}{2}, \frac{\pi}{2})\) to within a known level of accuracy. There are many possible choices here, but we wish to consider \((\gamma_k : k \in [0, n] \cap \Z)\) such that \(\tan \gamma_k = 2^{-k} \forall k \in [0,n] \cap \Z\).\\

We can note that the powers of 2 have a useful property, in that if \(m > n \in \N\) we see that \(\sum_{k=n}^{m-1} 2^k = 2^m - 2^n\). We wish to show that our choice for \(\gamma_k\) have a similar property which will be usefull in showing that they are a good choice for our CORIC algorithm.

%THM%
\begin{Cordic Gamma Property}
\label{THM_"Cordic Gamma Property"}
If \(m \in \Zpz\) and \(n \in \Zp\) such that \(m > n\) and \(\gamma_k = \tan^{-1}(2^-k) \forall k \in \Zpz\), then \(\gamma_m < \gamma_n + \sum_{k=m+1}^n \gamma_k\).
\end{Cordic Gamma Property}
\begin{proof}
We know that \(2^{-m} = 2^{-n} + \sum_{k=m+1}^n 2^-k\), and thus by applying \(\tan^{-1}\) to both sides we get:

\[\tan^{-1} 2^{-m} = \gamma_m = \tan^{-1}(2^{-m-1} + 2^{-m-2} + \cdots + 2^{-n} + 2^{-n})\]

Let \(a := 2^{-m-1} + 2^{-m-2} + \cdots + 2^{-n} + 2^{-n}\) and \(b := 2^{-m-2} + \cdots + 2^{-n} + 2^{-n}\). Obviously \(a < b\) and further we know that \(\tan^{-1}\) is continuous on \([a,b]\) and differentiable on \((a,b)\). Therefore we can apply the Mean Value Theorem from calculas to find that 

\[\exists c \in (a,b) : \frac{1}{c^2 + 1} = \frac{\tan^{-1}(b) - \tan^{-1}(a)}{b-a}\]

By re-arranging we see that 

\begin{align*}
	\tan^{-1}(b) &= \frac{2^{-m-1}}{c^2 + 1} + \tan^{-1}(a)\\
		&< \frac{2^{-m-1}}{2^{-2m-2} + 1} + \tan^{-1}(a)
\end{align*}

It can be shown, by considering the series expansion of \(\tan^{-1}(2^{-m-1})\), that \(\frac{2^{-m-1}}{2^{-2m-2} + 1} < \tan^{-1}(2^{-m-1}) \forall m \in \Zpz\); therefore we get that:

\[\tan^{-1}(b) < \tan^{-1}(2^{-m-1}) + \tan^{-1}(a)\]

Following this an using the assumed value of \(\gamma_{m+1}\), we see that:

\[\gamma_m < \gamma_{m+1} + \tan^{-1}(2^{-m-2} + \cdots + 2^{-n} + 2^{-n})\]

By repeating the above process we eventually see that:

\[\gamma_m < \sum_{k=m+1}^{n-1} \gamma_k + \tan^{-1}(2^{-n} + 2^{-n})\]

In a similar manner we can repeat the above process with \(a := \tan^{-1}(2^{-n})\) and \(b := \tan^{-1}(2^{-n} + 2^{-n})\). This will show that:

\begin{displaymath}
	\gamma_m < \gamma_n + \sum_{k=m+1}^{n}\gamma_n
\end{displaymath}

\end{proof}

Using the previous proposition we can then show that our \(\gamma_k\) have the property that every angle in \((-\frac{\pi}{2}, \frac{\pi}{2})\) can be approximated by either adding or subtracting successive \(\gamma_k\) to within a tolerance of \(\gamma_n\).

%THM%
\begin{Cordic Accuracy}
\label{THM_"Cordic Accuracy"}
If \(\gamma_k = \tan^{-1}(2^-k) \forall k \in \Z\), then for any \(n \in \N\) 
\[\exists \: (c_k\in \{-1,1\} : k \in [0,n] \cap \Z) \: : \: |\beta - \sum_{k=0}^nc_k\gamma_k| \le \gamma_n \quad \forall \: \beta \in (-\frac{\pi}{2}, \frac{\pi}{2})\]
\end{Cordic Accuracy}
\begin{proof}
We let \(\beta \in (-\frac{\pi}{2}, \frac{\pi}{2})\) and then will proceed by induction on \(n \in \N\).
\begin{description}
\item[\textrm{H\((n)\)}:] 
	\(\exists (c_k \in{-1, 1} : k \in [0, n] \cap \Z) : |\beta - \sum_{k=0}^nc_k\gamma_k| \le \gamma_n\)\\
\item[\textrm{H\((0)\)}:] 
	We have 4 cases to consider:\\
	\begin{description}
	\item[Case \(\beta \in [0, \frac{\pi}{4})\):]
		In this case \(-\frac{\pi}{4} \le \beta - \gamma_0 < 0\)\\
		Therefore \(|\beta - \gamma_0| \le \gamma_0\).
	\item[Case \(\beta \in [\frac{\pi}{4}, \frac{\pi}{2})\):]
		In this case \(0 \le \beta - \gamma_0 < \frac{\pi}{4}\)\\
		Therefore \(|\beta - \gamma_0| \le \gamma_0\).
	\item[Case \(\beta \in (-\frac{\pi}{4}, 0)\):]
		In this case \(0 < \beta + \gamma_0 < \frac{\pi}{4}\)\\
		Therefore \(|\beta - \gamma_0| < \gamma_0\).
	\item[Case \(\beta \in (-\frac{\pi}{2}, -\frac{\pi}{4}]\):]
		In this case \(-\frac{\pi}{4} < \beta - \gamma_0 \le 0\)\\
		Therefore \(|\beta - \gamma_0| < \gamma_0\).
	\end{description}
	Therefore we see that \textrm{H\((0)\)} holds true.
\item[\textrm{H\((n)\) \(\implies\) H\((n+1)\)}:]\hfill\break
	By \textrm{H\((n)\)} \(\exists (c_k \in {-1,1} : k \in [0,n] \cap \Z) : |\beta - \sum_{k=0}^nc_k\gamma_k| \le \gamma_n\); so let \(\beta_n := \beta - \sum_{k=0}^nc_k\gamma_k\).\\
	By Proposition \ref{THM_"Cordic Gamma Property"} we know that \(\gamma_n < 2\gamma_{n+1}\), and so we can proceed by case analysis:
	\begin{description}
	\item[Case \(\beta_n \in [0, \gamma_{n+1})\):]\hfill\break
		\(-\gamma_{n+1} \le \beta_n - \gamma_{n+1} < 0 \implies |\beta - \sum_{k=0}^{n+1}c_k\gamma_k| \le \gamma_{n+1}\) where \(c_{n+1} = -1\).
	\item[Case \(\beta_n \in [\gamma_{n+1}, \gamma_n)\):]\hfill\break
		\(0 \le \beta_n - \gamma_{n+1} < \gamma_{n+1} \implies |\beta - \sum_{k=0}^{n+1}c_k\gamma_k| \le \gamma_{n+1}\) where \(c_{n+1} = -1\).
	\item[Case \(\beta_n \in [-\gamma_{n+1}, 0)\):]\hfill\break
		\(0 \le \beta_n + \gamma_{n+1} < \gamma_{n+1} \implies |\beta - \sum_{k=0}^{n+1}c_k\gamma_k| \le \gamma_{n+1}\) where \(c_{n+1} = 1\).
	\item[Case \(\beta_n \in (-\gamma_n, -\gamma_{n+1})\):]\hfill\break
		\(-\gamma_{n+1} < \beta_n + \gamma_{n+1} < 0 \implies |\beta - \sum_{k=0}^{n+1}c_k\gamma_k| \le \gamma_{n+1}\) where \(c_{n+1} = 1\).
	\end{description}
\end{description}
	Therefore as we have found a suitable \(c_n\) in all cases then we have shown that \textrm{H\((n)\) \(\implies\) H\((n+1)\)}.
\end{proof}

With this proposition we see that our choice for \(\gamma_k\) is a good choice to use for the CORDIC algorithm as it covers the entire range of \((-\frac{\pi}{2}, \frac{\pi}{2})\).\\

Now, as stated before, the basis of our algorithm is to calculate \(\left(\begin{array}{c}\cos\beta\\\sin\beta\end{array}\right)\) by using rotations of a unit vector. By putting our values for \(\gamma_k\) into our rotation matrix we get the following:

\begin{displaymath}
\left(\begin{array}{cc}
	\cos\gamma_k & -\sin\gamma_k\\
	\sin\gamma_k & \cos\gamma_k
	\end{array}\right)
= \frac{1}{\sqrt{1 + 2^{-2k}}}
\left(\begin{array}{cc}
	1 & -2^{-k}\\
	2^{-k} & 1
\end{array}\right)
\end{displaymath}

Then if we take a current estimate of \(\left(\begin{array}{c}\cos\beta\\\sin\beta\end{array}\right)\) at step \(k\) to be \(\left(\begin{array}{c}x_n\\y_n\end{array}\right)\), we see that

\begin{displaymath}
	\left(\begin{array}{cc}
		\cos\gamma_k & - \sin\gamma_k\\
		\sin\gamma_k & \cos\gamma_k
	\end{array}\right)
	\left(\begin{array}{c}
		x_k \\ y_k
	\end{array}\right)
	= \frac{1}{\sqrt{1 + 2^{-2k}}}
	\left(\begin{array}{c}
		x_k - 2^{-k}y_k\\
		y_k + 2^{-k}x_k
	\end{array}\right)
\end{displaymath}

This gives a very simple formula for the update of \(x_k\) and \(y_k\), which can be used as the basis of the CORDIC Algorithm.\\

As seen in our proof of Proposition \ref{THM_"Cordic Accuracy"}, we can approximate our desire angle at step \(n\) by keeping a track of \(\beta_n := \beta - \sum_{k=0}^{n-1}c_k\gamma_k\). At step \(n\) we then have \(\beta_{n+1} = \beta_n - \gamma_n\) if \(\beta_{n+1} \ge 0\), and \(\beta_{n+1} = \beta_n + \gamma_n\) otherwise. This leads us to the general implementation of CORDIC for Trigonometric Functions:

%PCD%
\begin{lstlisting}[numbers=left,frame=single,mathescape,caption={General Cordic},label={PCD_"General_Cordic"}]
  CORIC($\beta \in (-\tfrac{\pi}{2}, \tfrac{\pi}{2}), n \in \N$):
      $x := 1$
      $y := 0$
      $k := 0$
      while $k < n$:
          if $\beta \ge 0$:
              $t := x$
              $x \mapsto \tfrac{1}{\sqrt{1 + 2^{-2k}}}(x - 2^{-k}y)$
              $y \mapsto \tfrac{1}{\sqrt{1 + 2^{-2k}}}(y + 2^{-k}t)$
              $\beta \mapsto \beta - \tan^{-1}(2^{-k})$
          else:
              $t := x$
              $x \mapsto \tfrac{1}{\sqrt{1 + 2^{-2k}}}(x + 2^{-k}y)$
              $y \mapsto \tfrac{1}{\sqrt{1 + 2^{-2k}}}(y - 2^{-k}t)$
              $\beta \mapsto \beta + \tan^{-1}(2^{-k})$
          $k \mapsto k + 1$
      return $(x, y)^T$
\end{lstlisting}

There are few improvements we can make on the general algorithm, however if we start to consider implementaions of the algorithm we can find several ways to make our algorithm more efficient.\\

First we consider the representation of our values in the program, and while in many of the previous algorithms a floating point \codeinline{double} value, as described in Section \ref{#SEC#}, we will see here that we wish to use a fixed point representation. If we have a fixed point representation of our values, then we are using an \(N\) bit integer to represent the value in question, with a fixed number of bits set aside for the integer part and the remainder for the fractional part. In this case the process of addition, subtraction as well as multiplication and division by powers of 2 is the same as that for integers.\\

In particular as our values never exceed the range of \((-2,2)\), then we can use \(N-2\) bits of our \(N\) bit integer to be the fractional part; this gives us a maximum precision of \(2^{2-N}\). Further as we are only performing multiplication and division by two, this operation can be performed by bitshifting the values, which is much quicker than actual integer multiplication.\\

Second we can precalculate all of the values needed for the algorithm to trade storage space for a reduction in computational complexity. The values which we need to pre-calculate are \(\gamma_k = \tan^{-1}(2^{-k})\) and \(\tfrac{1}{\sqrt{1+2^{-2k}}}\) for \(k \in [0, n) \cap \Z\). The first thing to note about this is that instead of calculating the multiplication \(\tfrac{1}{\sqrt{1+2^{-2k}}}\) at each stage we can actually take this value out of the loops and pre-calculate \(\prod_{k=0}^n \tfrac{1}{\sqrt{1+2^{-2k}}}\) for \(k \in [0, n) \cap \Z\). Using these precalculated products we can then replace \(x := 1\) with \(x := \prod_{k=0}^n \tfrac{1}{\sqrt{1+2^{-2k}}}\) in the initialisation stage.\\

Now to consider an actual implementation, suppose we are using the 16 bit integer \codeinline{int16\_t} to represent our values; which will have the leading two bits represent the integer part and the remaining 14 bits represent the fractional part. In this case the level of precision is \(2^{-14} = 0.00006103515625\) and futher we can show that as \(\gamma_{14} = \tan^{-1}(2^{-14}) \approx 2^{-14}\); therefore the largest we will choose \(n := 14\) to ensure the maximum possible accuracy, without performing excessive calculations\\

This means we can simplify our algorithm futher by calculating only \(\prod_{k=0}^{14} \frac{1}{\sqrt{1 + 2^{-2k}}}\) and \(\tan^{-1}(2^{-k}) \forall k \in [0,14] \cap \Z\). One futher note is that these values then need to be converted to approximations in our 16 bit fixed point representation. The first value is: 

\begin{align*}
	\prod_{k=0}^{14} \frac{1}{\sqrt{1 + 2^{-2k}}} &= 
		0.60725293651701023412897124207973889082\ldots\\
		&\approx \textrm{00.10011011011101}_2\\
		&= \textrm{26dd}_{16}
\end{align*}

Below is a table of all the angles in the relevant formats

%TBL%
\begin{center}
\begin{tabular}{|c|c|c|c|}
	\hline
	\(\gamma_k\) & Exact Form & Binary & Hexadecimal \\\hline
	\(\gamma_0\) & \(0.7853981633\ldots\)
		& \(\textrm{00.11001001000011}_2\)
		& \(\textrm{3243}_{16}\\\hline
	\(\gamma_1\) & \(0.4636476090\ldots\)
		& \(\textrm{00.01110110101100}_2\)
		& \(\textrm{1dac}_{16}\)\\\hline
	\(\gamma_2\) & \(0.2449786631\ldots\)
		& \(\textrm{00.00111110101101}_2\)
		& \(\textrm{0fad}_{16}\)\\\hline
	\(\gamma_3\) & \(0.1243549945\ldots\)
		& \(\textrm{00.00011111110101}_2\)
		& \(\textrm{07f5}_{16}\)\\\hline
	\(\gamma_4\) & \(0.0624188099\ldots\)
		& \(\textrm{00.00001111111110}_2\)
		& \(\textrm{03fe}_{16}\)\\\hline
	\(\gamma_5\) & \(0.0312398334\ldots\)
		& \(\textrm{00.00000111111111}_2\)
		& \(\textrm{01ff}_{16}\)\\\hline
	\(\gamma_6\) & \(0.0156237286\ldots\)
		& \(\textrm{00.00000100000000}_2\)
		& \(\textrm{0100}_{16}\)\\\hline
	\(\gamma_7\) & \(0.0078123410\ldots\)
		& \(\textrm{00.00000010000000}_2\)
		& \(\textrm{0080}_{16}\)\\\hline
	\(\gamma_8\) & \(0.0039062301\ldots\)
		& \(\textrm{00.00000001000000}_2\)
		& \(\textrm{0040}_{16}\)\\\hline
	\(\gamma_9\) & \(0.0019531225\ldots\)
		& \(\textrm{00.00000000100000}_2\)
		& \(\textrm{0020}_{16}\)\\\hline
	\(\gamma_{10}\) & \(0.0009765621\ldots\)
		& \(\textrm{00.00000000010000}_2\)
		& \(\textrm{0010}_{16}\)\\\hline
	\(\gamma_{11}\) & \(0.0004882812\ldots\)
		& \(\textrm{00.00000000001000}_2\)
		& \(\textrm{0008}_{16}\)\\\hline
	\(\gamma_{12}\) & \(0.0002441406\ldots\)
		& \(\textrm{00.00000000000100}_2\)
		& \(\textrm{0004}_{16}\)\\\hline
	\(\gamma_{13}\) & \(0.0001220703\ldots\)
		& \(\textrm{00.00000000000010}_2\)
		& \(\textrm{0002}_{16}\)\\\hline
	\(\gamma_{14}\) & \(0.0000610351\ldots\)
		& \(\textrm{00.00000000000001}_2\)
		& \(\textrm{0001}_{16}\)\\\hline
\end{tabular}
\end{center}

This allows us to then write the following method in C to calculate both \(\cos\beta\) and \(\sin\beta\), provided \(\beta \in [-\tfrac{\pi}{2}, \tfrac{\pi}{2}]\) is given in 16 bit fixed point representation:

%PCD%
\begin{codelisting}{16 bit Fixed Point CORDIC algorithm}
int16_t *cordic_16(int16_t beta)
{
	const int16_t GAMMA = {0x3243, 0x1dac, 0x0fad, 0x07f5, 0x03fe,
			       0x01ff, 0x0100, 0x0080, 0x0040, 0x0020,
			       0x0010, 0x0008, 0x0004, 0x0002, 0x0001};

	int16_t x = 0x26dd, y = 0x0000, t, result;

	for(int k = 0; k <= 14; ++k)
	{
		t = x;
		if(beta >= 0)
		{
			beta -= GAMMA[k];
			x = x - (y >> k);
			y = y + (t >> k);
		}
		else
		{
			beta += GAMMA[k];
			x = x + (y >> k);
			y = y - (t >> k);
		}
	}

	//This line is required by C to allow the value to be returned
	result = malloc(2 * sizeof(int16_t));
	
	result[0] = x;
	result[1] = y;
	return result;
}
\end{lstlisting}
\end{codelisting}

As can easily be seen in the algorithm the number of calculations each iteration is constant, and the number of iterations is fixed at 15. This means that the algorithm is an \(\bigO(1)\) algorithm, and guarantees an answer accurate to 4 decimal places as \(2^{-14} < 10^{-4}\). Further as the only calculations are integer addition, subtraction and bitshifting this method executes extremely quickly.\\

Similar methods exist for other fixed length formats such as using \codeinline{int32\_t} or \codeinline{int64\_t}. To examine in more detail how the method converges we will consider an implementation using \codeinline{int64\_t}, which will be approximating \(\cos(0.5)\). The code used is included in the Appendix \ref{#APP#} and can perform the calculations with \(n \le 63\). Below are some of the functions approximations for different values of \(n\):

{\fontfamily{pcr}\selectfont
%TBL%
\begin{center}
\begin{tabular}{|c|c|}
	\hline
	\(n\) & \textsf{Output with bold accurate digits}\\\hline
	1 & \textbf{0.}70710678118654757273731\\\hline
	2 & \textbf{0.}94868329805051376801827\\\hline
	3 & \textbf{0.8}4366148773210747346951\\\hline
	4 & \textbf{0.}90373783889353875853345\\\hline
	5 & \textbf{0.87}527458786899225984257\\\hline
	6 & \textbf{0.8}8995346811933362385360\\\hline
	\cdots & \cdots\\\hline
	19 & \textbf{0.87758}301847694786257392\\\hline
	20 & \textbf{0.877582}10404530012649360\\\hline
	21 & \textbf{0.877582561}26152311971111\\\hline
	22 & \textbf{0.877582}78986933524468128\\\hline
	\cdots & \cdots\\\hline
	53 & \textbf{0.8775825618903727}5873943\\\hline
	54 & \textbf{0.877582561890372}64771712\\\hline
	55 & \textbf{0.8775825618903727}5873943\\\hline
	56 & \textbf{0.8775825618903727}5873943\\\hline
	\cdots & \cdots\\\hline
	63 & \textbf{0.8775825618903727}5873943\\\hline
\end{tabular}
\end{center}}

This table shows us several interesting features of the algorithm, the first being that while there are points at which a certain number of decimal places are guaranteed; before that point the number of decimal places of accuracy can vary, such as in the first few iterations. As we know that the error after \(n\) iterations is at most \(\gamma_n = \tan^{-1}(2^{-n})\), then we can guarantee that we have at least \(d\) decimal places of accuracy if we use as least \(\log_2(\cot(10^{-d}))\) iterations.\\

Second there are some values of \(n\) which have uncharacteristically close approximations of the actual value, such as the case when 21 iterations are used. This arises due to the algorithm finding a good approximation for \(\beta\), but then successive numbers of iterations move away from this value, thus once more decreasing the number of decimal digits of accuracy.\\

Finally at the end of the able we see that from 55 iterations onwards, the results do not get any more accurate. It turns out this is due to the program converting the \codeinline{int64\_t} fixed point values into \codeinline{double} values, which typically have an precision of around \(2^{-55}\). If we instead modify the program to use a more precise floating point representation we see that the 53 to 56 section of the table becomes:

{\fontfamily{pcr}\selectfont
%TBL%
\begin{center}
\begin{tabular}{|c|c|}
	\hline
	\(n\) & \textsf{Output with bold accurate digits}\\\hline
	53 & \textbf{0.8775825618903727}3965747\\\hline
	54 & \textbf{0.877582561890372}68653156\\\hline
	55 & \textbf{0.87758256189037271}298609\\\hline
	56 & \textbf{0.8775825618903727}2621336\\\hline
\end{tabular}
\end{center}}

This is much more inline with what we would expect to see from the known error of the algorithm.\\

Now another use of CORIC is to effectively run it in reverse, which will allow us to calculate the Inverse Trigonometric functions. To do this we will start by considering the method for calculating \(\tan^{-1}\), and then use trigonometric identities to calculate both \(\cos^{-1}\) and \(\sin^{-1}\).\\

To accomplish this we will be fixing some initial values for \(\sin\theta\) and \(\cos\theta\), and then running the CORDIC algorithm to move the approximation of \(\sin\theta\) towards zero. In doing this we will effectively run our algorithm in reverse, and if we keep track of the angles that we rotate through we can find \(\tan^{-1}\).\\

We know that \(\tan\theta = \frac{\sin\theta}{\cos\theta}\), which means that if we have a current approximation \(\left(\begin{array}{c}x_k\\y_k\end{array}\right)\) then \(\frac{y_k}{x_k} \approx \tan\theta\). Using this, if we have an input of \(\tan\theta = z\) then we can take our initial values to be \(x_0 := \tfrac{1}{2}\) and \(y_0 := \tfrac{z}{2}\). This has the desired property that \(\tfrac{y_0}{x_0} = z\), and if we have \(y_n\) tending to 0 then the angle we approximate in the process will be \(\theta\). \\

If we again consider a 16 bit fixed point implementation for our algorithm we can implement it as follows:

%PCD%
\begin{codelisting}{16 bit Fixed Point CORDIC \(\tan^{-1}\)}
int16_t *cordic_atan_16(int16_t z)
{
	const int16_t GAMMA = {0x3243, 0x1dac, 0x0fad, 0x07f5, 0x03fe,
			       0x01ff, 0x0100, 0x0080, 0x0040, 0x0020,
			       0x0010, 0x0008, 0x0004, 0x0002, 0x0001};

	int16_t x = 0x2000, y = z >> 1, t, theta;

	for(int k = 0; k <= 14; ++k)
	{
		t = x;
		if(y < 0)
		{
			theta -= GAMMA[k];
			x = x - (y >> k);
			y = y + (t >> k);
		}
		else
		{
			theta += GAMMA[k];
			x = x + (y >> k);
			y = y - (t >> k);
		}
	}

	return theta;
}
\end{lstlisting}
\end{codelisting}

Similar to our considerations when dealing with the taylor method of calculating \(\tan^{-1}\), we need to ensure that the input value is not too large, and so can perform the same transformations to the value to ensure we are always calculating a value in the range \([0,1)\). Using this we can then use the identities \(\sin^{-1}(z) = \tan^{-1}(\frac{z}{\sqrt{1-z^2}})\) and \(\cos^{-1}(z) = \tan^{-1}(\frac{\sqrt{1 - z^2}}{z})\). \\

Obviously there are basic exceptional values that need to be checked for, in particular \(\cos^{-1}(0) = \tfrac{\pi}{2}\), and \(\sin^{-1}(\pm1) = \pm\tfrac{\pi}{2}\). If these values are checked before hand then we are never dividing by 0, \(z \in [-1,1]\cap\Z\), and thus we have a complete algorithm, that calculates the inverse Trigonometric Functions.\\

This method, like the CORDIC method for the regular Trigonometric Functions, has an approximation that is acccurate to withing \(\gamma_n\). Thus for our 16 bit implementation, the output will be accurate to within an error of \(2^{-14} = 0.00006103515625\), in particular guaranteeing at least 4 decimal places of accuracy. A final note is that the Inverse Trigonometric Functions, again much like the regular CORDIC algorithm, is an \(\bigO(1)\) algorithm with simple calculations, making the algorithm extremely efficient.

%SUB%
\subsection{Comparrison of Methods}

We have observed three different methods for calculating the Trigonometric Functions, as well as their inverses and so should compare their efficiency and accuracy properties.\\

First we will compare how quickly each algorithm approaches the correct value for different inputs of \(n\), and using \(\theta = 0.5\). The comparrison will use \codeinline{double} values for computation, so that all three methods may be equally compared. The table below compares the convergence of \(\cos\theta\), with the bold digits being the correct digits found:

{\fontfamily{pcr}\selectfont
%TBL%
\begin{center}
\begin{tabular}{|c|l|l|l|}
\hline
\(n\) & \textrm{geometric\_Cos(0.5, \(n\))}
	  & \textrm{taylor\_Cos(0.5, \(n\))}
	  & \textrm{CORDIC(0.5, \(n\))}\\\hline
1 & \textbf{0.87}6953125000000000
  & \textbf{1}.000000000000000000
  & \textbf{0.}707106781186547572\\\hline
2 & \textbf{0.877}426177263259887
  & \textbf{0.877}604166666666629
  & \textbf{0.}948683298050513768\\\hline
3 & \textbf{0.8775}43526076081437
  & \textbf{0.877}604166666666629
  & \textbf{0.8}43661487732107473\\\hline
4 & \textbf{0.8775}72806699400187
  & \textbf{0.87758256}2158978118
  & \textbf{0.}903737838893538758\\\hline
5 & \textbf{0.87758}0123327654892
  & \textbf{0.87758256}2158978118
  & \textbf{0.87}5274587868992259\\\hline
6 & \textbf{0.87758}1952264380182
  & \textbf{0.87758256189037}3424
  & \textbf{0.8}89953468119333623\\\hline
7 & \textbf{0.877582}409484792491
  & \textbf{0.87758256189037}3424
  & \textbf{0.8}82719918613777410\\\hline
8 & \textbf{0.8775825}23789035007
  & \textbf{0.8775825618903727}58
  & \textbf{0.87}9022003513595939\\\hline
9 & \textbf{0.8775825}52365041901
  & \textbf{0.8775825618903727}58
  & \textbf{0.877}152884812089639\\\hline
10& \textbf{0.8775825}59509040183
  & \textbf{0.8775825618903727}58
  & \textbf{0.87}8089122532394572\\\hline
\end{tabular}
\end{center}}

This table demonstrates that \textrm{taylor\_Cos} has the fastest convergence, and also demonstrates the staggered increase in accuracy as each step of the algorithm calculates two updates to \(\cos\theta\), and thus the output only gets more accurate every other value of \(n\). The \textrm{geometric\_Cos} method has the second best convergence, while the CORDIC algorithm lags behind, having inconsistent convergence as measured in correct digits.\\

Next we will note that all algorithms can guarantee 10 digits of accuracy in a fixed number of steps. In particular we can guarantee 10 digits of accuracy for \textrm{geometric\_Cos} when \(n \ge 16\), \textrm{taylor\_Cos} when \(n \ge 8\) and \textrm{CORDIC} when \(n \ge 34\). Using the lower bounds of each of these values for \(n\) we can directly compare the speed of the methods.\\

To compare the methods we will be testing 1000 random values in the range \([0, \frac{\pi}{2})\) for which we will calculate the cosine of with each method 100000 times. This will then also be compared to the standard C implementation of the \(\cos\) function, available in \codeinline{math.h}. The results of my personal testing follow, where the given times are for individual values, not individual method execution times:

{\fontfamily{pcr}\selectfont
%TBL%
\begin{center}
\begin{tabular}{|l|r|r|r|r|}
\hline
	& \codeinline{geometric\_cos} & \codeinline{taylor\_cos}
	& \codeinline{cordic\_cos} & \codeinline{builtin\_cos}\\\hline 
	\textsf{Total time:} & 16.029s & 7.937s & 21.471s & 0.243s\\\hline
	\textsf{Average time:} & 0.016s & 0.007s & 0.021s & 0.000s\\\hline
	\textsf{Minimum time:} & 0.015s & 0.007s & 0.020s & 0.000s\\\hline
	\textsf{Maximum time:} & 0.022s & 0.013s & 0.030s & 0.000s\\\hline
\end{tabular}
\end{center}
}

These values show that the fastest algorithm that we have discussed is Algorithm \ref{#ALG#} (\textrm{taylor\_Cos}), while the slowest is the CORDIC algorithm. However all of our Algorithms are much less efficient than the built-in \codeinline{cos} function of C. It turns out this discrepency is due to inefficient implementation as the \codeinline{cos} function also uses a Taylor approximation, but is implemented in a much lower-level method that optimises the execution of the code.\\

\TODO{Ref the C code}\\
\TODO{https://sourceware.org/git/?p=glibc.git;a=blob;f=sysdeps/ieee754/dbl-64/s\_sin.c;hb=HEAD#l281}\\

Next we will compare our methods for the Inverse Trigonometric Functions, starting with how they converge to the correct value, as detailed in the following table:

{\fontfamily{pcr}\selectfont
%TBL%
\begin{center}
\begin{tabular}{|c|l|l|l|}
\hline
\(n\) & \textrm{geometric\_aCos(0.5, \(n\))}
	  & \textrm{taylor\_aCos(0.5, \(n\))}
	  & \textrm{CORDIC(0.5, \(n\))}\\\hline
1 & \textbf{2.3}51425307918200591
&\textbf{2.}270796326794896735
&\textbf{2.3}56194490192344837\\\hline
2 & \textbf{2.34}7503635391542609
&\textbf{2.}327962993461563101
&\textbf{1}.892546881191538687\\\hline
3 & \textbf{2.346}521397812842746
&\textbf{2.}340568243461563113
&\textbf{2.}137525544318402914\\\hline
4 & \textbf{2.346}275724597314926
&\textbf{2.34}4244774711563117
&\textbf{2.}261880538865164602\\\hline
5 & \textbf{2.346}214299177873829
&\textbf{2.34}5470795757570225
&\textbf{2.3}24299348861121661\\\hline
6 & \textbf{2.34619}8942378459939
&\textbf{2.34}5913166442261221
&\textbf{2.3}55539182291389810\\\hline
7 & \textbf{2.34619}5103149716576
&\textbf{2.346}081295659538934
&\textbf{2.3}39915453670913247\\\hline
8 & \textbf{2.34619}4143336564508
&\textbf{2.3461}47594614218956
&\textbf{2.34}7727794731014228\\\hline
9 & \textbf{2.346193}903386887935
&\textbf{2.3461}74467628018511
&\textbf{2.34}3821564599047224\\\hline
10& \textbf{2.3461938}43452078375
&\textbf{2.3461}85594784405026
&\textbf{2.34}5774687115525836\\\hline

\end{tabular}
\end{center}}

Here we see for the inverse trigonometric functions the convergence speed has been altered with the Geometric method now having the fastest convergence, the Taylor Method converges much slower and the CORDIC method is more stable. One interesting behaviour that emerges for larger values of \(n\) in the \textrm{geometric\_aCos} is demonstrated in the following table:

{\fontfamily{pcr}\selectfont
%TBL%
\begin{center}
\begin{tabular}{|c|l|}
\hline
\(n\) & \textrm{geometric\_aCos(0.5, \(n\))}\\\hline
13 & \textbf{2.34619382}2083380897\\\hline
14 & \textbf{2.3461938}12716280469\\\hline
15 & \textbf{2.346193}737779483257\\\hline
\cdots & \cdots\\\hline
22 & \textbf{2.346}097524754926944\\\hline
23 & \textbf{2.34}1202123910687049\\\hline
24 & \textbf{2.3}51023238547698124\\\hline
\end{tabular}
\end{center}}

This behaviour arrises due to the use of \codeinline{double} to calculate values of very small magnitude, this causes the value to become effectively 0 and thus lead to the innacuracies seen. If we use a higher precision representation for the calculations we get the following table instead:

{\fontfamily{pcr}\selectfont
%TBL%
\begin{center}
\begin{tabular}{|c|l|}
\hline
\(n\) & \textrm{geometric\_aCos(0.5, \(n\))}\\\hline
13 & \textbf{2.346193823}718087586\\\hline
14 & \textbf{2.3461938234}83759158\\\hline
15 & \textbf{2.3461938234}25177051\\\hline
\cdots & \cdots\\\hline
22 & \textbf{2.3461938234056}50874\\\hline
23 & \textbf{2.346193823405649}980\\\hline
24 & \textbf{2.346193823405649}757\\\hline
\end{tabular}
\end{center}}

With this we see that Algorithm \ref{#ALG#} continues in the same pattern as before and is actually correct. So we may again look to time our functions to test their efficiency as compared to each other. To do this we will again use 1000 random values, this time in the range \((-1,1)\), each of which we will calculate \(\cos^{-1}\) using each method 100000 times. We note that the algorithms can guarantee 10 decimal places of accuracy for different values of \(n\), in particular \textrm{geometric\_aCos} when \(n \ge 18\), \textrm{taylor\_aCos} when \(n \ge 30\) and \textrm{CORDIC} when \(n \ge 50\).

{\fontfamily{pcr}\selectfont
%TBL%
\begin{center}
\begin{tabular}{|l|r|r|r|r|}
\hline
	& \codeinline{geometric\_cos} & \codeinline{taylor\_cos}
	& \codeinline{cordic\_cos} & \codeinline{builtin\_cos}\\\hline 
	\textsf{Total time:} & 27.273s & 14.358s & 29.142s & 2.143s\\\hline
	\textsf{Average time:} & 0.027s & 0.014s & 0.029s & 0.002s\\\hline
	\textsf{Minimum time:} & 0.026s & 0.014s & 0.028s & 0.001s\\\hline
	\textsf{Maximum time:} & 0.033s & 0.018s & 0.032s & 0.006s\\\hline
\end{tabular}
\end{center}}

Again this table shows that the Taylor method is the quickest of those analysed and teh CORDIC method is the slowest, however they also both are much slower than the built in methods. One thing to note is that the inverse trigonometric functions are simply less efficient to calculate, as can be seen in the execution time of the built-in method, which appears to be two orders of magnitude greater than the corresponting trigonometric method.\\

We conclude that for most implementations the Taylor method is the most appropriate method to use to ensure a high accuracy quickly. However the CORDIC algorithm is of use when more advanced features such as floating point type values, or hardware multipliers are not present; further it is possible to create hardware implementations of the CORDIC algorithm which can even further speed up the calculations. 

%SEC%
\section{Root Functions}
\label{SEC_"Root Functions"}

In this section of the document we will consider several methods for approximating root functions. For our purposes we are only going to consider roots of \(N \in \Rpz\), this is because if \(N \in \R^-\) then it follows that \(\sqrt{N} = i\sqrt{|N|}\).

%SUB%
\subsection{Digit by Digit Method}
\label{SUB_"Digit by Digit Method"}

The first method we will examine is an old method, that has been observed in Babylonian Mathematics over 2000 years ago, which is used to accurately generate the square root of numbers one digit at a time. This method differs from others discussed as it generates each digit of the root with perfect accuarcy, one at a time, thus in a theoretical sense this algorithm is the most accurate of the methods we will view; we will see however that this method is slow.\\

Now suppose we are looking for \(\sqrt{N}\), then we know that \(\sqrt{N} = a_010^n + a_110^{n-1} + a_210^{n-2} + \dots\) for some \(n \in \Z\); it then follows that \(N = (a_010^n + a_110^{n-1} + a_210^{n-1} + \dots)^2\). By expanding the quadratic value we get that \[N = a_0^210^{2n} + (20a_0 + a_1)a_110^{2n-2} + (20(a_010 + a_1) + a_2)a_210^{2n-4} + \dots + (20\sum_{i=0}^{k-1}a_i10^{k-i-1} + a_k)a_k10^{2n - 2k}\]

An observation should be made regarding the value of \(n\) that we use for the theorem. We could of course try different values of \(n\), in some structured procedure, that will find the largest \(n\) such that \(10^n \le N\). However we can note that \(log_{10}(\sqrt{N}) = \tfrac{1}{2}log_{10}(N)\), thus \(10^{\frac{1}{2}log_{10}(N)} = \sqrt{N}\). Using this information, and the fact that \(n \in \Z\), we can have \(n := \left\lfloor \tfrac{1}{2}log_{10}(N) \right\rfloor\). \\

This allows us to get successive apporximations of \(N\) where \(N_0 = a_0^210^{2n}\), \(N_1 = N_0 + (20a_0 + a_1)a_110^{2n-2}\), \(N_2 = N_1 + (20(a_010 + a_1) + a_2)a_210^{2n-4}\). This will alllow us to create an algorithm that will give successive approximations of \(sqrt{N} = a_010^n + a_110^{n-1} + \dots\), more importantly each approximation will give us the exact next digit in the decimal representation of \(\sqrt{N}\).\\

Thus we can have an iterative method to solve the problem, where at each stage we are trying to find the largest digit which satisfies the inequality \((20\sum_{i=0}^{k-1}a_i10^{k-i-1} + a_k)a_k10^{2n-2k} \le N - N_{k-1}\). Thus we get the following pseudocode, which outputs two sequences, one indicating the digits before the decimal point and one afterwards. I will use set notation to indicate the sequences, but in this case order is important and repetition is allowed.

%PCD%
\begin{lstlisting}[numbers=left,frame=single,mathescape,caption={Exact Digit by Digits Square Root}]
  exactRootDigits($N \in \Rpz, d \in \N$):
      $Digits_a := \emptyset$
      $Digits_b := \emptyset$
      $k := 0$
      $n := \left\lfloor\tfrac{1}{2}log_{10}(N)\right\rfloor$
      while $k < d$:
          $a_k := \max\left\{t \in [0, 9] \cap \Z : \left(20\sum_{i=0}^{k-1}a_i10^{k-i-1} + t\right)t10^{2n-2k} \le N\right\}$
          $N \mapsto N - \left(20\sum_{i=0}^{k-1}a_i10^{k-i-1} + a_k\right)a_k10^{2n-2k}$
          if $n-k < 0$:
              $Digits_b \mapsto Digits_b \cup \{a_k\}$
          else:
              $Digits_a \mapsto Digits_a \cup \{a_k\}$
          $k \mapsto k+1$
      if $Digits_a = \emptyset$:
          $Digits_a := \{0\}$
      if $Digits_b = \emptyset$:
          $Digits_b := \{0\}$
      return $(Digits_a, Digits_b)$
\end{lstlisting}

This method has a computational complexity of \(\bigO(d^2)\), as each loop requires the operations of summing \(k\) elements, and the loop is repeated for \(k = 0 \to d\). We will see that by considering some changes to the algorithm we can change the complexity class to be \(\bigO(d)\).\\

First we will note that line 5 is not an issue, as if we only care about the first significant digit of \(\tfrac{1}{2}log_{10}(N)\), then this is \(\bigO(|log(N)|)\). This can be seen as if we start from \(n = 0\) we can either count up or down until a we find \(10^{2n}\) at most or at least N, respectively. This obviously takes at most \(|log_{10}(N)|\) steps, giving us our stated complexity. We will also assume that \(\bigO(|log(N)|) \le \bigO(d)\), as we have already seen that we can manipulate our input N to be within a reasonable range.

Second we note that on line 7 we calculate \(\sum_{i=0}^{k-1}a_i10^{k-i-1}\) for each value of \(t\); we can reduce the complexity of this line by pre-calculating this value. However we can do even better if we consider that at step \(k+1\) we are calculating \(\sum_{i=0}^{k}a_i10^{k-i} = a_k + 10\sum_{i=0}^{k-1}a_i10^{k-i-1}\). Thus if we introduce \(P_0 := 0\), and fore each k we calculate \(P_{k+1} := 10P_k + a_k\), then we can reduce the complexity from \(\bigO(k)\) to \(\bigO(1)\).\\

This calculation of \(P_k\), then carries over to reduce the complexity of line 8 to be \(\bigO(1)\) instead of \(\bigO(k)\). Combining this we can create the modified algorithm below:

%PCD%
\begin{lstlisting}[numbers=left,frame=single,mathescape,caption={Exact Digit by Digits Square Root version 2}]
  exactRootDigits_v2($N \in \Rpz, d \in \N$):
      $Digits_a := \emptyset$
      $Digits_b := \emptyset$
      $k := 0$
      $n := \left\lfloor\tfrac{1}{2}log_{10}(N)\right\rfloor$
      $P_0 := 0$
      while $k < d$:
          $a_k := \max\left\{t \in [0, 9] \cap \Z : \left(20P_k + t\right)t10^{2n-2k} \le N\right\}$
          $N \mapsto N - \left(20P_k + a_k\right)a_k10^{2n-2k}$
          $P_{k+1} := 10P_k + a_k$
          if $n-k < 0$:
              $Digits_b \mapsto Digits_b \cup \{a_k\}$
          else:
              $Digits_a \mapsto Digits_a \cup \{a_k\}$
          $k \mapsto k+1$
      if $Digits_a = \emptyset$:
          $Digits_a := \{0\}$
      if $Digits_b = \emptyset$:
          $Digits_b := \{0\}$
      return $(Digits_a, Digits_b)$
\end{lstlisting}

This method is usefull, but can be difficult to implement as it requires high precision for the representation of the real value of \(N\). In my implementation using C, I utilised the MPFR library to utilise high precision integers, but still encountered issues regarding loss of precision.\\

As an example the table below shows the number of digits of accuracy I was able to calculate for \(\sqrt{2}\) using the above algorithm, compared to the number of bits of precision used in the calculations.\\

\begin{center}
\begin{tabular}{|p{3cm}|p{3cm}|}
\hline
Bits of Precision & Maximum Accuracy\\ \hline
8 & 2 \\ \hline
16 & 5 \\ \hline
32 & 9 \\ \hline
64 & 18 \\ \hline
128 & 39 \\ \hline
256 & 77 \\ \hline
512 & 154 \\ \hline
1024 & 308 \\ \hline
2048 & 615 \\ \hline
4096 & 1234 \\ \hline
8192 & 2466 \\ \hline
\end{tabular}
\end{center}

This data is highly structured and so we can hope to create a simple function that would allow us to calculate how much precision would be needed for a given number of digits of accuracy, at least for single digit inputs for \(N\). We can see that the average ratio of Precision to Accuracy is 3.41259..., which ranges from 3.31928... to 4.0. From this we can draw a general trend that Digits of Accuracy \(\approx\) 3.4 \(\times\) Bits of Precision; thus if we take the more generous assumption that Digits of Accuracy \(\appprox\) 4 \(\times\) Bits of Precision, we can use this to pre-determine the accuracy needed.\\

It should be noted that to ensure accuracy we should over-estimate the required precision, however if we overestimate the precision, then our calculations will be performed using unnecsarily large data structures and thus computation time will increase.\\

One particular use of this technique is to find an approximation of a squar root to it's integer part, calculated in base 2. This algorithm is of note as we will see that it has a computation time of \(\bigo(1)\).\\

The algorithm uses the same basis as the base 10 version, for it's calculations, but due to the nature of being in binary several changes can be made for computational efficiency. To do this we will view the problem as follows: if we know some \(r \in \Zpz\) which is our current approximation of our root, we are looking for some \(e \in \Zpz\) such that \((r+e)^2 \le N\). Expanding this out we get \(r^2 + 2re + e^2 \le N\), and if we keep track of \(M = N - r^2\), we can test if \(2re + e^2 \le M\).\\

Now we can consider our choice of \(e\), the most practical method is to test successeive \(e_m := 2^m\), where \(m\) is descending starting with \(m = \max{m \in \Zpz : 4^m \le N}\). We can use an iterative formula to build up the integer square root, where we start with \(r = 0, M = N\) and have \(r \mapto r + e_m\) whenever \(2re_m + e_m^2 \le M\), stopping when \(m < 0\). This is then implemented as follows:\\

%PCD%
\begin{lstlisting}[numbers=left,frame=single,mathescape,caption={Integer Square Root Algorithm}]
  integerSquareRoot($N \in \Zpz$):
      $M := N$
      $m := \max{m \in \Zpz : 4^m \le M}$
      $r := 0$
      while $m \ge 0$:
          if $2r(2^m) + 4^m \le M$:
              $M \mapsto M - 2r(2^m) + 4^m$
              $r \mapsto r + 2^m$
          $m \mapsto m - 1$
      return $r$
\end{lstlisting}

If we now conisder an implementation of the above algorithm using an unsigned integer system with \(K\) bits, where \(2 | K\). We will use \codeinline{res} to represent \(2re_m\), which means at the start of the algorihtm we will have \codeinline{res = 0}; similarly we can use \codeinline{bit} to represent \(e_m^2\). As we know that \(K\) bits are used and \(2 | K\), it then follows that the largest power of 4 less than the maximum representable value (\(2^K - 1\) is \(2^{K-2}\), which can be calculated as \codeinline{bit = 1 << (K - 2)} using bitshift operations. Finally we will use \codeinline{num} to represent \(M\).\\

Now that we have discussed the setup we can consider how to implement some of the steps above. First to implement line 3 we can simply keep dividing \codeinline{bit} by 4 while \codeinline{bit > num}, which can be efficiently implemented as \codeinline{bit >> 2} by using bitshifts in place of division by powers of 2. The same technique can be used in place of line 9, which leads us to re-evaluating our usage of line 5. As we are using bitshifting and a bitshift that would take a number past 0 instead results in 0, we also know that \(2 | K\) and so eventually we will reach \codeinline{bit == 1}, which represents \(m = 0\); therefore we can use \codeinline{bit > 0} as our stopping criteria on line 5.\\

Line 6 is easy to convert, given our definitions of \codeinline{res}, \codeinline{bit} and \{num}, as is line 7. All that remains is to consider how to update \codeinline{res}, which has two different ways of being updated depending on whether \codeinline{res + bit <= num}. If it is false that \codeinline{res + bit <= num}, then we wish for \codeinline{res} to represent \(2re_{m-1}\); this is easily acheived if we consider that \(2re_{m-1} = \frac{1}{2}(2re_m)\), which prompts the update \codeinline{res = res >> 1}. For the second case, when \codeinline{res + bit <= num} is true, we want \codeinline{res} to represent \(2(r+e_m)e_{m-1}\); to implement this we consider the following derivation:

\begin{displaymath}
\begin{align*}
	2(r+e_m)e_{m-1} 
		&= \frac{1}{2}\cdot 2(r+e_m)e_m\\
		&= \frac{1}{2}\cdot 2(re_m + e_m^2)\\
		&= \frac{1}{2}(2re_m) + e_m^2
\end{align*}
\end{displaymath}

Using this above derivation we see that we can calculate this as \codeinline{res = (res >> 1) + bit}. Below is a simple implementaion of this in C using the unsigned 32 bit integer type \codeinline{uint32\_t}. A more commented and slightly modified version can be found in Appendix \ref{#APP#}, File \ref{#FILE#}.

\begin{codelisting}{Integer Square Root in C}
uint32_t int_sqrt(uint32_t num)
{
	uint32_t res = 0, bit = (1 << 30);
	
	while (bit > num)
		bit = bit >> 2;
	
	while (bit > 0)
	{
		if (res + bit <= num)
		{
			num = num - (res + bit);
			res = (res >> 1) + bit;
		}
		else
			res = res >> 1;
		
		bit = bit >> 2;
	}

	return res;
}
\end{lstlisting}
\end{codelisting}

We should consider the final step of the loop, when \codeinline{bit == 1}. In this case when \codeinline{res} is updated we have \codeinline{res} represent either \(2(r+e_0)e_{-1} = r + e_0\), or \(2re_{-1} = r\); thus the algorithm exits with the correct value.\\

Now that the algorithm is correctly constructed using simple unsigned integer addition, subtraction and bitshifting (which we can assume all have computational time of \(\bigO{1}\)), we can look at the worst case complexity of the algorithm:

\begin{itemize}
\item The complexity of the set up of variables is constant time.
\item The worst case complexity would be to to have \codeinline{bit <= num} at the start.
\item The loop would execute 16 times for our 32 bit integers, and contains a single operation which is \(\bigO(1)\) complexity.
\begin{itemize}
	\item The worst case within the loop is to have \codeinline{res + bit <= num} for each iteration.
	\item Within the first \codeinline{if} branch there are a constant 4 operations.
	\item Each loop has an additional operation operation to update \codeinline{bit}.
	\item This makes 5 operations per loop, giving \(\bigO(1)\) complexity within the loops.
\end{itemize}
\end{itemize}

Therefore we see that the algorithm has \(\bigO(1)\) time complexity, and even has the same in storage complexity. In particular our 32 bit example requires 163 opertaions, including assignments, comparrisons and calucluations. This means that the integer square root of any number up to 4294967295 can be calculated extremely quickly.

%SUB%
\subsection{Bisection Method}
\label{SUB_"Bisection Method for Roots"}
\theoremstyle{plain}
\newtheorem{Bisection Converges}{Proposition}[subsection]

The Bisection Method is a general method for approximating the zero, \(\alpha\), of a function, \(f\), on a bounded interval, \(I := [a,b]\), where \(f\) has the property \(f(x)f(y) < 0 \forall (x,y) \in [a,\alpha)\times(\alpha, b]\); we may assume, without loss of generality, that \(f(x) < 0 \forall x \in [a, \alpha]\).\\

The bisection method starts with initial bounds \(a_0 = a, b_0 = b\), where the initial approximation for the root is \(x_0 = \frac{1}{2}(a+b)\). We will consider pseudocode of the iteration process, that uses \(b_n - a_a < \tau\) or \(f(x_n) = 0\) as exit criteria. Here \(\tau\) is a tolerance threshold, and if the exit criteria is met it means that \(|x_n - \alpha| \le \frac{\tau}{2}\), while the other exit criteria means we have reached an exact solution.\\

%PCD%
\label{PCD_"General Bisection Method"}
\begin{lstlisting}[frame=single,mathescape,caption={General Bisection Method}]
  bisectionMethod($a \in \R, b \in (a, \infty), f \in \mathcal{C}[a,b], \tau \in \Rp$)
      $a_0 := a$
      $b_0 := b$
      $x_0 := \tfrac{1}{2}(a+b)$
	  $n := 0$
	  while $f(x_n) \neq 0$ AND $b_n - a_n > \tau$:
          if $f(x_n) < 0$:
              $a_{n+1} := x_n$
              $b_{n+1} := b_n$
          else:
              $a_{n+1} := a_n$
              $b_{n+1} := x_n$
          $n \mapsto n+1$
          $x_n := \tfrac{1}{2}(a_n + b_n)$
      return $x_n$
\end{lstlisting}\\
		
For our purposes we are trying to find the zero of \(f(x) = x^2 - N\), which is a strictly increasing function on \(\Rpz\). If \(N >= 1\), then \(\sqrt{N} \in [0, N]\), while \(N < 1 \implies \sqrt{N} \in [0, 1]\). It is obvious that our function has the required property, and thus we get the following method for finding the square root of \(N\):\\

%PCD%
\begin{lstlisting}[frame=single,mathescape,caption={Bisection Method for Square Roots},label={PCD_"Square Root Bisection Method"}]
  bisectionSquareRoot($N \in \Rpz, \tau \in \Rp$)
      $a_0 := 0$
      $b_0 := \max{1, N}$
      $x_0 := \tfrac{1}{2}(a_0 + b_0)$
      $n := 0$
      while $x_n^2 - N \neq 0$ AND $b_n - a_n > \tau$:
          if $x_n^2 - N < 0$:
              $a_{n+1} := x_n$
              $b_{n+1} := b_n$
          else:
              $a_{n+1} := a_n$
              $b_{n+1} := x_n$
          $n \mapsto n+1$
          $x_n := \tfrac{1}{2}(a_n + b_n)$
      return $x_n$
\end{lstlisting}\\

The implementation of this method is efficiently acheived in C using only addition, subtraction and multiplication by a constant. Before this method is implemented, however, we must first consider if and or when it converges to the correct answer. From an intuitive standpoint we would assume that if there is only one root in the interval, it would follow that we would converge to the root.

%THM%
\begin{Bisection Converges}
\label{THM_"Bisecton Converges"}
\(\lim_{n \to \infty} x_n = \sqrt{N}\) for Algorithm \ref{PCD_"Square Root Bisection Method"}
\end{Bisection Converges}
\begin{proof}
To prove this statement it suffices to prove that \(\sqrt{N} \in [a_n, b_n] \forall n \in \N\) and \(\lim_{n\to\infty} |x_n - \sqrt{N}| = 0\).\\

\textit{Claim 1:} \(\sqrt{N} \in [a_n, b_n] \forall n \in \N\)
\begin{subproof}\\
\(a_0 := 0 \implies a_0 \le \sqrt{N}\)\\
\(b_0 := \max\{1, N\} \implies b_0 \ge \sqrt{N}\)\\
Therefore it is obvious that \(\sqrt{N} \in [a_0, b_0]\)\\
Now suppose \(\sqrt{N} \in [a_n, b_n]\) for some \(n \in \N\)\\
It should be noted that \(a_n, b_n, x_n \in \Rpz \forall n \in \N\) as \(a_0, b_0 \in \Rpz\) and all the subsequent values are derived from these using only addition and multiplication by positive factors.\\
We then see that \(x_n := \frac{1}{2}(a_n + b_n)\), and we consider the two cases that \(x_n^2 - N \le 0\) or \(x_n^2 - N \ge 0\).\\
\begin{description}
\item[Case \(x_n^2 - N \le 0\):]\\
	\(a_{n+1} := x_n, b_{n+1} := b_n\)\\
	It is therefore obvious that \(\sqrt{N} \le b_{n+1}\).\\
	Now we see that \(x_n^2 - N \le 0 \implies x_n^2 \le N \implies x_n \le N\) as all the values are non-negative.\\
	Thus \(\sqrt{N} \in [a_{n+1}, b_{n+1}]\).\\
\item[Case \(x_n^2 - N \ge 0\):]\\
	\(a_{n+1} := a_n, b_{n+1} := x_n\)\\
	It is therefore obvious that \(\sqrt{N} \ge a_{n+1}\).\\
	Now we see that \(x_n^2 - N \ge 0 \implies x_n^2 \ge N \implies x_n \ge N\) as all the values are non-negative.\\
	Thus \(\sqrt{N} \in [a_{n+1}, b_{n+1}]\).\\
\end{description}
Hence \(\sqrt{N} \in [a_n, b_n] \implies \sqrt{N} \in [a_{n+1}, b_{n+1}] \forall n \in \N\)\\
As \(sqrt{N} \in [a_0, b_0]\) then we see that \(\sqrt{N} \in [a_n, b_n] \forall n \in \N\)
\end{subproof}\\

\textit{Claim 2:} \(\lim_{n\to\infty}|x_n - \sqrt{N}| = 0\)
\begin{subproof}\\
Let \(n \in \N\) be arbitrary.\\
As \(x_n := \frac{1}{2}(a_n + b_n)\) then we see that \(|a_n - x_n| = |b_n - x_n| = \frac{1}{2}(b_n - a_n)\).\\
Now as \(\sqrt{N} \in [a_n, b_n]\) it follows that \(|\sqrt{N} - x_n| \le \frac{1}{2}(b_n - a_n\).\\
As the modulas function is a mapping from \(\R\) to \(\Rpz\), it is clear that \(|\sqrt{N} - x_n|\) is bounded below by 0.\\
Now as for each \(n \in \N\), either \(a_{n+1} = x_n\) or \(b_{n+1} = x_n\), we see that \(b_{n+1} - a_{n+1} = \frac{1}{2}(b_n - a_n)\). Further we can see that \(b_n - a_n \ge 0 \forall n \in \N\) because \(b_n \ge a_n\).\\
Therefore the sequence of \(\frac{1}{2}(b_n - a_n)\) is a strictly decreasing sequence that is bounded below, by 0. Thus \(\lim_{n\to\infty} \frac{1}{2}(b_n - a_n) = 0\)\\ 
Therefore \(\lim_{n\to\infty} |x_n - \sqrt{N}| = \lim_{n\to\infty}\frac{1}{2}(b_n - a_n) = 0\)
\end{subproof}

By using our two claims above we see that \(\lim_{n\to\infty} x_n = \sqrt{N}\).
\end{proof}

The algorithm can be generalised to search for \(\sqrt{k}{N}\), where \(k \in [2,\infty) \cap \Z\). We can do this by using the integer power function discussed previously in section \ref{SEC#}. This gives the following algorithm:
 
%PCD%
\label{PCD_"Bisection Method for General Roots"}
\begin{lstlisting}[frame=single,mathescape,caption={Bisection Method for General Roots}]
  kRootBisectionMethod($N \in \Rpz, k \in [2, \infty) \cap \Z, \tau \in \Rp$)
      $a_0 := 0$
      $b_0 := \max{1, N}$
      $x_0 := \tfrac{1}{2}(a_0 + b_0)$
      $n := 0$
      while $\textrm{intPow}(x_n, k) - N \neq 0$ AND $b_n - a_n > \tau$:
          if $\textrm{intPow}(x_n, k) - N < 0$:
              $a_{n+1} := x_n$
              $b_{n+1} := b_n$
          else:
              $a_{n+1} := a_n$
              $b_{n+1} := x_n$
          $n \mapsto n+1$
          $x_n := \tfrac{1}{2}(a_n + b_n)$
      return $x_n$
\end{lstlisting}\\

The proof that this converges to the correct value is very similar to the proof for square roots.\\

We can now consider the accuracy that can be acheived by our algorithm, for our purposes we will be considering \(\sqrt{N}\), though the same applies for \(\sqrt{k}{N}\). We know that \(\sqrt{N} \in [a_n, b_n] \forall n \in \N\), and in particular we know that either \(\sqrt{N} \in [a_n, x_n]\) or \(\sqrt{N} \in [x_n, b_n] \forall n \in \N\); therefore we know that \(\epsilon_n := \left|x_n - \sqrt{N}\right| \le \tfrac{1}{2}(b_n - a_n) \forall n \in \N\). Then as we know that \(b_{n+1} - a_{n+1} = \tfrac{1}{2}(b_n - a_n)\), we know that \(\epsilon_n \le \tfrac{1}{2^n}(b_0 - a_0)\).\\

We can consider that \(\forall N \in \Rpz \exists (r,k) \in [0,1]\times\Z : N = r \cdot 10^{2k}\); using this we know that \(\sqrt{N} = \sqrt{r} \cdot 10^k\). As we have the fixed initial bounds of \(a_0 = 0\) and \(b_0 = 1\), then if we are finding \(\sqrt{r}\) we know that \(\epsilon_n \le \tfrac{1}{2^n} \forall n \in \N\). Hence we can calculate the precision of our current estimate beforehand for any \(n \in \N\), and thus we can guarantee \(d\) significant digits of accuracy for \(r \in [0, 1]\).\\

To get this accuracy must find \(n \in \N\) such that \(\epsilon_n \le \tfrac{1}{10^d}\), to acheive this we must find \(n \in \N\) such that \(2^n \ge 10^d\). For example the following table indicates the required \(n\), required for certain significant digits of accuracy.

\begin{center}
	\begin{tabular}{|p{3cm}|p{3cm}|}
	\hline
	\(d\) & \(n : 2^n \ge 10^n\)\\
	\hline
	1 & 0\\\hline
	5 & 15\\\hline
	10 & 30\\\hline
	20 & 64\\\hline
	50 & 163\\\hline
	100 & 329\\\hline
	\end{tabular}
\end{center}
\TODO{Finish up writeup of code implementation, testing and examination}

%SUB%
\subsection{Newton's Method for Square Roots}
\label{SUB_"Newton for Square Roots"}

\theoremstyle{plain}
\newtheorem{SRNM Right-hand Convergence}{Proposition}[subsection]
\newtheorem{SRNM NR1 and NR2}[SRNM Right-hand Convergence]{Proposition}
\newtheorem{SRNM NR3 for v3}[SRNM Right-hand Convergence]{Proposition}

If we consider $f(x) = x^2 - N$ then if $x^\ast$ is a solution to $f(x) = 0$ we see that $x^\ast = \sqrt{N}$. As $f'(x) = 2x$, then the Newton's Method, will give $x_{n+1} = x_n - \frac{x^2 - N}{2x}$, where $x_0$ is a given initial guess.\\

We can see that, in C, each iteration will calculate \codeinline{x = x - (x*x - N) / (2*x)}, which requires 5 operations; however if we re-arrange our equation, we instead get \(x_{n+1} = \frac{1}{2}\right(x_n + \frac{N}{x}\right)\). Implementing our new iterative formula we get \codeinline{x = 0.5 * (x + N/x)}, which now uses only 3 operations.\\

We can then use the following pseudocode as the basis of our implementaions of the Newton-Raphson Method for Square Roots:

%PCD%
\label{PCD_"Newton Square Root Basic"}
\begin{lstlisting}[frame=single,mathescape,caption={Basic Newton Method for Square Root}]
  NewtonSquareRoot($N \in \R, x_0 \in \R, \tau \in (0,1)$):
      $n := 0$
      loop:
          $x_{n+1} := \tfrac{1}{2}(x_n + \tfrac{N}{x_n})$
          $\delta_n := |x_{n+1} - x_n|$
          if $\delta_n \leq \tau$:
              return $x_{n+1}$
          $n \mapsto n + 1$
\end{lstlisting}

Next we want to consider our initial estimate \(x_0\); it is prudent to first consider when our initial estimate will converge to the correct root. By looking at a graph of the function, and in particular the tangents to the curve, it would seem reasonable to wonder if \(\lim_{n\to\infty} x_n = \sqrt{N}\).

%THM%
\begin{SRNM Right-hand Convergence}
\label{THM_"SRNM Right-Hand Convergence"}
If \(x_0 \in \right(\sqrt{N}, \infty\right)\) and \(\left\{x_n : n\in\N\right\}\) is a sequnence of approximations of \(\sqrt{N}\) found via the Newton-Raphson Method, as detailed above, then:
\[\lim_{n\to\infty} x_n = \sqrt{N}\]
\end{SRNM Right-hand Convergence}

%PRF%
\begin{proof}
Suppose \(x_n > \sqrt{N}\), then
\begin{align*}
	x_{n+1} &= \frac{1}{2}\left(x_n + \frac{N}{x_n}\right)\\
		  &< \frac{1}{2}\left(x_n + \frac{N}{\sqrt{N}}\right) 
		  		&\mathrm{as } \sqrt{N} < x_n \implies \frac{1}{x_n} <
				\frac{1}{\sqrt{N}}\\
		  &= \frac{1}{2}\left(x_n + \sqrt{N}\right)\\
		  &< \frac{1}{2}(2x_n)\\
		  &= x_n
\end{align*}
Therefore we see that \(\left\{x_k : k \in [n, \infty) \cap \Z\right\}\) is a strictly decreasing sequence.\\
Now suppose that \(x_n \ge \sqrt{N}\) and then, for a contradiction, assume that \(x_{n+1} < \sqrt{N}\). We then see that:
\begin{align*}
	& \frac{1}{2}\left(x_n + \frac{N}{x_n}\right) < \sqrt{N}\\
	\implies & x_n + \frac{N}{x_n} < 2\sqrt{N}\\
	\implies & x_n^2 + N < 2\sqrt{N}x_n\\
	\implies & x_n^2 - 2\sqrt{N}x_n + N < 0\\
	\implies & \left(x_n - \sqrt{N}\right)^2 < 0
\end{align*}
This is a contradiction as \(x_n, \sqrt{n} \in \R \implies \left(x_n - \sqrt{N}\right)^2 \ge 0\).\\
Therefore \(x_n \ge \sqrt{N} \implies x_{n+1} \ge \sqrt{N}\).\\
Hence if \(x_0 > \sqrt{N}\), then it follows that \(\{x_n : n \in \N\}\) is a strictly decreasing sequence that is bounded below. Therefore by an elementary result from limit theory, we see that \(\lim_{n\to\inft} x_n = \sqrt{N}\).
\end{proof}\\

The most obvious choice for \(x_0\) would be \(N\), but we see that \(N \in (0,1)\), then \(N < \sqrt{N}\). In this case, we could choose \(x_0 = 1\) for the case that \(N \in (0,1)\). Therefore we can choose 
\[x_0 := \left\{\begin{array}{lcl}N &: &N \in\left(1,\infty\right)\\1 &: &N \in (0,1)\end{array}\right.\]\\

In our choice of \(x_0\), we have so far left out the cases where \(N \in \{0, 1}\). In both of these case we already know the correct answer, namely \(\sqrt{N} = N\) provided \(N \in {0, 1}\). Therefore we can exclude them from our calculations, as we can pre-asses the value of \(N\), simply returning the correct answer if one of these cases is encountered.\\

This then leads to an updated version of the above pseudocode:\\

%PCD%
\label{PCD_"Newton Square Root v1"}
\begin{lstlisting}[frame=single,mathescape,caption={Basic Newton Method for Square Root}]
  NewtonSquareRoot($N \in \Rpz, \tau \in (0,1)$):
      if $N \in \{0, 1\}$:
          return $N$
      if $N > 1$:
          $x_0 := N$
      else:
          $x_0 := 1$
      $n := 0$
      loop:
          $x_{n+1} := \tfrac{1}{2}(x_n + \tfrac{N}{x_n})$
          $\delta_n := |x_{n+1} - x_n|$
          if $\delta_n \leq \tau$:
              return $x_{n+1}$
          $n \mapsto n + 1$
\end{lstlisting}

\TODO{Write up examination and implementation of this pseudocode}\\

An alternative would be to use the integer square root method discussed in Section \ref{SUB_"Digit by Digit Method"} to improve our initial choice of \(x_0\). We will start by showing, that for intervals \(I \subset \Rp\), the first two criteria for quadratic convergence of the Newton Raphson method are met.

%THM%
\begin{SRNM NR1 and NR2}
\label{THM_"SRNM NR1 and NR2}
If \(I \subset \Rp\) then \(NR_1\) and \(NR_2\) are satisfied for \(f(x) = x^2 - N\)
\end{SRNM NR1 and NR2}

%PRF%
\begin{proof}
\(f(x) = x^2 - N \implies f'(x) = 2x \implies f''(x) = 2\)\\
Now as \(x \in \Rp \forall x \in I\), then it is obvious that \(f'(x) > 0\)\\
Therefore \(f'(x) \neq 0 \forall x \in I\), and so \(NR_1\) is satisfied.\\
As \(f''(x)\) is a constant function, then it is continuous on all of \(\R\).\\
Hence \(f''(x)\) is continuous \(\forall x \in I\) and so \(NR_2\) is satisfied.
\end{proof}

Now the integer square root function will always produce a root that is at most a distance of \(1\) from \(\sqrt{N}\); therefore we can consider \(I = [\sqrt{N} - 1, \sqrt{N} + 1]\). Now if \(N \le 1\), then \(I \seubset \Rp\) and so we cannot guarantee the satisfaction of \(NR_1\). Therefore we can proceed with our analysis of the case that \(N > 1\).\\

If \(N > 1\) we need to find when we can satisfy \(NR_3\). First, we remember that \(M := \sup{\left|\tfrac{f''(x)}{f'(x)}\right| : x \in I}\) and \(\epsilon_0 := \left|x_0 - \sqrt(N)\right|\). Then to satisfy \(NR_3\), we must have that \(M\epsilon_0 < 1\).\\

We can guarantee that \(\epsilon_0 \le 1\) because \(x_0 \in I\) from the integer square root algortihm; therefore it suffices to find the situation where \(M < 1\). As both \(f'\) and \(f''\) are continuous and non-zero on \(I\) it follows that \(M = \sup{x^{-1} : x \in I} = (\sqrt{N} - 1)^{-1}\). We then see that:
\begin{displaymath}
	\begin{align*}
		M < 1 &\iff \sqrt{N} - 1 > 1\\
			  &\iff \sqrt{N} > 2\\
			  &\iff N > 4
	\end{align*}
\end{displaymath}

Therefore we can get the following new choice for \(x_0\), and thus new pseudocode:
\begin{displaymath}
	x_0 := \left\{\begin{array}{lcl}
		1 &: &N \in (0,1)\\
		N &: &N \in (1,4]\\
		intSqrt(N) &: &N \in (4, \infty)
	\end{array}\right.
\end{displaymath}

%PCD%
\label{PCD_"Newton Square Root v2"}
\begin{lstlisting}[frame=single,mathescape,caption={Basic Newton Method for Square Root}]
  NewtonSquareRoot($N \in \Rpz, \tau \in (0,1)$):
      if $N \in \{0, 1\}$:
          return $N$
      if $N < 1$:
          $x_0 := 1$
      else:
          if $N \le 4$:
              $x_0 := N$
          else:
              $x_0 := $ IntSqrt($N$)
      $n := 0$
      loop:
          $x_{n+1} := \tfrac{1}{2}(x_n + \tfrac{N}{x_n})$
          $\delta_n := |x_{n+1} - x_n|$
          if $\delta_n \leq \tau$:
              return $x_{n+1}$
          $n \mapsto n + 1$
\end{lstlisting}

\TODO{Write up examination of different versions tried, such as using \(x_0 = N\), etc...}\\

If we consider any \(N \in \Rpz\), then \(\exists a \in \left[\frac{1}{2}, 1\right), b \in \Z : N = a \times 2^b\). Finding this value would be a hard as finding the logarithm of \(N\) base 2, but due to the representation of numbers within C, both standard C and MPFR have functions that allow us to extract these two values with minimal computational expenditure.\\

This helps as we can then narrow our problem, to only finding \(\sqrt{a} : a \in \left[\frac{1}{2}, 1\right)\), and then calculating 
\begin{displaymath}
	\sqrt{N} = \sqrt{a} \times 2^{\left\lfloor \frac{b}{2} \right\rfloor} \times \alpha \ \mathrm{where}\  
	\alpha = \left\{
		\begin{array}{lcl}
			1 & : & b \in 2\Z \\
			\sqrt{2} & : & b \in \Zp\setminus2\Z \\
			\frac{1}{sqrt{2}} & : & b \in \Zn\setminus\2\Z
		\end{array}\right.
\end{displaymath}

We then get the following algorithm, which implements this:

%PCD%
\begin{lstlisting}[frame=single,mathescape,caption={Newton Method for Square Root v3},label={PCD_"Newton Method for Square Root v3"}]
  NewtonSquareRoot($N \in \Rpz, \tau \in (0,1)$):
      Let $(a, b) :\in \left[\tfrac{1}{2}, 1\right)\times\Z$ s.t. $N = a*2^b$
      $x_0 := 1$
      if $b \equiv 0 \textrm{mod}\ 2$:
          $\alpha := 1$
      else:
          if $b > 0$:
              $\alpha := \sqrt{2}$
          else:
              $\alpha := \tfrac{1}{\sqrt{2}}$
      $n := 0$
      loop:
          $x_{n+1} := \tfrac{1}{2}(x_n + \tfrac{a}{x_n})$
          $\delta_n := |x_{n+1} - x_n|$
          if $\delta_n \leq \tau$:
              return $\alpha\cdot x_{n+1} \cdot 2^{\left\lfloor\frac{b}{2}\right\rceil}$
          $n \mapsto n + 1$
\end{lstlisting}

We must first consider the fact that the algorithm requires the pre-calculation of both \(\sqrt{2}\) and \(\tfrac{1}{\sqrt{2}}\), to be able to calculate all values. However, it turns out we can use the algorithm itself to generate these values as \(2 = \tfrac{1}{2} \cdot 2^2\), and as the exponent of 2 is even then the algorithm does not require \(\sqrt{2}\) for this computation. Similarly \(\tfrac{1}{2} = \tfrac{1}{2} \cdot 2^0\), which again is an even exponent. We can thus run our algorithm to find an arbitrarily accurate values for \(\sqrt{2}\) and \(\tfrac{1}{\sqrt{2}}\) to allow us to run the algorithm for other values.\\

With this observation can then consider \(N \in \left[\tfrac{1}{2}, 1\right)\). As this is a small range and, as per our previous algorithm, we use an initial guess of \(x_0 = 1\), then we can prove that our algorithm will converge quadratically to \(\sqrt{N}\).

%THM%
\begin{SRNM NR3 for v3}
\label{THM_"NR3 for v3"}
Algorithm \ref{PCD_"Newton Method for Square Root v3"}, satisfies the criteria of Theorem \ref{THM_"Quad Conv Newton"}, and thus has quadratic convergence to \(\sqrt{N}\).
\end{SRNM NR3 for v3}
\begin{proof}
To fulfill the criteria of Theorem \ref{THM_"Quad Conv Newton"}, we must find and interval \(I := [\sqrt{N}-r, \sqrt{N} + r]\) for some \(r \ge \epsilon_0\).\\

Consider \(\epsilon_0 = |\sqrt{N} - x_0| = 1 - \sqrt{N}\). We see that as \(N \ge \frac{1}{2}\) then \(\sqrt{N} \ge \sqrt{2}^-1\), and thus \(\epsilon_0 \le 1 - \sqrt{2}^-1\). Let us have \(r := 1 - \frac{1}{\sqrt{2}}\), and \(I\) as defined above.\\

If we look at the lower bound of \(I\), then we see that:
\begin{displaymath}
\begin{align*}
\sqrt{N} - r &\ge \frac{1}{\sqrt{2}} - (1 - \frac{1}{\sqrt{2}})\\
	&= \frac{2}{\sqrt{2}} - 1\\
	&= \sqrt{2} - 1 \\
	&> 0
\end{align*}
\end{displaymath}

Therefore we see that \(I \subset \Rp\), and so by Proposition \ref{THM_"SRNM NR1 and NR2} we get that \(\mathrm{NR}_1\) and \(\mathrm{NR}_2\) ar satisfied. It then remains to show that \(\mathrm{NR}_3\) is satisfied on \(I\).\\

Now by the definition in Theorem \ref{THM_"Quad Conv Newton"}, we have that \(M = \sup\left\{\frac{1}{2}\left|\frac{f''(x)}{f'(y)}\right| : x, y \in I\right\}\). We know that \(I\) is bounded, \(f''(x) = 2\) and \(f'(x) = 2x\) meaning that \(\frac{1}{2}\left|\frac{f''(x)}{f'(y)}\right| = \frac{1}{f'(x)}\) as \(x \in \Rp\).\\ 

Therefore our problem is reduced to finding \(\max\left\{\frac{1}{2x} : x \in I\right\}\), which is equivalent to finding \(\min\{x : x \in I\} = \sqrt{N} - r\). Therefore by passing this information back up the chain we get that \[M = \frac{1}{2(\sqrt{N} - r)}\]\\

Then we see that:
\begin{displaymath}
\begin{align*}
M\epsilon_0 &= \frac{1 - \sqrt{N}}{2(\sqrt{N} - r)}\\
	&\le \frac{1 - \frac{1}{\sqrt{2}}}{2(\sqrt{N} - r)} 
		& \textrm{as } \sqrt{N} \ge \frac{1}{\sqrt{2}}\\
	&\le \frac{1 - \frac{1}{\sqrt{2}}}{2(\frac{1}{\sqrt{2}}-r)}
		& \textrm{as } \sqrt{N} \ge \frac{1}{\sqrt{2}}\\
	&= \frac{1 - \frac{1}{\sqrt{2}}}{2(\frac{2}{\sqrt{2}} - 1)}\\
	&= \frac{1 - \frac{1}{\sqrt{2}}}{2\sqrt{2}(1-\frac{1}{\sqrt{2}})}\\
	&= \frac{1}{2\sqrt{2}}\\
	&< 1 & \textrm{as } 2\sqrt{2} > 1
\end{align*}
\end{displaymath}

As we have confirmed that \(M\epsilon_0 < 1\), then we have confirmed that \(\mathrm{NR}_3\) is satisfied on \(I\), and so the algorithm converges quadratically to the desired root.
\end{proof}

Using the previous proposition we can, similar to our previous methods, consider how many iterations would be needed to reach a required tolerance. To start we consider that, as mentioned in the proof or Theorem \ref{THM_"Quad Conv Newton"}, that \(\epsilon_n \le (M\epsilon_0)^{2^n - 1}\epsilon_0\).\\

We know that \(M\epsilon_0 \le \frac{1}{2\sqrt{2}}\) and that \(\epsilon_0 \le 1 - \frac{1}{\sqrt{2}}\), giving:
\[\epsilon_n \le \left(\frac{1}{2\sqrt{2}}\right)^{2^n - 1}\left(1 - \frac{1}{\sqrt{2}}\right)\]

Thus if we want to acheive a tolerance of \(\epsilon_n \le \tau\), then it suffices to find \(n \in \N_0\) such that:
\[\left(\frac{1}{2\sqrt{2}}\right)^{2^n - 1} \le \tau\]

Then,
\[(2^n - 1)\log\left(\frac{1}{2\sqrt{2}}\right) \le \log\left(\frac{\tau}{1 - \frac{1}{\sqrt{2}}}\right)\]

By noting that \(\log(\frac{1}{a}) = - \log(a)\), then we get
\[(1-2^n)\log(2\sqrt{2}) \le \log\left(\frac{\tau}{1 - \frac{1}{\sqrt{2}}}\right)\]

Once this is rearranged we get the following inequality:
\[2^n \ge \frac{\log\left(\frac{2(\sqrt{2} - 1)}{\tau}\right)}{\log(2\sqrt{2})}\]

By taking logarithms again and re-arranging we get that
\[n \ge \frac{\log\left(\frac{\log\left(\frac{2(\sqrt{2} - 1)}{\tau}\right)}{\log(2\sqrt{2})}\right)}{\log(2)} = \log_2\left(\log_{2\sqrt{2}}\left(2\frac{\sqrt{2} - 1}{\tau}\right)\right)\]

Now for an example, suppose we want to know how many iterations we need to perform to find \(\sqrt{N}\) to within 10 decimal places, i.e. \(\tau = 10^-10 = 0.0000000001\). We remember that \(\sqrt{N} \in [\frac{1}{2}, 1)\), and then we will apply transformations to this value afterwards, therefore this is equivalent to finding 10 significant digits of accuracy for our square root (ignoring any loss of accuracy that may arrise from multiplications afterwards).\\

Now in this case we want to find \(n \in \N\) such that \(n \ge log_2(log_{2\sqrt{2}}(2\cdot10^{10}(\sqrt{2}-1)))\). Using Wolfram Alpha to calculate this value we get that we need \(n \ge 4.457144...\) and so we can take \(n = 5\). This means that we could modify our algorithm and implementation to do 5 fixed iterations of Newton's Method to guarantee at least 10 decimal places of accuracy.\\

In terms of efficiency versus accuracy tradeoff modifying the problem thus would improve it's efficiency by removing, now unneccesary, calculation and comparrison of \(\delta_n\) at each stage. However this does need a fixed guaranteed accuracy, and therefore such a program would no longer be suitable if we needed to calculate a square root accurate to 15 decimal places.\\

Below is a table that lists the minimum \(n \in \N\) such that \(n\) satisfies our inequality, where our tolerance is \(10^k\) for some \(k \in \N\). This will give us the maximum number of iterations that must be performed for the required accuracy.

\begin{center}
\begin{tabular}{|p{3cm}|p{3cm}|}
\hline
\(k : \tau = 10^k\) & \(n\)\\\hline
5 & 4 \\\hline
10 & 5 \\\hline
100 & 8 \\\hline
1,000 & 12 \\\hline
1,000,000 & 22\\\hline
\end{tabular}
\end{center}

%SUB%
\subsection{Newton's Inverse Square Root Method}
\label{SUB_"Newton's Inverse Square Root Method"}

\theoremstyle{plain}
\newtheorem{Inv Sqrt Quad Conv}{Proposition}[subsection]

As discussed in Section \ref{#SEC#}, computers are more efficient at multiplication over division. We would therefore prefer to find a way of utilising Neton's Method without having to perform any costly division operations.\\

If we consider \(f(x) = N - \frac{1}{x^2}\) then if \(x^\ast\) is a solution to \(f(x) = 0\) we see that \(x^\ast = \frac{1}{\sqrt{N}}\). As \(f'(x) = \frac{2}{x^3}\), then the Newton's Method, will give \[x_{n+1} = x_n - \frac{N - \frac{1}{x_n^2}}{\frac{2}{x_n^3}} = x_n\left(\frac{3}{2} - \frac{N}{2}x_n^2\right)\] where \(x_0\) is a given initial guess. As can be seen this algorithm requires no division if we multiply by real constants rather than the division implied above.\\

We can then consider that, similar to Algorithm \ref{PCD_"Newton Method for Square Root v3"}, any \(N\) can be represented as \(a \cdot 2^b\) where \(a \in \left[\tfrac{1}{2}, 1\right)\). This will, again allow us to narrow our problem to a known range of values, by using the following transormations.
\begin{displaymath}
\begin{align*}
N = a \cdot 2^b &\implies \tfrac{1}{N} = \tfrac{1}{a} \cdot 2^{-b}\\
	&\implies \tfrac{1}{\sqrt{N}} = \tfrac{1}{a}\cdot2^{\lfloor\frac{-b}{2}\rceil} \cdot \alpha
		&\alpha := \left\{
			\begin{array}{lcl}
				1 & : & b \equiv 0 \mod 2\\
				\sqrt{2} & : & b \equiv 1 \mod 2, b \in \Zn\\
				\frac{1}{\sqrt{2}} & : & b \equiv 1 \mod 2, b \in \Zp\\
			\end{array}\right.\\
	& \implies \sqrt{N} = N \cdot \tfrac{1}{\sqrt{a}} \cdot 2^{\lfloor\frac{-b}{2}\rceil} \cdot \alpha
\end{align*}
\end{displaymath}

Therefore we only need to calculate inverse square roots for values of \(N\) in the range \([\tfrac{1}{2}, 1)\). Thus giving us the following algorithm:\\

%PCD%
\begin{lstlisting}[frame=single,mathescape,caption={Newton Inverse Square Root Method},label={PCD_"Newton Inverse Square Root"}]
  NewtonSquareRoot($N \in \Rpz, \tau \in (0,1)$):
      Let $(a, b) :\in \left[\tfrac{1}{2}, 1\right)\times\Z$ s.t. $N = a*2^b$
      $x_0 := 1$
      if $b \equiv 0 \textrm{mod}\ 2$:
          $\alpha := 1$
      else:
          if $b > 0$:
              $\alpha := \tfrac{1}{\sqrt{2}}$
          else:
              $\alpha := \sqrt{2}$
      $n := 0$
      loop:
          $x_{n+1} := x_n(\tfrac{3}{2} + \tfrac{a}{2}x_n^2)$
          $\delta_n := |x_{n+1} - x_n|$
          if $\delta_n \leq \tau$:
              return $N\cdot\alpha\cdot x_{n+1} \cdot 2^{\left\lfloor\frac{-b}{2}\right\rceil}$
          $n \mapsto n + 1$
\end{lstlisting}

With this method we can once again consider it's convergence properties, in particular does it satisfy the criteria for quadratic convergence in Theorem \ref{THM_"Quad Conv Newton"}.

%THM%
\begin{Inv Sqrt Quad Conv}
\label{THM_"Inv Sqrt Quad Conv"}
Algorithm \ref{PCD_"Newton Inverse Square Root"} satisfies the criteria of Theorem \ref{THM_"Quad Conv Newton"}, and thus has quadratic convergence to \(\sqrt{N}\).
\end{Inv Sqrt Quad Conv}
\begin{proof}
We know that we only need to consider \(N \in [\frac{1}{2}, 1)\), and therefore \(\sqrt{N}^{-1} \in (1, \sqrt{2}]\). Also \(x_0 = 1\) and so we see that 
\[\epsilon_0 = |x_0 - \sqrt{N}^{-1}| = \sqrt{N}^{-1} - x_0 \le \sqrt{2} - 1\]

Now let \(r := \epsilon_0 = \sqrt{N} - 1\) and \(I := [\sqrt{N}^{-1} - r, \sqrt{N}^{-1}]\). If we consider the lower bound of I we see that \(\sqrt{N}^{-1} - (\sqrt{N}^{-1} - 1) = 1\), and in particular \(0 \notin I\).\\

Next we know that \(f(x) = N - x^{-2}\), and therefore we get \(f'(x) = 2x^{-3}\), \(f''(x) = -6x^{-4}\). It is obvious that \(\nexists x \in \R : f'(x) = 0\), which means that \(f'(x) \neq 0 \forall x \in I\) and so \(\mathrm{NR}_1\) is satisfied. Also as \(f''\) is only discontinuous at \(x = 0\) and \(0 \notin I\), then \(f''(x)\) is continuous \(\forall x \in I\), meaning this satisfies \(\mathrm{NR}_2\).\\

Now \(M = \sup\left\{\tfrac{1}{2}\left|\frac{2x^3}{6y^4}\right| : x, y \in I\rifht\}\), we can simplify the function we are trying to minimise to get \(\tfrac{1}{6}\frac{x^3}{y^4}\). It is obvious that in order to maximise this function we should find the largest possibe \(x\) and smallest possible \(y\), as both are positive. Hence by taking \(x = \sqrt{N}^{-1} + r\) and \(y = 1\), then \(M = \frac{1}{6}(2\sqrt{N}^{-1} - 1)^3 \le \frac{1}{6}(2\sqrt{2} - 1)^3\).\\

Now we consider \(M\epsilon_0\):\\

\begin{displaymath}
\begin{align*}
	M\epsilon_0 &=\frac{1}{6}(2\sqrt{N}^{-1} - 1)^3(\sqrt{N} - 1)\\
		&\le \frac{1}{6}(2\sqrt{2} - 1)^3(\sqrt{2} - 1)\\
		&\approx 0.42199376\ldots\\
		&< 1
\end{align*}
\end{displaymath}

Therefore as \(M\epsilon_0 < 1\) we have satisfied \(\mathrm{NR}_3\), and as such we have quadratic convergence of our method to \(\sqrt{N}^{-1}\).
\end{proof}

We now have two methods that converge quadratically to  
 
%SUB%
\subsection{Methods for other roots}
\TODO{Fill this out later}


\section{Logarithms and Exponentials}
\subsection{Taylor Series Method}
\subsection{Hyperbolic Series Method}
\subsection{CORDIC}
\TODO{Fill this out with stuff}



\section{Preliminary References}
http://math.exeter.edu/rparris/peanut/cordic.pdf \\
Inside your Calculator by Gerald R Rising \\
Wolfram Alpha \\

\appendix
\section{Code}
\lstset{basicstyle=\ttfamily,
		language=C,
		backgroundcolor=\color{cBg},
		basicstyle=\footnotesize,
		commentstyle=\color{cCm},
		frame=L,
		keywordstyle=\color{cKw},
		showstringspaces=false,
		stringstyle=\color{cSt},
		tabsize=2,
		mathescape=false}
\renewcommand{\thelstlisting}{}
\renewcommand{\lstlistingname}{File}

In this appendix I list the entirety of the code which implement the algorithms discussed in the body of this document. The entirety of the codebase, as well as LaTeX files related to this document can be found on GitHub at \url{https://github.com/Ybrad/Year-4-Project}.

\subsection{General Code}
\label{SUB_"General Code"}

\\General Utilities File:
\lstinputlisting[caption={utilities.c}]{../Code/utilities.c}

\\Trigonometric Utilities File:
\lstinputlisting[caption={trig\_utilities.c}]{../Code/trig_utilities.c}

\\Header Files for Utilities:
\lstinputlisting[caption={utilities.h}]{../Code/utilities.h}
\lstinputlisting[caption={trig\_utilities.h}]{../Code/trig_utilities.h}
\lstinputlisting[caption={log\_exp\_utilities.h}]{../Code/log_exp_utilities.h}

\\Makefile for the project:
\lstinputlisting[caption={makefile},language=make]{../Code/makefile}

\subsection{Square Root Code}
\label{SUB_"Square Root Code"}

\\Code for Exact Square Root Metods:
\lstinputlisting[caption={exact\_root.c}]{../Code/exact_root.c}

\\Code for the Bisection Methods:
\lstinputlisting[caption={bisect\_root.c}]{../Code/bisect_root.c}

\\Code for Newton Square Root Methods:
\lstinputlisting[caption={newton\_root.c}]{../Code/newton_sqrt.c}

\\Code for Newton Inverse Square Root Methods:
\lstinputlisting[caption={newton\_inv\_sqrt.c}]{../Code/newton_inv_sqrt.c}

\\Header files for Square Root Code:
\lstinputlisting[caption={exact\_root.h}]{../Code/exact_root.h}
\lstinputlisting[caption={bisect\_root.h}]{../Code/bisect_root.h}
\lstinputlisting[caption={newton\_root.h}]{../Code/newton_sqrt.h}
\lstinputlisting[caption={newton\_inv\_sqrt.h}]{../Code/newton_inv_sqrt.h}

\subsection{Trigonometric Code}
\label{SUB_"Trigonometric Code"}

\\Code for Geometric Trigonometric Functions:
\lstinputlisting[caption={geometric\_trig.c}]{../Code/geometric_trig.c}

\\Code for Geometric Inverse Trigonometric Functions:
\lstinputlisting[caption={geometric\_inv\_trig.c}]{../Code/geometric_inv_trig.c}

\\Code for Taylor Trigonometric Functions:
\lstinputlisting[caption={taylor\_trig.c}]{../Code/taylor_trig.c}

\\Code for Taylor Inverse Trigonometric Functions:
\lstinputlisting[caption={taylor\_inv\_trig.c}]{../Code/taylor_inv_trig.c}

\\Code for CORDIC Functions:
\lstinputlisting[caption={cordic\_trig.c}]{../Code/cordic_trig.c}

\\Header files for Trigonometric Functions:
\lstinputlisting[caption={geometric\_trig.h}]{../Code/geometric_trig.h}
\lstinputlisting[caption={geometric\_inv\_trig.h}]{../Code/geometric_inv_trig.h}
\lstinputlisting[caption={taylor\_trig.h}]{../Code/taylor_trig.h}
\lstinputlisting[caption={taylor\_inv\_trig.h}]{../Code/taylor_inv_trig.h}
\lstinputlisting[caption={cordic\_trig.h}]{../Code/cordic_trig.h}

\subsection{Exponential and Logarithm Code}
\label{SUB_"Exponential and Logarithm Code"}

\\Code for Integer Exponentiation:
\lstinputlisting[caption={int\_exp.c}]{../Code/int_exp.c}

\\Code for Taylor Exponentials and Logarithms:
\lstinputlisting[caption={taylor\_exp\_log.c}]{../Code/taylor_exp_log.c}

\\Code for Hyperbolic Logarithms:
\lstinputlisting[caption={hyperbolic\_log.c}]{../Code/hyperbolic_log.c}

\\Code for Continued Fraction Exponentials:
\lstinputlisting[caption={cont\_frac\_exp.c}]{../Code/cont_frac_exp.c}

\\Header Files for Exponential and Logarithmic Functions:
\lstinputlisting[caption={int\_exp.h}]{../Code/int_exp.h}
\lstinputlisting[caption={taylor\_exp\_log.h}]{../Code/taylor_exp_log.h}
\lstinputlisting[caption={hyperbolic\_log.h}]{../Code/hyperbolic_log.h}
\lstinputlisting[caption={cont\_frac\_exp.h}]{../Code/cont_frac_exp.h}


\end{document}
